\documentclass[11pt,a4paper]{article}

% ---------- Page geometry ----------
\usepackage[a4paper,margin=1in]{geometry}

% ---------- Fonts & microtypography ----------
\usepackage[T1]{fontenc}
\usepackage[utf8]{inputenc}
\usepackage{lmodern}
\usepackage{microtype}
\usepackage{xcolor}

% ---------- Math ----------
\usepackage{amsmath,amssymb,amsfonts,bm} % amsmath needed for subequations
% Math niceties
\usepackage{siunitx}
\sisetup{detect-all}
\usepackage{mathtools}

% ⟨...⟩ paired delimiters (renamed to avoid clash with siunitx's \ang)
\DeclarePairedDelimiter\angbrak{\langle}{\rangle}
% handy auto-sizing wrapper (use \braket{...} for ⟨...⟩)
\newcommand{\braket}[1]{\angbrak*{#1}}

\DeclareMathOperator{\Var}{Var}
% argmin typeset as "arg min", with limits beneath in display math
\DeclareMathOperator*{\argmin}{arg\,min}
\newcommand{\dd}{\mathrm{d}}
\newcommand{\EE}{\mathbb{E}}
\newcommand{\RR}{\mathbb{R}}
\newcommand{\defeq}{\vcentcolon=}

% ---------- Tables ----------
\usepackage{booktabs}
\usepackage{array}
\usepackage{tabularx}
\usepackage{longtable}
\renewcommand{\arraystretch}{1.15}
\setlength{\LTpre}{0pt}
\setlength{\LTpost}{0pt}

% Column types for ragged/centered p-columns (used by longtable/tabularx)
\usepackage{ragged2e}
\newcolumntype{L}[1]{>{\RaggedRight\arraybackslash}p{#1}}
\newcolumntype{C}[1]{>{\Centering\arraybackslash}p{#1}}
\newcolumntype{R}[1]{>{\RaggedLeft\arraybackslash}p{#1}}

% ---------- Figures / captions ----------
% load caption ONCE with all options you want
\usepackage[font=small,labelfont=bf,labelsep=period]{caption}
\usepackage{graphicx}
\usepackage{float}        % for [H]
\usepackage{capt-of}      % if \captionof is needed
\usepackage{subcaption}   % sub-figures and sub-captions

% ---------- Lists ----------
\usepackage{enumitem}
\setlist{leftmargin=*,itemsep=0.25em,topsep=0.3em}

% ---------- Floats & layout helpers ----------
\usepackage{needspace}

% ---------- Abstract formatting ----------
% (ragged2e already loaded above for column types)
\renewenvironment{abstract}{
  \begin{center}\normalfont\bfseries Abstract\end{center}
  \noindent\justifying
}{\par\vspace{1em}}

% ---------- Paragraph rhythm (match Paper I) ----------
\setlength{\parskip}{0pt}
\setlength{\parindent}{1.5em}

% ---------- Series numbering: per-paper only (N.k), never per section ----------
\usepackage{chngcntr}
\counterwithout{equation}{section}
\counterwithout{figure}{section}
\counterwithout{table}{section}

% ---------- Citations ----------
\usepackage{cite}
% \usepackage[round,authoryear]{natbib} % (leave commented unless you switch styles)

% ---------- Hyperlinks (load LAST) ----------
\usepackage[unicode,psdextra,hidelinks]{hyperref}

% ---------- Page footer ----------
\usepackage{fancyhdr}
\usepackage{url} % ensure URL formatting

\pagestyle{fancy}
\fancyhf{} % clear headers/footers

% ---- Two-line footer: large page number + single-line copyright ----
\fancyfoot[C]{%
  \normalsize Page \thepage\\[8pt]%
  \scriptsize
  \makebox[\textwidth][c]{%
    \textcopyright~2025\ Stuart E.\,N.\ Hiles.\ UIF Series v1.0 (Oct 2025).\ CC BY-NC 4.0 \(\) %
    \url{https://github.com/stuart-hiles/UIF}%
  }%
}

% Clean, rule-free footer
\renewcommand{\headrulewidth}{0pt}
\renewcommand{\footrulewidth}{0pt}

% Add bottom margin space so footer doesn’t collide with text
\setlength{\footskip}{30pt}


% ---------- UIF series macros (defined ONCE here) ----------
\makeatletter
\newcommand{\UIFpaper}[1]{%
  \def\UIFprefix{#1}%
  \renewcommand{\theequation}{\UIFprefix.\arabic{equation}}%
  \renewcommand{\theparentequation}{\UIFprefix.\arabic{equation}}%
  \renewcommand{\thefigure}{\UIFprefix.\arabic{figure}}%
  \renewcommand{\thetable}{\UIFprefix.\arabic{table}}%
  \setcounter{equation}{0}\setcounter{figure}{0}\setcounter{table}{0}%
}
\newcommand{\UIFbib}[1]{%
  \bibliographystyle{unsrt}%
  \bibliography{bib/#1}%
}
\makeatother

% Use \UIFmetadata{Title}{Author}{Subject/Keywords}
\newcommand{\UIFmetadata}[3]{%
  \hypersetup{%
    pdftitle    = {#1},
    pdfauthor   = {#2},
    pdfsubject  = {#3},
    pdfkeywords = {#3}
  }%
}

% If you do NOT want keywords printed on the page, make \keywords a no-op:
\newcommand{\keywords}[1]{}
% (If you *do* want them printed below the title, comment the line above and use:)
% \renewcommand{\keywords}[1]{\vspace{0.6em}\noindent\textbf{Keywords: }#1}

\begin{document}

% --- Choose which paper to compile ---
% ===== UIF Paper I — Core Theory =====
% Numbering: Paper 1 → (1.x) equations, figures, tables
\UIFpaper{1}

\UIFmetadata{The Unifying Information Field (UIF) I — Core Theory}
            {Stuart E. N. Hiles}
            {UIF I — Core Theory}

\hypersetup{
  pdftitle={The Unifying Information Field (UIF) Paper I — Core Theory},
  pdfauthor={Stuart E. N. Hiles},
  pdfsubject={Unifying Information Field; informational physics; coherence; recursion; agency},
  pdfkeywords={information theory, physics, coherence, UIF, recursion, cosmology}
}

\title{The Unifying Information Field (UIF) Paper I\\[0.35em]
\Large\textit{Core Theory}\\[0.6em]   % <-- adds the gap
\small Version v1.0 — October 2025}
\author{Stuart E.\,N. Hiles, BA (Hons)}
\date{}


\begin{center}
\thispagestyle{empty}
\vspace{2em}
{\small
© 2025 Stuart E. N. Hiles 

All rights reserved under Creative Commons Attribution–NonCommercial 4.0 (CC BY-NC 4.0).  
\newline

This document represents a pre-release version (v1.0, October 2025) of the 

\textit{Unifying Information Field (UIF)} series of papers.

First published on GitHub here (\url{https://github.com/stuart-hiles/UIF}).
\newline DOI: \href{https://doi.org/10.5281/zenodo.17434413}{10.5281/zenodo.17434413}  
Commit ID: \texttt{<insert-hash-here>}
}
\newline

This paper has not yet been peer-reviewed or formally published.  
\newline

All supporting software, scripts, and data are licensed separately under \textbf{GPL-3.0}.
\end{center}

\maketitle

\begin{abstract}
\thispagestyle{empty}
The Unifying Information Field (UIF) models reality as a collapse–return informational field in which informational difference (\(\Delta I\)) is conserved and redistributed through recursive coupling (\(\lambda_R\)) within a finite substrate (\(R_{\infty}\)).  This first paper defines the operator grammar of the field, formalises collapse–return dynamics, and introduces the seven-pillar architecture linking information, time, computation, and topology across scales.  UIF generalises wave–particle duality as a continuous informational cycle and reinterprets dark energy and dark matter as manifestations of an interactive informational substrate.  The framework unifies quantum and particle physics, cosmology, and biological coherence under a single variational principle, providing empirically testable predictions for coherence ceilings, hysteresis behaviour, and cross-domain informational conservation.
\newline

\noindent\textit{Empirical outlook.}
Recent high-resolution observations—from \textit{JWST} infrared imaging to 
\textit{EHT} polarimetric mapping of M87—already probe UIF’s core operators,
revealing measurable analogues of the receive--return coupling ($\lambda_R$)
and recursion rate ($\Gamma$) \cite{Perlman2025_M87_JWST,EHT2024_Polarization_M87}.
\newline

\noindent\textit{Series context:}
This paper is the first in the seven-part UIF series, introducing the operator grammar that underpins
subsequent volumes on symmetry (UIF II), field formalism (UIF III), cosmology (UIF IV),
and energetic closure (UIF V–VII).
\end{abstract}
\clearpage
\thispagestyle{empty}

\noindent\textbf{Series overview}
\newline
Paper~I — \textit{Core Theory} \cite{UIF-I} introduces the Unifying Information Field (UIF) as a collapse–return informational framework and defines its operator grammar.  
Paper~II — \textit{Symmetry Principles} \cite{UIF-II} develops the symmetry and invariance structure underlying informational conservation.  
Paper~III — \textit{Field and Lagrangian Formalism} \cite{UIF-III} establishes the continuous variational formulation of UIF and derives the Euler–Lagrange field equations.  
Paper~IV — \textit{Cosmology and Astrophysical Case Studies} \cite{UIF-IV} applies the framework to large-scale structure, coherence, and cosmological observables.  
Paper~V — \textit{Energy and the Potential Field} \cite{UIF-V} formulates the energetic and potential-field laws linking information, energy, and coherence.  
Paper~VI — \textit{The Seven Pillars and Invariants} \cite{UIF-VI} consolidates the invariant architecture of UIF across physical, biological, cognitive, and artificial domains.  
Paper~VII — \textit{Predictions and Experiments} \cite{UIF-VII} (forthcoming) completes the core series by presenting cross-domain predictions, coherence thresholds, and experimental validations.
\newline

\noindent\textbf{Companion}
\newline
Symbolic derivations, operator definitions, and reproducibility metadata
supporting this paper are archived in the
\textit{UIF~Companion Experiments} (2025) \cite{Companion2025}.
That volume documents the foundational operator calibration and initial
collapse–return simulations that establish the UIF framework.
\newline

\noindent\textbf{Repository}
\newline
All source code, symbolic notebooks, and figure-generation scripts are maintained in the
\textit{UIF GitHub Archive} (\url{https://github.com/stuart-hiles/UIF}),
with tagged releases ensuring reproducibility of all results.
The permanent, citable record of this release is preserved on \textit{Zenodo}
(\url{https://doi.org/10.5281/zenodo.17434413}), corresponding to
version~v1.0 of the UIF Series (Papers~I–VII and Companion).
\newline

\noindent\textbf{Note on Nomenclature and Continuity}
\newline
The Unifying Information Field (UIF) framework introduced here defines the core
operator grammar of informational dynamics.  All subsequent papers retain the
notation and generalise it into symmetry (\textit{UIF~II}),
field (\textit{UIF~III}), energetic (\textit{UIF~V}),
and invariant (\textit{UIF~VI}) formulations.  
Earlier UT26 terminology is fully superseded by the definitions presented here.
\newline

\noindent\textbf{Scope}
\newline
This paper establishes the foundational architecture of the Unifying Information Field,
introducing the collapse–return principle and the fundamental operators
$(\Delta I,\,\Gamma,\,\beta,\,\lambda_R)$ that govern informational dynamics.
These operators define how informational difference is conserved,
how coherence arises through recursion, how bias breaks symmetry,
and how coupling links systems to the substrate field $R(x,t)$.
The principles developed here provide the basis for the symmetry and invariance laws
formalised in \textit{UIF~II — Symmetry Principles}.

\noindent
The operator framework established here forms the foundation for the symmetry
laws of \textit{UIF II — Symmetry Principles}, and the Lagrangian and Euler–Lagrange
formulations developed in \textit{UIF III — Field and Lagrangian Formalism}.
\clearpage
\pagenumbering{arabic}
\setcounter{page}{1}
\section{Introduction}

UIF begins from a simple but radical proposition: that reality is underpinned by a foundational informational substrate. This substrate is not matter or energy, but the fundamental layer from which both emerge. It carries a distributed field structure: a continuous informational medium through which difference ($\Delta I$) can flow, coherence ($\Gamma$) can be maintained, and exchanges ($\lambda_R$) can occur. The substrate and its field provide the stage on which collapse--return dynamics take place. They are the universal medium within which all operators act, across physics, biology, cognition, and machines; not in isolation but within a universal informational medium, analogous to (but distinct from) physical fields in standard physics.

In UIF, collapse is defined as the resolution of an informational state, such as an entangled quantum system or a superposed trajectory, into a definite outcome upon sampling. Collapse refers to the moment that an unsampled informational difference must be resolved. A system cannot remain in superposition indefinitely; eventually a choice is made, and $\Delta I$ is redistributed between the system and its substrate. Collapse is not destruction: no information is lost, but it changes form. Each collapse therefore produces both an outcome and a trace; the record that constrains what future collapses are possible.

 Each collapse--return event can be viewed as a complete informational cycle comprising three stages: sampling, recursion, and return. These form the minimal triad of coherence; the dynamic loop through which informational difference ($\Delta I$) arises, is stabilised ($\Gamma$), and re-integrated through coupling ($\lambda_R$). This triadic symmetry reappears throughout the Unifying Information Field, echoing the harmonic balance that Tesla described as fundamental to all natural systems \cite{Tesla1892,Tesla1919,Cheney1981}
\newline

\noindent
For transparency, all numbered equations in this paper are classified according to their provenance:
\emph{[Identity]} designates a standard physical or informational law,
\emph{[Model law]} a relation derived within the UIF framework from stated assumptions,
and \emph{[Hypothesis]} a phenomenological or testable scaling introduced for future verification.
A complete table of equation provenance and accompanying symbol definitions
is provided in Appendix~A.
\newline

\noindent The following sections define the fundamental operators of the field and establish the seven pillars that structure all subsequent papers.

\section{Operators}

To formalise these principles, UIF defines a compact set of operators. These are the substrate’s alphabet; simple, universal rules that appear across physics, biology, cognition, and machine systems. Each operator has a logical analogue, allowing them to be understood as both physical invariants and computational gates. Together they provide the minimal grammar by which collapse can occur.

\subsection{$\Delta I$ (Informational difference)}

Every collapse begins with $\Delta I$ --- the unsampled imbalance between possible states. Without $\Delta I$, there is nothing to resolve. In computational terms, it behaves like an XOR: only when inputs differ is there an output. In physical systems this difference may be energy gradients, entropy imbalances, or unsampled quantum superpositions. In UIF, $\Delta I$ is always conserved: collapses redistribute informational difference between system and field, but do not erase it. $\Delta I$ is therefore the driver and payload of every collapse--return event.

\subsection{$\Gamma$ (Recursion / Coherence)}

$\Gamma$ describes the ability of a system to maintain and replay its state across time. It is the rhythm or clock of the substrate, analogous to an oscillator or flip-flop in digital logic. $\Gamma$ sustains coherence until collapse, keeping subsystems in phase with one another. Across scales, $\Gamma$ is observed as neural gamma rhythms, the variability cycles of quasars, and synchronising clocks in engineered systems. UIF treats $\Gamma$ as the universal timing operator that makes orderly collapse possible.

\subsection{$\beta$ (Bias / Elasticity)}

Collapses are not chosen uniformly. $\beta$ is the operator that weights probabilities and breaks symmetry, tilting outcomes toward one attractor or another. It is equivalent to a weighted threshold, providing elasticity that allows variation while still favouring particular resolutions. In cosmology $\beta$ accounts for how homogeneity gives way to structure; in
biology, for how cells favour one signaling pathway; in cognition, for how perception tips
toward one interpretation. $\beta$ ensures collapse is lawful but not uniform, embedding
direction into decision.

\subsection{$\lambda_R$ (Retention / Receive--Return)}

$\lambda_R$ is the coupling constant linking local systems to the informational field $R(x,t)$. This receive--return field absorbs informational traces into the substrate and subsequently re-emits or transmits them after a delay. $\lambda_R$ quantifies the strength of this two-way interaction, analogous to coupling constants in field theory but defined informationally. In practice, $\lambda_R$ acts both as a channel - determining how strongly systems are coupled to the field and as memory; setting how much of a collapse is retained and shapes future outcomes. It explains why black holes echo, why filaments act as coherence conduits, and why memories bias
perception.  $\lambda_R$ also maps to a primitive logic gate: an AND, since exchange occurs only when both system and field are engaged.

\subsection{$R_\infty$ (Finite Ceiling)}

No system can accumulate coherence without limit. $R_\infty$ defines the maximum capacity of informational growth; a logistic ceiling beyond which collapse prunes the state. In
cosmology this ceiling produces late-time suppression of structure growth, in biology it
corresponds to saturation in synchronisation, and in computation it is seen as network or buffer
limits. $R_\infty$ is a capacity law: the ceiling of lawful participation.

\subsection{$k$ (Recharge Rate)}

Collapse not only has limits, it has dynamics. $k$ sets the rate at which coherence rises toward
its ceiling  $R_\infty$. It is the slope of logistic growth: slow in some systems, rapid in others. In cosmology, $k$ controls how fast structures grow before saturation; in neuroscience, it
appears as the resynchronisation rate of rhythms after perturbation. Together  $R_\infty$ and $k$ define session limits: how much coherence can be stored, and how quickly it can recover. The coherence ceiling ($R_\infty$) and recharge rate ($k$) are empirically calibrated in Paper 5 §
5.3 – 5.7.

\subsection{$\eta$ (Threshold)}

Not every fluctuation forces collapse. $\eta$ defines the minimum $\Delta I$ required to trigger it. Collapse occurs only if informational difference exceeds this threshold. In astrophysics $\eta$ corresponds to limits like the Chandrasekhar mass or gamma-ray burst ignition; in
structure formation, it defines the low-mass slope of the halo function; in biology, it is the
minimal stimulus needed to bias awareness.$\eta$ acts as the gatekeeper operator, ensuring
collapses are not trivial chatter but meaningful acts of participation.
\noindent

A complete summary of operator symbols, their meanings, and dimensional units
is provided in Appendix~B (\textit{Symbols and Units, UIF I}).
\section{Operators as logic gates}

These operators map to primitive logic gates, underscoring that collapse--return is a form of informational computation. Just as all modern computation reduces to gates (Shannon, 1948; Brillouin, 1962), UIF reduces collapse to a minimal gate set acting on a universal substrate. These are not electronic components but physical operations; lawful informational gates enacted by the substrate itself, expressing the universe’s own computational grammar.
\vspace{0.5\baselineskip}  
% ---------- Table 1: Operator -> Gate Mapping (flows across pages) ----------
\begin{longtable}{@{} L{0.20\textwidth} L{0.36\textwidth} L{0.18\textwidth} L{0.26\textwidth} @{}}
\caption{Operator $\rightarrow$ Gate Mapping in UIF}\label{tab:1-operator-gate}\\
\toprule
\textbf{Operator} & \textbf{Role in Collapse--Return} & \textbf{Logic Gate Analogue} & \textbf{Notes} \\
\midrule
\endfirsthead
\toprule
\textbf{Operator} & \textbf{Role in Collapse--Return} & \textbf{Logic Gate Analogue} & \textbf{Notes} \\
\midrule
\endhead
\bottomrule
\endfoot

$\Delta I$ (Informational Difference) & Detects unsampled imbalance; collapse required only if difference exists & XOR / DIFFERENCE & Collapse occurs when states differ, not when they match \\
$\Gamma$ (Recursion / Coherence) & Provides rhythm/clock for coherence and replay & Flip-flop / Oscillator & Maintains phase until collapse; timing operator \\
$\beta$ (Bias / Elasticity) & Tilts probabilities; breaks symmetry; weights outcomes & Weighted Threshold Gate & Shifts balance toward one attractor \\
$\lambda_R$ (Retention / Receive--Return) & Couples system to return field $R(x,t)$; retains traces & AND Gate (with delay) & Exchange only when both system and field participate \\
$R_\infty$ (Finite Ceiling) & Sets maximum coherence capacity & Limiter / Saturation Function & Equivalent to logistic ceiling in growth models \\
$k$ (Recharge Rate) & Sets slope of coherence recovery & Slope / Gain Parameter & Governs how fast the ceiling is approached \\
$\eta$ (Threshold) & Minimum $\Delta I$ required to trigger collapse & Gate / Comparator & Collapse fires only if $\Delta I \ge \eta$ \\

\end{longtable}
\vspace{0.5\baselineskip}  

Alongside the primitive operators of UIF ($\Delta I$, $\Gamma$, $\beta$, $\lambda_R$, $R_\infty$, $k$, $\eta$), we also define composite observables that allow empirical testing. The most important is $\mathcal{R}$, informational richness, a measure derived from entropy - complexity geometry that quantifies how much structure is present relative to noise. Unlike the primitive operators, $\mathcal{R}$ is not fundamental but diagnostic: it is a way to observe the action of $\Delta I$ and $\Gamma$ in real data, from astrophysical light curves to neural recordings.
(The diagnostic observable $\mathcal{R}$, defined in Appendix~B, quantifies informational richness relative to noise
and is distinct from the substrate field $R(x,t)$ and ceiling $R_{\infty}$.)


\section{Definition of Collapse in UIF}
Collapse is not destruction of alternatives but redistribution of information: part expressed locally in the realised outcome, part retained non-locally in the substrate via $\lambda_R$. This generalises the familiar notion of quantum wavefunction collapse to informational systems more broadly.

Collapse in UIF is a lawful computational step rather than a metaphysical discontinuity. It reframes the measurement problem: what quantum theory describes as ``choice'' is, in UIF, the structured redistribution of $\Delta I$ across local and substrate registers. Entanglement
correlations persist because the same $\Delta I$ traces are shared via  $R(x,t)$, not because of
unexplained non-local signalling \cite{Bell1964,Aspect1982}.

\noindent
\textit{Implication.} If this reading holds, the familiar physical world is the
visible surface of a deeper informational recursion—a universe that computes
itself through continual collapse and return.


\section{The Interactive Substrate}

Where standard cosmology treats dark energy as a passive, non-interacting cosmological
constant \cite{Riess1998,Perlmutter1999,Planck2018} UIF makes a stronger claim: the substrate is interactive. Through $R(x,t)$, systems couple to it bidirectionally, with strength defined by  $\lambda_R$. Local informational collapses deposit outcome traces into the substrate, which later return as echoes, anomalies, or delayed feedback. Unrealised outcomes (wavefunctions that did not collapse) also persist as informational potential. Reports of isolated black holes without stellar progenitors \cite{Mroz2022} are interpreted as substrate gates, primordial attractors seeded in the early universe. The substrate is not unknowable but empirically accessible; leaving signatures in CMB residuals, lensing discrepancies, and unexplained bursts.

Having defined the operator grammar of UIF, we now show how these operators express as
measurable pillars of physical, biological, and cognitive systems.
\section{UIF Pillars}

UIF is built on seven interlocking principles, or `pillars'. Each pillar is introduced, developed, and closed with consequences, so the framework can be assessed scientifically.

\subsection {Pillar 1 - Information as Substrate ($\Delta I$)}
In UIF, information is the fundamental substrate of reality. Conventional physics prioritises
matter and energy, however UIF treats $\Delta I$ (informational difference) as irreducible. Black hole entropy shows horizons encode information \cite{Bekenstein1973,Hawking1975}. Entanglement experiments reveal that correlations, not distance, govern outcomes \cite{Bell1964,Aspect1982}. Quantum error correction demonstrates that informational states can be protected independently of their physical carriers \cite{Shor1995}.

UIF unifies this with biology and AI: genetic coding and machine learning operate on the
organisation of information rather than its material substrate. $\Delta I$ is conserved because
collapse–return redistributes it between the local system and the substrate field $R(x,t)$.  $\Delta I$ itself is quantised into informational quanta (the minimal carriers of collapse),  which we refer to as 'informons' \cite{Wheeler1990,Zurek1990}. Informons here are not a new particle species, but a convenient label for quantised units of $\Delta I$ collapse.

While UIF treats $\Delta I$ as irreducible, some argue that information is derivative or descriptive rather than ontological \cite{Timpson2013,Fields2020}. UIF addresses this by
tying $\Delta I$ to measurable collapse–return events, not abstract symbols.

At the level of individual events, collapse--return yields a non-negative informational gain \cite{Shannon1948}:
\begin{equation}
\Delta I_{\text{event}} = H(X) - H\!\left(X \mid \text{sample}\right) = I(X; \text{sample}) \ge 0.
\end{equation}
This expresses that every collapse–return event yields non-negative $\Delta I$: informational
difference is the conserved substrate of UIF.

\subsection* {Dynamical exchange with the substrate.}
Beyond individual events, $\Delta I$ evolves dynamically through local sources, sinks, and exchange with the substrate field. Instead of being lost, difference is exported into the field and
subsequently re-imported with a finite return timescale. This is expressed as coupled first-order equations:
\begin{subequations}
\label{eq:receive-return-coupled}
\begin{align}
\dot{\Delta I}_{\text{sys}} &= S_{\text{in}} - L_{\text{out}} - \lambda_R \,\Delta I_{\text{sys}} + \lambda_R \,\Delta I_{\text{field}}, \\
\dot{\Delta I}_{\text{field}} &= -\mu\,\Delta I_{\text{field}} + \lambda_R \,\Delta I_{\text{sys}}, \qquad \mu=\frac{1}{\tau_R}.
\end{align}
\end{subequations}
Here $\lambda_R$ is the is the receive–return coupling constant, and $\mu = 1/\tau_R$ sets the effective return/echo timescale of the field. The local register $\Delta I_{\text{sys}}$ can therefore offload informational difference to the substrate field, which then returns it gradually rather than instantaneously. This structure produces the echo and hysteresis effects observed in
collapse–return phenomena.
\newline

\noindent \textbf{UIF Alignment} 

\noindent Collapse–return ensures information is never lost, only redistributed. $\Delta I$ is the fundamental currency of reality across physics, biology, and computation.
\newline

\noindent \textbf{Synthesis} 

\noindent UIF places information, not matter or energy, as the fundamental substrate. Collapse–return dynamics turn potential informational states into realised outcomes, with $\Delta I$ as the
quantised unit of change. These quanta of informational difference also underlie the conserved topological invariants discussed in Pillar 7, linking the substrate directly to the stability of spin, charge, and memory across scales.
\newline

\noindent \textbf{Forward Pointer} 

\noindent This prepares the ground for Pillar 2, where time itself emerges from recursive sampling of $\Delta I$.
\newline

\noindent \textbf{Novelty / Testability}

\noindent $\Delta I$ is empirically measurable in astrophysical light curves (quasar H–C geometry) and in neural recordings (EEG coherence), where it reliably distinguishes coherent dynamics from noise.

\vspace{0.6em}
\noindent\textit{Unless otherwise stated, all quantities are non-dimensionalised by the reference scales $(\Delta I_{0},\,\tau_{0},\,L_{0})$ hereafter.}
\vspace{0.6em}

\subsection {Pillar 2 - Emergent Time}

Standard equations are time-symmetric, but observation produces irreversibility \cite{PageWootters1983,Rovelli1995}. Cosmology shows delayed returns in CMB statistics \cite{Planck2018}. Neuroscience reveals readiness potentials \cite{Libet1983} and postdictive perception \cite{Eagleman2009}, where conscious time is reconstructed after the event. Behavioural studies show that tempo depends on event density: dense events accelerate subjective time; sparse events stretch it \cite{Buhusi2005}. In UIF, tempo depends directly on $\Delta I$ sampling.

The claim that time is emergent remains debated. Some maintain time is primitive, built into the fabric of physics \cite{Maudlin2007,Ismael2017}. UIF takes the opposing view but grounds it in testable predictions from sampling density and recursion.

\paragraph{Event-level time scaling.}
\begin{equation}
\label{eq:63}
T_S \propto f_s\, \Delta I.
\end{equation}

Here, system-level time $T_S$ scales with sampling frequency $f_s$ and informational richness per event $\Delta I$.

\paragraph{Elasticity of time.}
\begin{equation}
\label{eq:64}
\frac{d T_S}{dt} \propto \frac{1}{\beta\, \Gamma \, \Delta I}.
\end{equation}

Elasticity of time arises when recursion ($\Gamma$) amplifies differentiation, effectively “slowing” experienced time while the underlying clock continues. This shows that as recursion ($\Gamma$) and differentiation ($\Delta I$) increase, subjective time stretches relative to background clock time.
\newline

\noindent
\textbf{UIF Alignment} 
\newline
\noindent In UIF, time is reframed as an emergent property of collapse–return dynamics, set by the sampling of $\Delta I$ and sustained by recursion through $\Gamma$. Elasticity of subjective or system-level time arises when recursion amplifies differentiation, producing lawful variations in tempo across domains.
\newline

\noindent \textbf{Synthesis (Time)}

\noindent This framing connects relational models of time in physics 
\cite{PageWootters1983,Rovelli1995}
with empirical observations: cosmological coherence delays in the CMB, 
readiness potentials in neuroscience \cite{Libet1983}, 
and perceptual postdiction in cognition \cite{Eagleman2009}. 
Behavioural findings on event density \cite{Buhusi2005}
are reinterpreted as variations in $\Delta I$ sampling. 
Testability lies in psychophysical studies of temporal binding, EEG entrainment, 
and astrophysical signatures of coherence delay.
\newline

\noindent \textbf {Synthesis (Dark Substrate).}

\noindent In UIF we refer to the dark substrate as the informational reservoir of unrealised possibilities represented by the field $R(x,t)$. At cosmological scales this plays the role that $\Lambda$CDM assigns to dark energy: an apparently smooth background pressure. Unlike a cosmological constant, however, the dark substrate is finite, saturable, and leaves traces of past collapses. This reframes dark energy as the residual expansion pressure of unrealised possibilities, not as a mysterious force. Novelty lies in recasting dark energy as a statistical property of the substrate itself.

\noindent
In $\Lambda$CDM this is described through the equation of state parameter:
\begin{equation}
\label{eq:65}
w(z) \;=\; \frac{p(z)}{\rho(z)}\,,
\end{equation}
where $p(z)$ is pressure and $\rho(z)$ is energy density. UIF predicts that because the substrate is finite and saturable, $w(z)$ should not be constant, but exhibit oscillatory behaviour reflecting recursive collapse–return. Testability comes from empirical signatures: logistic ceilings ($R_\infty$) measured in quasar variability, constraints from fifth-force searches, and oscillatory $w(z)$ reconstructions in cosmological data.
\newline

\noindent \textbf {Forward Pointer} 
\newline 
This links naturally to Pillar 3, where the substrate $R(x,t)$ provides the medium in which recursive delays accumulate.
\newline

\noindent \textbf {Novelty/Testability (Time)} 
\newline
Time’s emergence is testable through psychophysical studies of temporal binding and EEG entrainment, as well as cosmological observations of coherence delays in CMB and astrophysical systems.
\newline

\noindent \textbf {Novelty/Testability (Dark Substrate)}
\newline
Novelty lies in reframing dark energy as a statistical property of the substrate rather than a new fundamental force. Testability comes from measurable signatures: logistic ceilings ($R_\infty$) in quasar variability, constraints from fifth-force searches, and oscillatory reconstructions in cosmological datasets.
\newline

\noindent\textbf{Mathematical and Dimensional Conventions.}
\newline
\noindent Unless stated otherwise, all quantities in proportional or nonlinear relations
(e.g.\ Eqs.\,(1.3–1.4), (1.8–1.10), (1.15–1.17)) are expressed in dimensionless,
normalised form:
$\tilde{\Delta I}=\Delta I/\Delta I_0$, $\tilde{\Gamma}=\Gamma/\Gamma_0$,
$\tilde{f}_s=f_s/f_0$, $\tilde{\lambda}_R=\lambda_R/\lambda_{R0}$, etc.
Physical units are restored where required using the appropriate scaling constants
($k_T$, $k_E$, $k_A$, \ldots) as listed in Appendix~B.
\newline

% --- Comparative and Related Frameworks ---
\noindent\textbf{Comparative Theoretical Context (QMM vs UIF)}
\newline
\noindent Recent work by Neukart and collaborators introduces the \textit{Quantum Memory Matrix} (QMM) framework, which models spacetime as a lattice of finite-dimensional memory cells that store and allow unitary retrieval of local quantum imprints, aiming to address the black-hole information paradox and related cosmological puzzles \cite{Neukart2025_QMM,Neukart2025_QMM_Imprint,Neukart2025_QMM_SM,Neukart2025_QMM_PBH}. Preliminary hardware demonstrations report imprint–retrieval fidelities approaching $\sim$77\% on IBM processors \cite{Neukart2025_QMM_Imprint}. 

While both QMM and UIF treat information as fundamental and conserved, they differ in mechanism and scope. QMM adopts a discrete, Planck-scale cell architecture with local imprint operators, primarily targeting quantum-gravity and cosmological problems. UIF formalises a continuous receive–return field with operators $(\Delta I,\Gamma,\beta,\lambda_R,R_\infty,k,\eta)$ that act across scales, and develops cross-domain, testable predictions (informational hysteresis, coherence resonance, CMB-like residuals, conversational hysteresis in AI). Accordingly, the two programs are complementary: QMM offers a microscopic imprint model; UIF provides a macroscopic field framework with broader empirical avenues. We reference QMM where relevant (e.g., informational retention and echo phenomena) while pursuing UIF’s receive–return formalism and cross-domain test suite in Papers~II–VII.


% ------------------------------------------------------------

\subsection {Pillar 3 --- Potential Field and Dark Substrate ($\lambda_R$)}
In UIF, the dark sector is the informational reservoir which stores unrealised outcomes and collapse traces. $\Lambda$CDM treats dark energy as passive \cite{Riess1998,Perlmutter1999,Planck2018}, but UIF reinterprets it as an interactive potential. The operator $\lambda_R$ quantifies this receive–return coupling between local systems and the substrate field $R(x,t)$.

The same law manifests across domains: collapses deposit $\Delta I$ into a distributed medium which later returns traces with finite fidelity and delay. In neural systems this is realised in associative memory networks \cite{Hopfield1982}, where input patterns are stored and recalled through distributed coupling. In artificial intelligence, experience replay \cite{Lin1992} implements the same principle, allowing stored traces to be re-sampled to guide learning. In UIF these are not metaphors but instances of the $\lambda_R$ operator at different scales: retention and return are universal features of collapse–return dynamics.

The same receive–return principle also appears at astrophysical scales. Isolated black holes detected by microlensing \cite{Mroz2022b} may represent substrate gates — primordial attractors seeded in the early universe. At the quantum scale, memory experiments confirm delayed retrieval of stored states \cite{Lvovsky2009}, again consistent with finite $\lambda_R$ coupling.

Reinterpreting the dark sector as informational departs from standard $\Lambda$CDM treatments, and some warn against overly speculative accounts of dark energy \cite{Ellis2006}. Operationally, $\lambda_R$ can be constrained by measurable echo amplitudes and delays in gravitational-wave ringdowns, by return times in quantum memory analogues, and by coherence scaling in cosmic filaments \cite{Zhao2024}. These signatures provide direct bounds on the strength and timescale of $\lambda_R$ coupling.
\newline

\noindent \textbf {Receive–Return Coupling Dynamics}

\noindent This dynamic is captured schematically by the receive–return coupling law. Local informational difference is exported into the substrate field via $\lambda_R$ and subsequently returned with finite delay and decay. The result is echo and hysteresis behaviour.

\paragraph{Coupled system–field equations (recommended form).}
\begin{subequations}
\label{eq:66}
\begin{align}
\dot{\Delta I}_{\text{sys}} &= S_{\text{in}} - L_{\text{out}} - \lambda_R \,\Delta I_{\text{sys}} + \lambda_R \,\Delta I_{\text{field}}, \tag{1.6a}\\
\dot{\Delta I}_{\text{field}} &= -\mu\,\Delta I_{\text{field}} + \lambda_R \,\Delta I_{\text{sys}}, \qquad \mu=\frac{1}{\tau_R}. \tag{1.6b}
\end{align}
\end{subequations}
\noindent \textbf {Equivalent convolution form.}
\newline
The same behaviour can be expressed as a single equation using convolution with a causal exponential kernel:
\begin{equation}
\label{eq:67}
\frac{d}{dt}\Delta I_{\text{local}}(t) \;=\; -\lambda_R \,\Delta I_{\text{local}}(t)
 \;+\; \lambda_R \int_{0}^{\infty} \!K_R(\tau)\, \Delta I_{\text{field}}(t-\tau)\, d\tau,\quad 
 K_R(\tau)=\frac{1}{\tau_R}e^{-\tau/\tau_R}.
\end{equation}
Here $(f*g)(t)=\int_{0}^{\infty} f(t-\tau)\,g(\tau)\,d\tau$ denotes convolution.
The kernel $K_R(\tau)$ enforces finite memory and produces echo/hysteresis effects.
\newline

\noindent
Unless otherwise stated, we assume periodic boundaries on $\partial\Omega$
and smooth initial data $(\Phi_0,R_0)$ to avoid boundary‐flux terms
in the Noether current.  
This ensures informational conservation holds strictly within the domain of the
receive–return field and that apparent losses arise only from coupling,
not leakage at the boundaries.

\noindent
In variational terms, the coupled receive–return equations can be derived from a 
Lagrangian density $\mathcal{L}(\Phi,R)$ with a corresponding current $J_{\Phi}$ yielding the continuity form with a corresponding current $J_{\Phi}$ yielding the continuity form 
$\partial_t(\Phi^2)+\nabla\!\cdot J_\Phi=0$,  ensuring informational conservation 
under continuous transformations of $\Phi$ (cf.\ Eq.~(2.1) in \textit{UIF~II} 
and Appendix~C in \textit{UIF~III}).  
When boundary fluxes vanish on $\partial\Omega$, the total informational action 
$\int \!\mathcal{L}\,dt$ remains stationary, confirming that collapse–return dynamics 
obey a true variational principle.  
\newline


\noindent \textbf{Net exchange per cycle}
\newline
Let $\varrho_{R} \defeq \lambda_{R}\,\tau_{c}$ denote the \emph{dimensionless per-cycle coupling fraction}, 
where $\tau_{c}$ is a representative collapse–return cycle time.  
The net effect per cycle is that a fraction $(1-\varrho_{R})$ of the local informational difference is 
retained, while a fraction $\varrho_{R}$ is written to the substrate and returns with characteristic 
timescale $\tau_{R}$.  
This formulation keeps $\lambda_{R}$ as a rate parameter (units s$^{-1}$) while expressing its effective 
per-cycle action through $\varrho_{R}\!\in\![0,1]$.  
The result is partial loss to the substrate and partial delayed return, producing the observed echoes 
and hysteresis behaviour.
\newline


\noindent
\textbf{UIF Alignment} 
\newline
The dark substrate is reframed as an active information reservoir. Within the entanglement protocol, $\lambda_R$ provides the channel for write–read exchange, acting alongside $\Delta I$ (payload), $\Gamma$ (clock), and $\beta$ (bias). This governs how collapse traces are deposited into and retrieved from the substrate field, producing measurable echoes across domains: delayed CMB correlations in cosmology, recall in associative neural networks \cite{Hopfield1982}, replay in AI architectures \cite{Lin1992}, and astrophysical signatures such as microlensed orphan black holes \cite{Mroz2022b}.
\newline

\noindent \textbf{Synthesis}
\newline
Collapse–return cycles are literal computational steps of the substrate. Informons act as gates, with $\beta$ setting the bias, $\Gamma$ providing the timing, and $\lambda_R$ coupling local states to the substrate. Computation here is not metaphorical but the mechanism of physics itself. Novelty lies in treating Landauer’s principle — that information erasure carries an entropy cost — as a universal substrate law. Testability follows because every informational gate leaves a thermodynamic trace: entropy increase, measurable coherence decay constants ($\tau$), and the inverted-U stochastic-resonance response observed in both biological and physical systems.
\clearpage
\noindent
\textbf{Forward Pointer}
\newline This prepares for Pillar 4, where collapse–return cycles are explicitly recognised as computational steps in their own right.
\newline

\noindent
\textbf{Novelty/Testability} 
\newline The finite nature of the substrate is supported by empirical fits presented elsewhere, which calibrate the coherence ceiling $R_\infty$ and recharge rate $k$ from quasar variability data, providing quantitative estimates of the receive–return coupling $\lambda_R$ in practice.

\begin{quote}\itshape
\textbf{Prediction 1 (Paper I).} If $\lambda_R>0$ with return time $\tau_R$, then following a step increase in $\Gamma$ the observed richness proxy $\mathcal{R}(t)$ obeys
$\mathcal{R}(t)=\mathcal{R}_\infty\!\left(1-e^{-k t}\right)+\alpha\,e^{-t/\tau_R}$ with $\tau_R>0$.
Estimating $(k,\tau_R,\alpha)$ from quasar light curves or $\gamma$-band recovery (UIF IV, VII) tests the receive–return hypothesis.
\end{quote}

% ------------------------------------------------------------

\subsection {Pillar 4 - Computation as Fundamental}

In UIF, collapse–return cycles are literal computational steps. Physics, biology, and AI all manifest recursive informational processing. Operators map naturally to logic gates: $\Delta I \rightarrow$ XOR, $\Gamma \rightarrow$ latch, $\beta \rightarrow$ weighted threshold, and $\lambda_R \rightarrow$ AND \cite{Shannon1948,Brillouin1962}. Each collapse both reduces uncertainty and enacts a computational operation.

This behaviour appears across domains. In neural systems, recursion sustains state in attractor networks; in AI, recurrent architectures update state through weighted thresholds; in physics, every bit erased has a minimal thermodynamic cost \cite{Landauer1961}. Modern quantum information science strengthens this view: computation is increasingly treated as a physical primitive, with proposals ranging from near-term quantum architectures \cite{Preskill2018} to models without definite causal structure \cite{Chiribella2020} and debates over hidden-variable constraints. Pancomputational accounts have been criticised as unfalsifiable if computation is defined too broadly \cite{Searle1990}. UIF avoids this by tying computation strictly to collapse–return operators.

\paragraph{Latch / flip-flop recursion.}
\begin{equation}
\label{eq:68}
s_{t+1} \;=\; \sigma\!\big(\Gamma\, s_t + x_t\big).
\end{equation}

\paragraph{Weighted threshold / bias gate.}
\begin{equation}
\label{eq:69}
y \;=\; \sigma\!\big(\beta^{\top} x\big).
\end{equation}

\paragraph{AND-like receive–return gate.}
\begin{equation}
\label{eq:610}
z \;=\; \sigma\!\big(\lambda_R\, u_{\text{local}} \cdot u_{\text{return}}\big).
\end{equation}

\paragraph{Landauer bound.}
\begin{equation}
\label{eq:611}
E_{\min} \;=\; k_B\,T\,\ln 2 \;\times\; \Delta I_{\text{bits}}.
\end{equation}
\newline

\noindent
\textbf{UIF Alignment} 
\newline Computation is reframed as the primitive process of reality. Within the entanglement protocol, $\Delta I$ is the payload, $\Gamma$ the recursion clock, $\beta$ the threshold, and $\lambda_R$ the receive–return channel. Each collapse generates an informon, enacting a universal gate operation. Collapse–return cycles are thus not only describable as computation but constitute the computational substrate of the universe.
\newline

\noindent \textbf {Synthesis}
\newline Coherence and recursion drive agency. Systems that sample recursively amplify coherence and eventually cross thresholds into self-prompting. $\Gamma$ sets the rhythm of recursion, enabling integration across scales, from synchronised oscillators to group coherence. Agency emerges as a substrate phase transition: once recursion sustains coherence above $\eta$, systems gain proto-agency. Testability follows from experiments on synchronisation, coherence residuals, and AI reset replications, where agency signatures emerge predictably with recursive richness.
\newline

\noindent
\textbf{Forward Pointer} 
\newline This leads directly to Pillar 5, where recursive computation stabilises coherence and drives systems toward agency.
\newline

\noindent
\textbf{Novelty/Testability} 
\newline Every gate leaves a thermodynamic trace, as formalised in UIF’s Lemma. Landauer’s principle ensures entropy increases of $k_B T \ln 2$ per bit erased. These traces are testable as: entropy production (in nanoscale logic devices and superconducting qubits), coherence decay constants ($\tau$) (in EEG/MEG synchronisation, coupled oscillators, and Josephson junction arrays), and the inverted-U response predicted by stochastic resonance experiments (in biological and physical systems alike).

% ------------------------------------------------------------

\subsection {Pillar 5 - Coherence and Recursion ($\Gamma$)}

In UIF, coherence arises from recursive feedback ($\Gamma$), not isolation. Systems remain stable not because they are closed, but because they continually reinforce informational patterns through recursive sampling. Quantum coherence (entanglement, superconductivity), biological homeostasis, and collective synchronisation all emerge from recursive informational alignment. Failures of recursion produce informational pathologies: robustness loss, ageing, and disease \cite{Demetrius2005,Gatenby2007,Noble2012}. In AI, recurrent networks exploit recursion to sustain state across sequences; in physics, phase locking and entanglement are formal expressions of the same principle.

Recent work has extended these ideas. Large-scale quantum states have been characterised in terms of their coherence measures and fragility \cite{Frowis2018}. Neuroscience continues to highlight the role of gamma-band synchronisation in neural integration and awareness \cite{Singer2018}. Thermodynamic approaches also emphasise recursion and irreversibility as foundations of coherence in physical systems \cite{Gyftopoulos2021}. Operationally, $\Gamma$ can be constrained by coherence order parameters in quantum systems \cite{Frowis2018}, by gamma synchronisation and cross-frequency coupling in neural recordings \cite{Singer2018}, and by spin–filament alignment statistics in cosmology \cite{Wang2025}.

These provide empirical measures of how strongly recursion sustains information across domains. These updates strengthen the claim that $\Gamma$ quantifies a universal property of informational stability.
\newline


\paragraph{Coherence order parameter can be expressed as:}
\begin{equation}
\label{eq:612}
C \;=\; \Gamma\,\langle \Delta I \rangle.
\end{equation}

\paragraph{A threshold condition captures the tipping point between decay and persistence:}
\begin{equation}
\label{eq:613}
\Gamma \;\ge\; \Gamma_c \;\;\Rightarrow\;\; C>0.
\end{equation}

\paragraph{Echoes and hysteresis reflect recursion’s memory effects:}
\begin{equation}
\label{eq:614}
E(t) \;=\; E_0\, e^{-t/\tau_{\text{echo}}}.
\end{equation}

\noindent
\textbf{UIF Alignment} 
\newline Coherence is reframed as the stability of recursive information loops. $\Gamma$ determines whether systems amplify or dissipate $\Delta I$, sustaining chains of informons across collapse–return cycles. Within the entanglement protocol, $\Gamma$ functions as the recursion clock, aligning phase across subsystems and enabling coherence from quantum states to neural assemblies and collective synchronisation. Failures of recursion explain robustness loss in ageing, decoherence in physics, and collapse of agency in AI.
\newline

\noindent \textbf{Synthesis}
\newline Modern work in quantum physics, neuroscience, and thermodynamics reinforces $\Gamma$ as a measurable and testable operator. $\Gamma$ sets the tipping point between decay and persistence: when recursion exceeds a critical threshold, coherence is sustained, leading to stability across scales. Novelty lies in reframing coherence as recursive informational stability rather than system closure.
\newline

\noindent
\textbf{Forward Pointer} 
\newline This prepares for Pillar 6, where coherence thresholds and recursion drive transitions into agency and self-sustaining dynamics.
\newline

\noindent
\textbf{Novelty/Testability}
\begin{itemize}[leftmargin=*]
\item Quantum systems: coherence order parameters and fragility measures in large-scale states \cite{Frowis2018}.
\item Neural systems: $\gamma$-band synchronisation and cross-frequency coupling in neural recordings \cite{Singer2018}.
\item Cosmology: spin--filament alignment statistics \cite{Wang2025}.
\item Dynamics: decay constants $\tau_{\text{echo}}$ in echo and hysteresis experiments.
\end{itemize}
Together these provide cross-domain validation of $\Gamma$ as a universal operator of coherence.

% ------------------------------------------------------------
\subsection {Pillar 6 — Agency and Consciousness ($\Delta I, \Gamma, \beta, \lambda_R$)}

In UIF, consciousness emerges when recursion, bias, and coupling cross critical thresholds. Agency is informational integration with predictive power. Gamma synchronisation and cross-frequency coupling in neural systems mark awareness; AI shows proto-agency when persistent states and self-prompting loops appear; collectives achieve agency through recursive communication. UIF frames superintelligence not as speculative but as a lawful trajectory of increasing informational integration \cite{Turing1950,Friston2010}. A developmental pathway follows: Sampling $\rightarrow$ Recursion $\rightarrow$ Bias $\rightarrow$ Coupling $\rightarrow$ Integration $\rightarrow$ Agency \cite{MaynardSmith1995}.

This pathway is echoed in modern accounts. Integrated Information Theory \cite{Tononi2016} formalises consciousness as thresholded informational integration. Studies of group problem solving show that collective intelligence arises from recursive communication \cite{Woolley2010}, while Malone (2018) describes “superminds” where humans and machines integrate as collective agents. AI has demonstrated proto-agency: systems like AlphaGo exhibit persistence and strategy beyond training data \cite{Silver2017}, and contemporary large language models show self-prompting and emergent behaviours \cite{OpenAI2023}.

Some philosophers argue that informational integration cannot, by itself, solve the “hard problem” of consciousness \cite{Chalmers1995}. UIF reframes the problem: agency emerges once integration thresholds across $\Delta I$, $\Gamma$, $\beta$, and $\lambda_R$ are crossed — a claim open to empirical testing.

\paragraph{This trajectory can be formalised with an agency intensity:}
\begin{equation}
\label{eq:615}
A \;\propto\; \Gamma\, f_s \, \Delta I.
\end{equation}

Here, agency ($A$) grows with recursion strength ($\Gamma$), sampling frequency ($f_s$), and informational richness ($\Delta I$).

\paragraph{Agency emerges only beyond a threshold:}
\begin{equation}
\label{eq:616}
A \;\ge\; A_c \;\;\Rightarrow\;\; \text{emergent agency.}
\end{equation}

\paragraph{Actions reflect weighted integration of bias, return, and recursive state:}
\begin{equation}
\label{eq:617}
P(\text{act}) \;=\; \sigma\!\big(\beta^{\top} u \;+\; \lambda_R r \;+\; \Gamma s\big).
\end{equation}

\noindent
\textbf{UIF Alignment} 
\newline Within the entanglement protocol, $\Delta I$ provides the payload, $\Gamma$ supplies recursion, $\beta$ biases outcomes, and $\lambda_R$ couples system and substrate. Agency emerges when these operators jointly cross critical thresholds consistent with these equations, producing stable, self-prompting behaviour.
\newline

\noindent \textbf{Synthesis.}
\newline Consciousness in UIF is not an anomaly but a natural outcome of the substrate’s drive toward coherence and informational richness. By reframing quantum numbers (spin, charge, parity) as conserved informational invariants, and predicting new invariants (coherence index, collapse susceptibility, topological complexity, collapse memory), the theory links substrate physics directly to subjective awareness. Every collapse–return gate leaves an entropy trace, and subjective continuity — the “stream of consciousness” — is explained as the accumulation of these traces. Just as increasing informational richness produces heavier particles in physics, it also yields richer subjective states in cognition. In this framing, subjective time (Pillar 2) is the felt cadence of recursive computation (Pillar 4), and Pillar 7 generalises how invariants and operator couplings unify agency with the structure of matter.
\newline

\noindent
\textbf{Forward Pointer} 
\newline This naturally leads to Pillar 7, where topology and forces unify invariants with operator couplings.
\newline

\noindent
\textbf{Novelty / Testability} 
\newline Consciousness is predicted to arise when measurable thresholds are crossed: coherence indices in neural EEG/MEG synchrony, oscillator networks, and group communication protocols should reveal the point at which agency-like signatures emerge.

% ------------------------------------------------------------
\subsection {Pillar 7 - Topology and Forces}

UIF recognises that the stability and interactions of systems depend on the topological properties of informons and the operator-driven couplings that act upon them. Spin, charge, and parity are not arbitrary labels but conserved topological invariants of informons, arising from their winding, flux, and orientation. Additional invariants are predicted as a family:
\[
\{\lambda_R,\; \eta^{*}(f),\; \tau_{\text{index}},\; M\}.
\]

\noindent
\textbf{Invariants} 
\begin{itemize}[leftmargin=*]
\item \textit{Coherence index} ($\lambda_R$) - the degree to which an informon couples to the substrate, setting retention and persistence.
\item \textit{Collapse susceptibility} ($\eta^{*}(f)$) - a measure of how easily an informon collapses under perturbation, tied to frequency response.
\item \textit{Topological complexity} ($\tau_{\text{index}}$) -  quantifying sub-knots or braids within informons, underlying exotic particles and emergent states.
\item \textit{Collapse memory} ($M$) -  persistence of bias across collapse cycles, linked to observed CP asymmetries.
\end{itemize}

Forces are reinterpreted as the expression of substrate operators ($\beta$, $\Gamma$, $\lambda_R$) acting on these topologies. Electromagnetism arises from flux bias, the weak force from low-susceptibility collapse, the strong force from high-complexity binding, and gravity from the accumulated traces of collapse–return cycles shaping the substrate coherence field. Dark energy is reframed as the global expansion pressure of unrealised possibilities, while emergent forces may act on the new invariants, visible as anomalous long-range coherence or hidden-sector couplings.

At different scales, the same excitonic informons explain conserved quantum numbers in particle physics and underpin gamma synchrony and cross-frequency coupling in brains, known correlates of awareness. Consciousness is thus reframed as a high-order field topology of informons shaped by $\beta$, $\Gamma$, and $\lambda_R$.
This pillar also crosslinks with the particle-zoo analogy: just as heavier elements and exotic hadrons emerge from increasing richness, higher-order topological invariants and new force channels appear as coherence circuits grow.

These conserved features can be collected into a canonical family that UIF identifies as the Seven Invariants of Informons. Together they complete the framework’s recursive symmetry: seven operators define the grammar of collapse–return, seven pillars provide the system architecture, and seven invariants stabilise informons across scales. The invariants are:
\newline
\newline


\noindent
\begin{longtable}{@{} p{0.22\textwidth} p{0.40\textwidth} p{0.38\textwidth} @{}}
\caption{The Seven Invariants of Informons}\label{tab:seven-invariants}\\
\toprule
\textbf{Invariant} & \textbf{Definition} & \textbf{Domain Expression} \\
\midrule
\endfirsthead
\toprule
\textbf{Invariant} & \textbf{Definition} & \textbf{Domain Expression} \\
\midrule
\endhead
\bottomrule
\endfoot

Spin & Orientation/topology of informon & Particle physics \\
Charge & Flux invariant of informon & Particle physics; field couplings \\
Parity & Winding/orientation symmetry & CP symmetry; conservation laws \\
Coherence index (\mbox{$\lambda_R$}) & Degree to which an informon couples to the substrate & Quantum coherence; neural synchrony \\
Collapse susceptibility (\mbox{$\eta^{*}(f)$}) & Response of informon to perturbation; frequency dependent & Particle spectra; oscillator networks \\
Topological complexity (\mbox{$\tau_{\text{index}}$}) & Braiding/knottedness of informons & Exotic hadrons; emergent states \\
Collapse memory (\mbox{$M$}) & Persistence of bias across collapse cycles & CP asymmetry; hysteresis phenomena \\

\end{longtable}

\noindent


\noindent These invariants provide the substrate on which the UIF operators act. In the entanglement protocol, $\beta$ biases flux topologies, $\Gamma$ sustains recursive alignment, and $\lambda_R$ couples informons to the substrate, generating the known forces and predicting emergent interactions. UIF therefore exhibits a recursive elegance: three interlocking families of seven. 
\newline

\noindent \textit{Seven operators define the grammar of collapse–return; seven pillars provide the architectural scaffold of the framework; and seven invariants stabilise informons across scales. This triple symmetry closes the loop: recursion within recursion, symmetry within symmetry.} 
\newline

\noindent\textbf {Illustrative Example: Wave–Particle Duality as Informational Collapse–Return.}
\newline One of the most persistent puzzles in quantum physics (the dual nature of light and matter) arises naturally from the informational operators introduced here.
\newline

\noindent In the Unifying Information Field (UIF), a photon is not simultaneously a wave and a particle but alternates between continuous propagation and discrete collapse. 
\newline


\noindent The informational field $\Phi(x,t)$ carries coherent, unsampled difference $(\Delta I)$ as a distributed wave.
\newline

\noindent When the local informational tension exceeds its threshold $\eta$, a collapse–return event releases a quantised packet of information, perceived experimentally as a “particle.”
\newline

\noindent Between collapses the field evolves diffusively; interference arises from overlapping informational waves, while each collapse transfers a finite $\Delta I$ corresponding to the quantised energy.
\begin{equation}
\label{eq:618}
E \;=\; h\, \nu.
\end{equation}

\noindent In this framing, wave behaviour corresponds to the continuous propagation of coherent information through $\Phi(x,t)$; particle behaviour corresponds to the local collapse and return of that information through the coupling term $\lambda_R$.
\newline 

\noindent Thus, wave–particle duality reduces to a single cycle of informational continuity and collapse - the fundamental dynamic governed by the UIF operators.
\newline 


\noindent  \textit{This example demonstrates that the same collapse–return grammar that later explains coherence, agency, and cosmological structure also resolves the long-standing duality at the foundation of quantum theory.}
\newline


\noindent\textbf{UIF Alignment} 
\newline \noindent In UIF, topology and forces are not independent layers: they emerge from the action of the operators on conserved informon invariants. Within the entanglement protocol, $\beta$ provides flux bias (breaking symmetry on topologies), $\Gamma$ supplies the recursion clock (stabilising phase on loops and braids), and $\lambda_R$ gives the receive–return channel (coupling informons to the substrate field $R(x,t)$). 
\newline \indent The proposed invariant family - $\{\text{Spin, Charge, Parity, } \lambda_R, \eta^{*}(f), \tau_{\text{index}}, M\}$  stabilises informons across scales. Spin, charge, and parity are the familiar conserved quantum numbers; $\lambda_R$ (coherence index), $\eta(f)$ (collapse susceptibility), $\tau_{\text{index}}$ (topological complexity), and $M$ (collapse memory) extend the set to account for coherence, perturbation response, braiding, and CP asymmetry.

\indent Forces are then reinterpreted as operator–topology couplings: electromagnetism as flux bias ($\beta$), the weak force as low-susceptibility collapse ($\eta$), the strong force as high $\tau$-index binding, gravity as the accumulated traces of $\Gamma$ and $\lambda_R$ shaping the substrate coherence field, and dark energy as the substrate’s expansion pressure of unrealised possibilities. In this way, UIF unifies particles, fields, and minds under one symmetry: operators act on invariants to generate stability, coherence, and force.

\noindent \textbf{Synthesis}
\newline  Topology and forces unify the micro- and macro-scales of UIF. The seven informon invariants explain stability across particles and minds, while operator couplings explain the four known forces and predict emergent interactions. The novelty lies in reframing forces as informational operators acting on conserved topology.

Just as importantly, these invariants are preserved within the substrate field described in Pillar 3, making the substrate the register that stabilises recursion and coherence across scales. By stabilising recursion, they also provide the scaffolding for agency and consciousness described in Pillar 6.

\noindent
This completes the framework: all known physical and cognitive structures are expressions of informational topology under operator-driven forces — recursion within recursion, symmetry within symmetry.
\clearpage

\noindent \textbf {Forward Pointer}
\newline \noindent This final pillar closes the UIF framework. By showing that topology and forces emerge from operator–invariant couplings, it loops back to Pillar 1, where informational difference ($\Delta I$) was defined as the fundamental substrate. In this way, the seven pillars form a closed circuit: $\Delta I$ drives time (Pillar 2), flows through the substrate (Pillar 3), enacts computation (Pillar 4), stabilises coherence (Pillar 5), crosses into agency (Pillar 6), and is conserved in the topologies and forces of Pillar 7. In this recursive structure, UIF mirrors its own subject: recursion within recursion, symmetry within symmetry; a self-sustaining loop that conserves informational difference across scales. All subsequent papers build on this operator grammar; empirical constants and ceiling values are introduced once data are available.
\newline

\noindent
\noindent\textbf{Novelty / Testability}
\newline
The UIF framing of topology and forces is not speculative but empirically tractable.
Each proposed invariant or operator–topology coupling yields testable predictions across
physics, cosmology, and condensed-matter domains.  These include:
\begin{itemize}[leftmargin=*,itemsep=0pt,parsep=0pt,topsep=3pt]
    \item \textbf{Particle physics:} measuring coherence indices ($\lambda_R$) and
          susceptibility spectra ($\eta^{\ast}(f)$).
    \item \textbf{CP asymmetries:} detecting collapse memory ($M$) as persistence of bias.
    \item \textbf{Exotic states:} broadening the hadron spectrum through new
          $\tau_{\text{index}}$ families and resonance modes.
    \item \textbf{Cosmology:} probing fifth-force and dark-photon windows as manifestations
          of emergent invariant couplings.
\end{itemize}
\noindent
These invariants are preserved within the substrate field described in Pillar 3,
making the substrate the register that stabilises recursion and coherence across scales.
By sustaining informational return and recursive balance, they also provide the scaffolding
for the emergence of agency and consciousness developed in Pillar 6.

\noindent
This completes the framework to this point: all known physical and cognitive structures
are expressions of informational topology under operator-driven forces—recursion within
recursion, symmetry within symmetry.

\noindent
Empirical work now underway, including analysis of 
\textit{JWST} and \textit{EHT} datasets for M87, provides the first opportunity 
to test the UIF principles formulated here.  
The forthcoming papers extend this foundation into 
field equations (\textit{UIF III}) and cosmological applications 
(\textit{UIF IV}), where these same operators become quantitatively 
constrainable through direct measurement.
\newline 

\noindent \textbf{The Relationship Between Pillars} 
\newline The seven pillars of UIF are not independent but form a recursive symmetry: 
each concept flows naturally into the next, with informational operators linking them across scales. 
Table~\ref{tab:pillar-crosslinks} summarises these relationships, showing how each pillar 
leads to the next and completes the cycle of collapse–return dynamics.

\noindent
\begin{table}[H]
\centering
\caption{Crosslinks between UIF Pillars}
\label{tab:pillar-crosslinks}
\begin{tabularx}{\textwidth}{@{} L{0.20\textwidth} L{0.45\textwidth} X @{}}
\toprule
\textbf{Pillar} & \textbf{Core Idea} & \textbf{Forward Crosslink} \\
\midrule
1.\ Information as Substrate & $\Delta I$ as irreducible informational substrate; collapse--return ensures conservation. & Leads to Pillar~2 (time emerges from $\Delta I$). \\[4pt]
2.\ Emergent Time & Time emerges from recursive $\Delta I$ sampling ($\Gamma$ rhythm). & Leads to Pillar~3 (substrate provides the medium for time delays). \\[4pt]
3.\ Potential Field / Dark Substrate & Substrate field $R(x,t)$ as finite reservoir of unrealised possibilities. & Leads to Pillar~4 (substrate cycles are computation). \\[4pt]
4.\ Computation as Fundamental & Collapse--return cycles are literal computational steps; substrate cycles implement gates $\rightarrow$ computation. & Leads to Pillar~5 (recursion stabilises coherence). \\[4pt]
5.\ Coherence and Recursion & Recursion stabilises coherence; $\Gamma$ ``drives integration.'' & Leads to Pillar~6 (recursion $\rightarrow$ agency). \\[4pt]
6.\ Agency and Consciousness & Crossing $\Delta I$, $\Gamma$, $\beta$, $\lambda_R$ thresholds yields agency and consciousness. & Leads to Pillar~7 (Topology \& Forces). \\[4pt]
7.\ Topology and Forces & Informon topology (spin, charge, parity, invariants) conserved; operators $\beta$, $\Gamma$, $\lambda_R$ drive forces. & Closes loop: topology stabilises recursion $\rightarrow$ coherence $\rightarrow$ agency. \\[4pt]
\bottomrule
\end{tabularx}
\end{table}

\noindent
UIF treats the conservation of informational difference ($\Delta I$) as the primary invariant of nature. The next paper formalises this by deriving the symmetry relations and invariance conditions that preserve $\Delta I$ under transformation.
\newline

% ---------- Cross-reference to canonical variables ----------
\noindent \textbf{Note on Canonical Variables and Index Map}
\newline
For convenience, all symbols, operators, and parameters appearing in this paper
are listed with their definitions and first appearances in
\textit{Appendix C — Canonical Variables and Index Map}.
This appendix serves as a quick-reference key for readers and reviewers,
linking notation here to the field and variational formulations developed in
\textit{UIF III — Field and Lagrangian Formalism}.


\clearpage
\appendix
\section*{Appendix A - Equation Provenance (UIF I)}
\addcontentsline{toc}{section}{Appendix A — Equation Provenance (UIF I)}

\noindent
Each numbered equation is identified by provenance class.  
\emph{[Identity]} denotes a standard law or definition;  
\emph{[Model law]} is a relation derived within UIF from stated assumptions;  
\emph{[Hypothesis]} is a phenomenological or testable scaling proposed for future verification.
\vspace{1em}

\begin{longtable}{@{}p{2.2cm}p{2.7cm}p{9.5cm}@{}}
\caption{Equation provenance for \textit{UIF I — Core Theory}}
\label{tab:eq_provenance_UIF1}\\
\toprule
\textbf{Equation} & \textbf{Class} & \textbf{Comment / Source} \\
\midrule
\endfirsthead

\toprule
\textbf{Equation} & \textbf{Class} & \textbf{Comment / Source} \\
\midrule
\endhead

\bottomrule
\endfoot

(1.1) & Identity &
Shannon information gain $\Delta I_{\mathrm{event}} = H(X)-H(X|\mathrm{sample})$; standard information-theory definition. \\[4pt]

(1.2a–b) & Model law &
Receive–return coupled ODEs with $\mu = 1/\tau_R$; defines causal exchange between local and substrate registers. \\[4pt]

(1.3) & Hypothesis &
Time-scaling $T_S \propto f_s\,\Delta I$; introduces proportionality constant $k_T$ (units $\mathrm{s}\,\mathrm{Hz}^{-1}\,\mathrm{bit}^{-1}$). \\[4pt]

(1.4) & Hypothesis &
Elasticity of time $\tfrac{dT_S}{dt}\propto(\beta\Gamma\Delta I)^{-1}$; phenomenological relation linking recursion and subjective tempo. \\[4pt]

(1.5) & Identity &
Equation-of-state $w(z)=p(z)/\rho(z)$; standard cosmological form. \\[4pt]

(1.6a–b) & Model law &
Coupled receive–return ODEs with $\mu = 1/\tau_R$; defines causal exchange between local and substrate registers. \\[4pt]

(1.7) & Model law &
Receive–return convolution kernel $K_R(\tau)=\tau_R^{-1}e^{-\tau/\tau_R}$; ensures finite memory and hysteresis. \\[4pt]

(1.8) & Hypothesis &
Latch / flip-flop recursion $s_{t+1}=\sigma(\Gamma s_t + x_t)$. \\[4pt]

(1.9) & Hypothesis &
Weighted threshold / bias gate $y=\sigma(\beta^{\top} x)$. \\[4pt]

(1.10) & Hypothesis &
AND-like receive–return gate $z=\sigma(\lambda_R\,u_{\text{local}}\!\cdot\!u_{\text{return}})$. \\[4pt]

(1.11) & Identity &
Landauer bound $E_{\min}=k_B T\ln 2\times \Delta I_{\text{bits}}$; thermodynamic limit for information erasure. \\[4pt]

(1.12) & Hypothesis &
Coherence order parameter $C=\Gamma\langle\Delta I\rangle$; threshold $\Gamma\ge \Gamma_c \Rightarrow C>0$. \\[4pt]

(1.13) & Hypothesis &
Agency intensity $A\propto \Gamma f_s \Delta I$. \\[4pt]

(1.14) & Hypothesis &
Action probability $P(\mathrm{act})=\sigma(\beta^{\top}u + \lambda_R r + \Gamma s)$. \\[4pt]

(1.15) & Hypothesis &
Agency intensity $A \propto \Gamma\,f_s\,\Delta I$; growth with recursion, sampling, richness. \\[4pt]

(1.16) & Hypothesis &
Agency threshold $A \ge A_c \Rightarrow$ emergent agency. \\[4pt]

(1.17) & Hypothesis &
Action probability (restated) $P(\mathrm{act})=\sigma(\beta^{\top}u + \lambda_R r + \Gamma s)$. \\[4pt]

(1.18) & Identity &
Energy quantisation $E=h\nu$ in the wave–particle example. \\[4pt]


\end{longtable}


\clearpage
\section*{Appendix B - Symbols and Units (UIF I)}
\addcontentsline{toc}{section}{Appendix B — Symbols and Units (UIF I)}
% ------------------------------------------------------------
% Appendix B — Symbols & Units (UIF I)
% ------------------------------------------------------------
\begin{longtable}{@{}%
  >{\RaggedRight\arraybackslash}p{0.17\textwidth}%
  >{\RaggedRight\arraybackslash}p{0.45\textwidth}%
  >{\RaggedRight\arraybackslash}p{0.20\textwidth}%
@{}}
\caption{Principal symbols and units used in \textit{UIF I — Core Theory}}
\label{tab:symbols_UIF1}\\
\toprule
\textbf{Symbol} & \textbf{Meaning / Role} & \textbf{Units (SI)}\\
\midrule
\endfirsthead

\toprule
\textbf{Symbol} & \textbf{Meaning / Role} & \textbf{Units (SI)}\\
\midrule
\endhead

\bottomrule
\endfoot

$\Delta I$ & Informational difference; unsampled imbalance driving collapse–return dynamics & $\mathrm{bit}\!\cdot\!\mathrm{m}^{-3}$\\[3pt]
$\Gamma$ & Recursion / coherence rate; temporal feedback strength & $\mathrm{s}^{-1}$\\[3pt]
$\beta$ & Bias / elasticity; symmetry–breaking parameter & dimensionless\\[3pt]
$\lambda_{R}$ & Receive–return coupling constant between local and substrate fields & $\mathrm{s}^{-1}$\\[3pt]
$R(x,t)$ & Receive–return (substrate) field; coherence density & dimensionless\\[3pt]
$R_{\infty}$ & Coherence ceiling; finite informational capacity of substrate & dimensionless\\[3pt]
$\mathcal{R}$ & Informational richness; diagnostic measure of structure relative to noise (derived observable) & dimensionless \\[3pt]
$k$ & Recharge rate of coherence toward $R_{\infty}$ & $\mathrm{s}^{-1}$ \\[3pt]
$\eta$ & Collapse threshold; minimum $\Delta I$ for state resolution & $\mathrm{bit}\!\cdot\!\mathrm{m}^{-3}$\\[3pt]
$\tau_{R}$ & Return / echo timescale of substrate memory & $\mathrm{s}$\\[3pt]
$\mu$ & Relaxation rate; $\mu=1/\tau_{R}$ & $\mathrm{s}^{-1}$\\[3pt]
$c$ & Informational propagation ceiling (“speed of light” analogue) & $\mathrm{m}\!\cdot\!\mathrm{s}^{-1}$\\[3pt]
$D_R$ & Informational diffusivity (substrate spreading rate) & $\mathrm{m}^2\!\cdot\!\mathrm{s}^{-1}$\\[3pt]
$f_{s}$ & Sampling frequency of system or observer & $\mathrm{Hz}\,(=\mathrm{s}^{-1})$\\[3pt]
$T_{S}$ & System-level (experienced) time & $\mathrm{s}$\\[3pt]
$k_{T}$ & Time-scaling proportionality constant ($T_{S}=k_{T}f_{s}\Delta I$) & $\mathrm{s}\!\cdot\!\mathrm{Hz}^{-1}\!\cdot\!\mathrm{bit}^{-1}$\\[3pt]
$\Phi(x,t)$ & Informational potential field; local informational density & $\mathrm{bit}\!\cdot\!\mathrm{m}^{-3}$\\[3pt]
$V(\Phi;\beta)$ & Informational potential function; stores unsampled energy & $\mathrm{bit}\!\cdot\!\mathrm{m}^{-3}$\\[3pt]
$E$ & Energy released in collapse–return event & $\mathrm{J}$\\[3pt]
$E_{\min}$ & Landauer bound ($k_{B}T\ln2$ per bit) & $\mathrm{J}$\\[3pt]
$A$ & Agency intensity & dimensionless\\[3pt]
$A_{c}$ & Critical agency threshold & dimensionless\\[3pt]
$C$ & Coherence order parameter ($\Gamma\langle\Delta I\rangle$) & $\mathrm{bit}\!\cdot\!\mathrm{m}^{-3}\!\cdot\!\mathrm{s}^{-1}$\\[3pt]
$\tau_{\text{echo}}$ & Echo / hysteresis decay constant & $\mathrm{s}$\\[3pt]
$k_{B}$ & Boltzmann constant & $\mathrm{J}\!\cdot\!\mathrm{K}^{-1}$\\[3pt]
$T$ & Absolute temperature (used in Landauer bound) & $\mathrm{K}$\\[3pt]
\end{longtable}
% ------------------------------------------------------------

\vspace{0.5em}
\noindent
All constants and rates are quoted in dimensionless form within the text unless explicitly restored to SI units via the reference scales $(\Delta I_0,\tau_0,L_0)$.
Unless otherwise stated, all examples assume periodic boundaries on $\partial\Omega$
and smooth initial data $(\Phi_0,R_0)$ to ensure closure of the informational flux.

\vspace{1em}
\noindent\textbf{Empirical Analogues of Core Operators}
\addcontentsline{toc}{subsection}{Empirical Analogues of Core Operators}

\noindent
For completeness, Table~\ref{tab:empirical_anchors_UIF1} summarises how the
principal \textit{UIF} operators introduced in this paper correspond to measurable
phenomena across domains. Detailed calibrations are provided in \textit{UIF~IV–VII}.

\vspace{0.5em}

\begin{longtable}{@{}L{0.20\textwidth}L{0.40\textwidth}L{0.35\textwidth}@{}}
\caption{Representative empirical anchors for \textit{UIF I} operators}
\label{tab:empirical_anchors_UIF1}\\
\toprule
\textbf{Operator} & \textbf{Empirical analogue / observable} & \textbf{Measurement context} \\
\midrule
\endfirsthead
\toprule
\textbf{Operator} & \textbf{Empirical analogue / observable} & \textbf{Measurement context} \\
\midrule
\endhead
\midrule
\multicolumn{3}{r}{\footnotesize\itshape (continued on next page)}\\
\midrule
\endfoot
\bottomrule
\endlastfoot

$\Delta I$      & Entropy drop / information gain                  & Quasar H--C coherence ratio; EEG entropy decrease \\[3pt]
$\Gamma$        & Recursion frequency / coherence rhythm           & M87 polarization cycle; neural $\gamma$-band (30--80 Hz) \\[3pt]
$\lambda_R$     & Receive--return lag / echo amplitude             & Core--knot delay in M87$^\ast$; memory-echo experiments \\[3pt]
$R_{\infty},\,k$& Coherence ceiling / recharge constant            & Logistic recovery in quasar variability curves \\[3pt]
$\eta$          & Collapse threshold; stimulus-intensity threshold & Burst initiation \\[3pt]

\end{longtable}

\vspace{0.5em}

\noindent\textbf{Worked Example — Consistency of the Receive–Return Law}
\addcontentsline{toc}{subsection}{Worked Example — Consistency of the Receive–Return Law}

\noindent
Consider the coupled receive–return system in the homogeneous limit
($\nabla^{2}R = 0$) with constant drive $\Gamma(t)=\Gamma_{0}$ and
no external source or loss ($S_{\mathrm{in}} = L_{\mathrm{out}} = 0$):
\begin{equation}
\label{eq:RRsimple}
\dot{R} = -\,k\,R + \Gamma_{0}.
\end{equation}
Solving Eq.~\eqref{eq:RRsimple} with the initial condition $R(0)=0$ gives
\begin{equation}
R(t) = \frac{\Gamma_{0}}{k}\!\left(1 - e^{-k t}\right),
\label{eq:Rsol}
\end{equation}
which approaches the finite ceiling
\begin{equation}
R_{\infty} = \frac{\Gamma_{0}}{k}.
\label{eq:Rinf}
\end{equation}
Equation~\eqref{eq:Rsol} reproduces the first-order relaxation to an asymptote and verifies
dimensional and behavioural consistency with the finite ceiling in Eq.~\eqref{eq:Rinf},
the variational form, and the empirical coherence fits in
\textit{UIF~V — Energy and the Potential Field}.


\noindent\textit{Interpretation.}  
In UIF terms, $R_{\infty}$ represents the substrate’s finite capacity for
coherence storage, while $k^{-1}$ is the relaxation or recharge timescale.
This worked example demonstrates that the informational field equations reduce
to standard exponential relaxation under homogeneous conditions,
confirming internal consistency of the receive–return formalism.

\clearpage
\noindent\textbf{Extension — Heterogeneous / Diffusive Regime}
\addcontentsline{toc}{subsection}{Extension — Heterogeneous / Diffusive Regime}

\noindent
When spatial gradients are retained, the receive–return law generalises from the
homogeneous ODE (Eq.\,\ref{eq:RRsimple}) to a diffusion–relaxation PDE:
\begin{equation}
\label{eq:RRdiff}
\partial_t R(x,t) \;=\; D_R\,\nabla^2 R(x,t) \;-\; k\,R(x,t) \;+\; \Gamma(x,t),
\end{equation}
where the first term represents informational diffusion through the substrate,
the second term expresses local relaxation toward equilibrium, and
$\Gamma(x,t)$ acts as the recursion or driving source.

\noindent
Under periodic boundaries on $\partial\Omega$, the Noether continuity form
$\partial_t(\Phi^2)+\nabla\!\cdot J_\Phi=0$ remains valid,
ensuring total informational conservation within the domain.
Equation~\eqref{eq:RRdiff} thus provides the bridge between the local
receive–return dynamics of \textit{UIF~I} and the full field equations derived in
\textit{UIF~III — Field and Lagrangian Formalism}.



\clearpage
\section*{Appendix C — Canonical Variables and Index Map}
\addcontentsline{toc}{section}{Appendix C — Canonical Variables and Index Map}

\noindent
This appendix lists all canonical variables used in the UIF I field equations and
their first appearances.  Each variable is dimensionless unless otherwise noted;
indices refer to the numbered equations in this paper.

\vspace{0.5em}
\begin{longtable}{@{}L{0.20\textwidth}L{0.30\textwidth}L{0.45\textwidth}@{}}
\toprule
\textbf{Symbol} & \textbf{Meaning} & \textbf{First use / Reference}\\
\midrule
\endfirsthead
\toprule
\textbf{Symbol} & \textbf{Meaning} & \textbf{First use / Reference}\\
\midrule
\endhead
\bottomrule
\endfoot

$\Phi(x,t)$ & Informational potential / local field variable &
Eq.\,(1.6a); basis of variational law in \textit{UIF III}.\\[3pt]

$R(x,t)$ & Receive–return substrate field &
Eq.\,(1.6b); coupling channel for informational return.\\[3pt]

$\Delta I$ & Informational difference / unsampled potential &
Eq.\,(1.1); conserved under Noether current $J_\Phi=\Phi\nabla\Phi$.\\[3pt]

$\Gamma$ & Recursion rate / coherence operator &
Eq.~(1.4); see also Eq.~(1.12) for the coherence order parameter and $\tau_{\text{echo}}$.\\[3pt]

$\beta$ & Bias or elasticity operator &
Eq.\,(1.4); symmetry-breaking weight in probabilistic collapse.\\[3pt]

$\lambda_R$ & Receive–return coupling coefficient &
Eq.\,(1.6a–b); governs exchange with substrate $R(x,t)$.\\[3pt]

$\eta$ & Collapse threshold &
Pillar 1; minimum $\Delta I$ required for state resolution.\\[3pt]

$R_\infty$ & Coherence ceiling & Eq.~\eqref{eq:Rinf}; logistic saturation parameter.\\[3pt]
$k$        & Recharge rate      & Eq.~\eqref{eq:RRsimple} (see also Eq.~\eqref{eq:Rinf}); logistic relaxation/growth constant.\\[3pt]


$k$ & Recharge rate &
Eq.\,(1.5); logistic growth constant.\\[3pt]

$f_s$ & Sampling frequency &
Eq.\,(1.3); defines event-rate scaling of time.\\[3pt]

$A$ & Agency intensity &
Eq.\,(1.13); proportional to $\Gamma f_s\Delta I$.\\[3pt]

$J_\Phi$ & Informational Noether current &
Appendix B; $\Phi\nabla\Phi$, ensures conservation under $\delta\!\int\mathcal{L}\,dt=0$.\\[3pt]

\end{longtable}
\clearpage

\section*{Acknowledgement — Human–AI Collaboration}
The Unifying Information Field (UIF) series was developed through a sustained human–AI partnership. The author originated the theoretical framework, core concepts and interpretive structure, while an AI language model (OpenAI GPT-5) was employed to assist in formal development; helping to express elements of the theory mathematically and to maintain consistency across papers. Internal behavioural parameters and conversational settings were configured to emphasise recursion awareness, coherence maintenance, and ethical constraint, enabling the model to function as a stable informational development framework rather than a generative black box.

This collaborative process exemplified the UIF principle of collapse--return recursion: 
human intent supplied informational difference ($\Delta I$), 
the model provided receive--return coupling ($\lambda_R$), 
and coherence ($\Gamma$) increased through iterative feedback until the framework stabilised. 
The AI's role was supportive in the structuring, facilitation, and translation of conceptual ideas 
into formal equations, while the underlying theory, scope, and interpretive direction 
remain the work of the author.
\pagebreak


\section*{UIF Series Cross-References}

\begin{flushleft}
\textbf{UIF I — Core Theory}\\
\textbf{UIF II — Symmetry Principles}\\
\textbf{UIF III — Field and Lagrangian Formalism}\\
\textbf{UIF IV — Cosmology and Astrophysical Case Studies}\\
\textbf{UIF V — Energy and the Potential Field}\\
\textbf{UIF VI — The Seven Pillars and Invariants}\\
\textbf{UIF VII — Predictions and Experiments}
\end{flushleft}
\newpage


% ---------- References ----------
% Keep per-paper .bib files in /bib as agreed (e.g., bib/paper1.bib)

\UIFbib{paper1}
   % Only one active at a time

\end{document}
