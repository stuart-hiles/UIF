% ===== UThe Unifying Information Field (UIF) --- Companion Experiments =====
% Numbering: Paper 1 → (1.x) equations, figures, tables
%\UIFpaper{1}

\UIFmetadata{The Unifying Information Field (UIF) --- Companion Experiments: Empirical Validation of Papers I--IV}
            {Stuart E. N. Hiles}
            {The Unifying Information Field (UIF) --- Companion Experiments}

\hypersetup{
  pdfauthor   = {Stuart E. N. Hiles},
  pdftitle    = {The Unifying Information Field (UIF) --- Companion Experiments: Empirical Validation of Papers I--IV},
  pdfsubject  = {Empirical validation of UIF; UIF emulator; collapse--return dynamics; operator calibration; coherence ceiling R_infty; recharge rate k; receive--return coupling lambda_R; recursion rate Gamma; symmetry thresholds; hysteresis; cosmology-lite model; reproducibility},
  pdfkeywords = {UIF, Unifying Information Field, empirical validation, informational physics, collapse--return, lambda_R, R_infty, k, Gamma, symmetry breaking, hysteresis, Goldilocks stability, cosmology-lite emulator, reproducibility, data transparency}
}

\title{The Unifying Information Field (UIF) Companion Experiments:\\[0.35em]
\Large\textit{Empirical Validation of Papers I--IV}\\[0.6em]   
\small Version v2.0 — November 2025}
\author{Stuart E.\,N. Hiles, BA (Hons)}
\date{}

\begin{center}
\thispagestyle{empty}
\vspace{2em}
{\small
© 2025 Stuart E. N. Hiles \\[4pt]
Licensed under the Creative Commons Attribution–NonCommercial International 4.0 
\newline (CC BY-NC 4.0) License. \\[6pt]
This document represents a pre-release version (v2.0, November 2025) of the  
\textit{Unifying Information Field (UIF)} series of papers. \\[6pt]
First published on GitHub: \url{https://github.com/stuart-hiles/UIF} \\[4pt]
DOI (Concept): \href{https://doi.org/10.5281/zenodo.17478715}{110.5281/zenodo.17478715} \\[4pt]
Series DOI: \href{https://doi.org/10.5281/zenodo.17434412}{10.5281/zenodo.17434412} \\[4pt]
Commit ID: \texttt{6192db8} \\[10pt]
This paper has not yet been peer-reviewed or formally published. \\[6pt]
All supporting software, scripts, and data are licensed separately under \textbf{GPL-3.0}.
\end{center}

\maketitle

\begin{abstract}
\thispagestyle{empty}
\noindent
This document reports methods, datasets, parameter sweeps, and emulator outputs
that empirically validate the UIF series (Papers~I--IV). Operator-specific
measurements and posteriors are presented for the coherence ceiling ($R_\infty$),
recharge rate ($k$), receive--return coupling ($\lambda_R$), recursion rate
($\Gamma$), collapse threshold ($\eta^{\ast}$), and bias ($\beta$).
Experiments align with the main papers as:
\textbf{Experiment~I} (Paper~I: core calibration),
\textbf{Experiment~II} (Paper~II: symmetry \& thresholds),
\textbf{Experiment~III} (Paper~III: variational collapse--return),
and \textbf{Experiment~IV} (Paper~IV: cosmology---lite emulator).
All code and data locations are listed in Appendix~\nameref{app:repro}.
\newline
\end{abstract}
\clearpage
\thispagestyle{empty}

\noindent\textbf{Series overview} 
\newline \noindent Paper~I introduces the Unifying Information Field (UIF) as a collapse--return informational framework and defines its operator grammar;
Paper~II develops the symmetry and invariance principles underlying informational conservation;
Paper~III establishes the field and Lagrangian formalism;
Paper~IV applies the framework to cosmology and astrophysical case studies;
Paper~V formulates the energetic and potential field laws; and
Paper~VI presents the seven pillars and invariants, consolidating the theoretical architecture of UIF across physical, biological, and cognitive and artificial domains.
\vspace{0.4em}
\newline

\noindent\textbf{Companion}
\newline\noindent  Experimental methods, emulator sweeps, operator calibration results, and reproducibility metadata supporting this series are presented in the \textit{UIF Companion Experiments} (2025) \cite{Companion2025}. \newline\textit{Scope}
This document serves as the empirical companion to \textit{UIF~Papers~I–IV},
providing the reproducible experiments, operator calibrations, and emulator
outputs that underpin the theoretical framework.
It consolidates the numerical and analytical foundations introduced under the
working title \textit{UT26} and formalised in the UIF series.
\newline

\noindent\textbf{Repository}
\newline
Source code, emulator outputs, and figure-generation scripts are maintained in the
UIF GitHub Archive (\url{https://github.com/stuart-hiles/Unifying-Information-Field}),
together with datasets supporting \textit{UIF~Papers~I–V} and the Companion series.
Each experiment is versioned by \texttt{RUN\_TAG} with configuration files, logs,
and figures archived for reproducibility.
\newline

\noindent\textbf{Forthcoming work}
\newline
Paper~VII (\textit{UIF~VII — Predictions and Experiments}) will complete the core series by presenting cross-domain predictions, coherence thresholds, and proposed experimental validations.
An additional \textit{UIF~Companion II — Extended Experiments} (forthcoming, 2026) will expand the empirical programme beyond the current emulator framework, incorporating biological, AI-domain, and collective synchronisation studies.
\newline

\noindent\textbf{Note on Nomenclature and Continuity}
\newline The Unifying Information Field (UIF) framework presented here evolved from an
earlier working title, \textit{UT26}. All emulator runs and parameter
conventions remain continuous with UT26, but the theory and notation have been
consolidated into the UIF series (Papers~I---IV) for formal publication.
\clearpage
% ===========================================================
\pagenumbering{arabic}
\setcounter{page}{1}
\section*{Introduction}
\noindent
The Unifying Information Field (UIF) framework models reality as a
collapse–return informational field in which informational difference
($\Delta I$) is conserved and redistributed through recursive coupling
($\lambda_R$) within a finite substrate ($R_\infty$).
Across the UIF series, Papers~I–IV establish the theoretical structure:
operator grammar and conservation laws (UIF~I---II),
field and Lagrangian formalism (UIF~III),
and cosmological applications (UIF~IV).

\noindent
The present paper provides the empirical foundation of that framework.
It reports the numerical experiments, parameter sweeps, and emulator
outputs used to calibrate the UIF operators
($\Delta I$, $\Gamma$, $\beta$, $\lambda_R$, $\eta^{\ast}$, $R_\infty$, $k$)
and to verify their predicted relations across scales.
These experiments test the UIF variational and symmetry principles in
synthetic and cosmological contexts, establishing the quantitative
values cited in the main series.

\noindent
Each experiment corresponds to one theoretical tier of UIF:
\textbf{Experiment~I} calibrates informational diffusion and coherence;
\textbf{Experiment~II} measures symmetry‐breaking thresholds;
\textbf{Experiment~III} validates the variational collapse–return law;
and \textbf{Experiment~IV---V} extend the emulator to cosmological scales,
producing the operator calibration used in \textit{UIF~IV}.
Together they provide the reproducible bridge between the UIF field
equations and measurable observables.

Current cosmological models such as $\Lambda$CDM have achieved remarkable success in describing the universe’s large-scale structure, background radiation, and expansion history. 
Their predictive precision across multiple observational domains—baryon acoustic oscillations, cosmic microwave background anisotropies, and weak-lensing statistics—demonstrates the coherence of the standard model at cosmological scales. 
However, these frameworks remain phenomenological, relying on dark-energy and dark-matter components whose informational or physical basis is not yet understood.

Yet tensions remain. The Hubble constant ($H_{0}$) discrepancy between local and CMB-based measures 
points to model incompleteness \cite{Verde2019,Riess2022}. 
Curvature debates \cite{DiValentino2020} and proposals of contraction scenarios 
\cite{BoyleFinnTurok2023} highlight that fundamental questions are unresolved. 
UIF complements $\Lambda$CDM by reframing the cosmos informationally: 
the universe is the evolution of an informational field $\Phi(x,t)$, 
coupled via $\lambda_R$ to a receive--return field $R(x,t)$. 
Collapse--return dynamics govern its initial state, expansion, horizons, topology, entropy budgets, and possible fates. 
The aim is not to replace $\Lambda$CDM but to express its empirical signatures through 
the informational operators of UIF, providing an alternative parameterisation testable against DESI, Euclid, and LSST data.
\newline 

\noindent \textbf{Informational operators and cosmological mapping}
\newline The UIF operators describe how information evolves and stabilises across scales:

\noindent $\Delta I$ quantifies informational difference or potential, $\Gamma$ measures recursion and coherence, $\beta$ encodes symmetry breaking and bias, $\lambda_R$ defines the coupling between local systems and the substrate field $R(x,t)$, and $\eta$ sets the threshold at which collapse occurs.

\noindent In cosmology, these parameters manifest as density perturbations, feedback processes, coupling constants, and critical thresholds that regulate structure growth and expansion dynamics.
\newline 

\noindent\textbf{UIF framework and theoretical foundations}
\newline The cosmological model developed here is grounded in the seven-pillar architecture established across the earlier UIF papers. These pillars describe the progression from information as substrate, through emergent time and potential fields, to computation, coherence, agency, and conserved topological invariants. The present work applies this integrated framework to cosmic scales, showing that the same informational laws governing collapse, recursion, and coherence also govern the universe’s large-scale structure and evolution.
\newline
\clearpage
\noindent \textbf{Relation to the Preceding Papers}
\newline \noindent
This Companion builds directly on \textit{UIF~I — Core Theory}, 
\textit{UIF~II — Symmetry Principles}, 
\textit{UIF~III — Field and Lagrangian Formalism}, 
and \textit{UIF~IV — Cosmology and Astrophysical Case Studies} 
\cite{UIF-I,UIF-II,UIF-III,UIF-IV}. 
The earlier papers established the theoretical structure of the Unifying Information Field (UIF): 
an informational substrate in which all systems evolve through collapse--return dynamics governed by 
operators $(\Delta I,\,\Gamma,\,\beta,\,\lambda_R,\,\eta^{\ast},\,R_\infty,\,k)$.
Together these define how informational difference is conserved, how coherence arises through recursion, 
and how stability is maintained within a finite coherence ceiling $R_\infty$.

\noindent
The present work provides the empirical validation of that framework. 
It reports the numerical experiments, parameter sweeps, and emulator outputs that calibrate each operator 
and verify their predicted relationships across scales—from local informational collapse to cosmological structure formation. 
While \textit{UIF~I–IV} established the theoretical and cosmological foundations, 
this Companion presents the measurable evidence that those same laws operate consistently in simulated informational fields. 
It therefore serves as the reproducible empirical counterpart to the main UIF series, 
linking theoretical constructs to directly observable quantities.
\newline

\noindent\textbf{Units and dimensional closure}
\newline
All dimensional statements inherit the SI‐consistent mapping defined in
\textit{UIF III — Appendix D}, including the information–energy constant
$\alpha$ (J bit$^{-1}$) and reference scales $(\Delta I_0,\tau_0,L_0)$.


% ===========================================================
%  METHODS NOTE — Cosmology–Lite Emulator Implementation
% ===========================================================
\section*{Methods / Supplementary Note: Cosmology–Lite Emulator}
\noindent
A lightweight three-dimensional lattice emulator was implemented in
\texttt{Python/Numpy} to demonstrate the collapse–return dynamics
predicted by the Unifying Information Field (UIF) framework.
This numerical code originated under an earlier working title of the theory,
\textit{UT26 (Unified Theory 2026)} \cite{UT26_Emulator}, which served as the prototype framework
for the informational field model later formalised as UIF.
The present version aligns with the operator grammar and equations
established across \textit{UIF I–IV}.

\noindent
A coherence field $s(x,t)$ of size $N\times N\times N$ evolves over
$T$ timesteps according to the operators
$\Delta I$, $\Gamma$, $\beta$, $\lambda_R$, and $\eta^{\ast}$.
Initial conditions use a Gaussian random field seeded with
BAO-like wiggles.

\noindent
The update rules incorporate recursion ($\Gamma$),
bias ($\beta$), substrate coupling ($\lambda_R$),
collapse threshold ($\eta^{\ast}$), and trace accumulation,
with an effective drift term coupling overdensity to substrate damping.
Outputs include mean coherence, entropy–complexity indices,
cumulative prunes, power spectrum $P(k)$, weak-lensing-like convergence
maps, and a toy halo-mass function (FoF on thresholded $\delta I$).

\noindent
Typical baseline runs used
$N=128$, $T=500$,
with operator parameters
$(\beta=3.0,\;\lambda_R\simeq0.20,\;\eta^{\ast}\simeq0.55,\;\Gamma\simeq0.9)$,
drive amplitude $\approx0.9$, and noise level $\approx0.3$.
These values produced the stable “Goldilocks-band” regime
described in \textit{UIF IV § 4.2}, matching analytical expectations for
collapse–return equilibrium.
Source code, configuration files, and derived datasets
(\texttt{ut26\_cosmo3d\_outputs})
are archived with this Companion document and referenced in the Supplementary Material. Full run tags, folders, and figure build scripts are listed in Appendix~\nameref{app:repro}.
\clearpage

% ===========================================================
\section*{Informational Operators and Measurement Mapping}

\noindent
For clarity, this Companion adopts the operator grammar defined in
\textit{UIF~I–IV}. The seven primary operators quantify the
informational dynamics measured in all experiments:
informational difference ($\Delta I$), recursion rate ($\Gamma$),
bias or elasticity ($\beta$), receive--return coupling ($\lambda_R$),
collapse threshold ($\eta^{\ast}$), coherence ceiling ($R_\infty$),
and recharge rate ($k$).
Each operator corresponds to a measurable quantity in the emulator, as summarised in Table~\ref{tab:operator-mapping}.
\newline

\noindent\textbf{Operator Provenance and Transparency}
\newline For transparency, each UIF operator listed in Table~\ref{tab:operator-mapping} is classified according to its
provenance, following the scheme used throughout the UIF series:
[Identity] denotes a quantity defined directly within the UIF theoretical framework or inherited from established physical law;
[Model law] designates a relation or parameter empirically derived from emulator measurements using UIF equations;
and [Hypothesis] indicates a proposed or testable scaling introduced for future validation.
A complete operator provenance matrix is provided in Appendix~\nameref{app:provenance}. 
\noindent Table~\ref{tab:operator-mapping} summarises how each UIF operator
is measured in the emulator and the corresponding observable used for calibration.
\vspace{0.5em}
\begin{longtable}{@{}L{0.17\textwidth}L{0.33\textwidth}L{0.45\textwidth}@{}}
\caption{Informational operators and their empirical observables}\label{tab:operator-mapping}\\
\toprule
\textbf{Operator} & \textbf{Conceptual Role} & \textbf{Measured Observable}\\
\midrule
\endfirsthead
\toprule
\textbf{Operator} & \textbf{Conceptual Role} & \textbf{Measured Observable}\\
\midrule
\endhead
\bottomrule
\endfoot
$\Delta I$ & Informational difference driving collapse–return. & Local variance and gradient magnitude of the field $s(x,t)$.\\[4pt]
$\Gamma$ & Recursion / coherence rate sustaining stability. & Drive frequency; mean temporal coherence $\langle s\rangle$.\\[4pt]
$\beta$ & Bias / elasticity controlling lawful asymmetry. & Softmax fit to hysteresis or symmetry-breaking slope.\\[4pt]
$\lambda_R$ & Receive–return coupling between system and substrate. & Ratio of cumulative prunes to trace integral $R(t)$; echo amplitude.\\[4pt]
$\eta^{\ast}$ & Collapse threshold defining fragile vs.\ stable regimes. & Transition boundary in $\gamma$-sweep and Goldilocks maps.\\[4pt]
$R_\infty$ & Finite coherence ceiling / saturation level. & Logistic envelope of $R(t)$ or $C$–$H$ relation.\\[4pt]
$k$ & Recharge rate governing coherence recovery. & Exponential decay constant $k=1/\tau_R$ from hysteresis runs.\\
\end{longtable}

\noindent These mappings provide the basis for all parameter estimates reported
in Experiments~I–V and for the operator calibration summary in
Table~\ref{tab:operator-calibration}.

\clearpage
\section*{S-CLASS — UIF$\to$CLASS Mapping (Linear Regime)}
\addcontentsline{toc}{section}{S-CLASS — UIF→CLASS Mapping (Linear Regime)}

\noindent\textbf{Scope.}
This note provides the technical bridge between the UIF cosmology-lite framework
and standard Boltzmann solvers (CLASS/CAMB). It enables direct comparison with
$\Lambda$CDM likelihood pipelines (Cobaya, MontePython).

\noindent\textbf{Substitutions.}
Implement the UIF receive–return and recursion operators as:
\begin{itemize}[leftmargin=1.5em]
  \item \textbf{Growth sector:}\quad replace \(G\!\to\!G_{\rm eff}(k,z)\) and add the causal
        source term $\mathcal S[\lambda_R,k,\Gamma]$ with kernel
        \(K_R(\tau)=\tau_R^{-1}e^{-\tau/\tau_R}\), where $\tau_R=k^{-1}$.
  \item \textbf{Poisson sector:}\quad apply \(G_{\rm eff}=G[1-\epsilon(\lambda_R,k,z)]\)
        in \texttt{equations.c}.
  \item \textbf{Background DE:}\quad allow mild drift
        \(w(z)=-1+\varepsilon(z)\) (const., tanh, or weak-sinusoidal form).
\end{itemize}

\noindent\textbf{Parameter ranges / priors.}
\(R_\infty\!\in[0.85,0.93]\), \(k\!\in[0.3,1.2]~\mathrm{Gyr^{-1}}\),
\(\lambda_R\!\in[0.1,0.5]\), and \(|\varepsilon(z)|\!\lesssim\!0.05\).

\noindent\textbf{Unit tests.}
(i) $\lambda_R=0$ → recover $\Lambda$CDM; 
(ii) finite $\tau_R$ → causal, stable growth suppression; 
(iii) mock BAO recovery within DESI tolerance.

\noindent\textbf{Repository stub.}
Reference implementation: \texttt{uif-class/} branch in the UIF GitHub archive,
with \texttt{.param} templates and Cobaya interface scripts.


\clearpage

% ===========================================================
\section*{Experiment I --- Informational Difference Calibration (supports UIF I)}
\textbf{Purpose} 
\newline Calibrate the informational diffusion–coherence floor and establish baseline operator readouts for $(\Delta I,\lambda_R)$ under controlled conditions.
\newline

\noindent\textbf{Data} 
\newline Synthetic lattice fields ($N=96^{3}$, $T=300$) generated with default operator settings, and, where applicable, public astrophysical time-series for cross-check (e.g., quasar light-curves referenced in the main text).
\newline

\noindent\textbf{Methods} 
\newline Initialise lattice field $s(x,t)$ with random Gaussian perturbations; propagate using receive--return feedback; compute mean coherence $\langle s\rangle$, entropy $H(s)$, and gradient norms $\|\nabla s\|$ to characterise baseline diffusion versus return.
\newline

\noindent\textbf{Results}
\newline Baseline outputs (Table \ref{tab:exp1-baseline}, Figure S1) show that mean coherence 
$\langle s\rangle$ rises toward a finite ceiling ($R_\infty\simeq0.5$) while cumulative pruning increases 
linearly with timestep. Power‐spectrum and $\kappa$ projections retain BAO‐like structure, confirming 
stability of the default operator configuration.
\newline

\noindent\textbf{Interpretation}
\newline The baseline calibration confirms that informational diffusion and return processes reach a finite coherence ceiling ($R_\infty$) and follow lawful pruning behaviour. 
Mean coherence $\langle s\rangle$ saturates near 0.5 while cumulative pruning grows monotonically, demonstrating that collapse–return dynamics conserve informational difference and accumulate trace proportionally to system activity. 
These results establish the canonical baseline for subsequent sensitivity and stability experiments.
\vspace{0.5em}
\begin{longtable}{@{}L{0.30\textwidth}L{0.65\textwidth}@{}}
\caption{Experiment I: baseline parameters and outputs}\label{tab:exp1-baseline}\\
\toprule
\textbf{Parameter / Output} & \textbf{Value / Description}\\
\midrule
\endfirsthead
\toprule
\textbf{Parameter / Output} & \textbf{Value / Description}\\
\midrule
\endhead
\bottomrule
\endfoot
Grid & $N=96^3$; timesteps $T=300$\\
Operators & $(\lambda_R, \Gamma, \beta, \eta^{\ast})$ initial sweep ranges\\
Outputs & mean coherence $\langle s\rangle$, entropy $H(s)$, gradient norms $\|\nabla s\|$\\
Notes & Baseline diffusion vs.\ return characterisation\\
\end{longtable}
\vspace{-1.9em}
% ---------- Figure S1: Baseline emulator outputs (A–D) ----------
\begin{center}
\begin{tabular}{@{}c@{\hspace{0.8em}}c@{}}
  \includegraphics[width=0.47\linewidth]{figures/exp4A_mean_coherence.png} &
  \includegraphics[width=0.38\linewidth]{figures/exp4B_cumulative_pruning.png} \\
  \small (A) Mean coherence vs.\ time ($R_\infty$ ceiling) &
  \small (B) Cumulative pruning (lawful trace) \\[0.8em]
  \includegraphics[width=0.47\linewidth]{figures/exp4C_power_spectrum.png} &
  \includegraphics[width=0.48\linewidth]{figures/exp4D_kappa_map.png} \\
  \small (C) $P(k)$ with BAO preserved (log-y) &
  \small (D) Weak-lensing-like $\kappa$ projection \\
\end{tabular}

\vspace{0.4em}
\captionof{figure}{\textbf{Figure S1. Baseline emulator outputs.} Panels A–D show (A) finite coherence ceiling $R_\infty$, (B) lawful pruning and trace accumulation, (C) BAO-preserving power spectrum $P(k)$, and (D) weak-lensing-like $\kappa$ projection sensitive to $\eta^{\ast}$ and $\lambda_R$.}
\label{fig:baseline_S1}
\end{center}
\clearpage

% ===========================================================
\section*{Experiment II --- Symmetry \& Threshold Dynamics (supports UIF II)}
\textbf{Purpose} 
\newline Validate symmetry-breaking and threshold behaviour under $\gamma$-like forcing; measure $(\Gamma,\eta^{\ast},\beta)$.
\newline

\noindent\textbf{Data} 
\newline Synthetic lattice runs ($N=96^{3}$, $T=300$) swept across drive amplitude and frequency. 
Each run logs mean coherence $\langle s\rangle$, entropy $H(s)$, cumulative pruning, and collapse classification. 
Parameter sweeps were performed over amplitude $A\in[0.40,1.00]$ and frequency $f\in[0.02,0.12]$ cycles/step, matching the UIF II theoretical ranges for resonance-driven collapse thresholds.
\newline

\noindent\textbf{Methods} 
\newline Apply amplitude–frequency sweep; classify fragile / stable / runaway regimes; compute pruning activity and refined collapse criterion $|\langle s\rangle-0.5|>0.001$ or $>10\%$ deviation in pruning counts from baseline. 
Fit bias parameter $\beta$ from the softmax slope of the resulting stability surface (UIF II Eq.\,2.3).
\newline

\noindent\textbf{Results}
\newline Amplitude–frequency sweeps (Figure S2) reveal a threshold transition at $A\simeq0.55$. 
Total pruning activity and $\langle s\rangle$ remain stable below this amplitude but increase sharply 
above it, producing collapse only under $\gamma$-like, high-frequency forcing.
\newline

\noindent\textbf{Interpretation}
\newline The $\gamma$-sweep verifies that collapse–return dynamics remain stable under sub-threshold forcing and destabilise only above threshold, producing the predicted transition from fragile to active regimes. 
Pruning activity and mean coherence patterns show resonance-like behaviour near the threshold band, validating the UIF II prediction that symmetry breaking requires above-threshold, $\gamma$-like perturbations. 
This defines the sensitivity boundary of the informational field.
\noindent The sweep configuration and classifier settings are listed in \autoref{tab:exp2-grid}.
\vspace{0.5em}
\begin{longtable}{@{}L{0.26\textwidth}L{0.68\textwidth}@{}}
\caption{Experiment II: sweep grid and classifier summary}\label{tab:exp2-grid}\\
\toprule
\textbf{Quantity} & \textbf{Specification}\\
\midrule
\endfirsthead
\toprule
\textbf{Quantity} & \textbf{Specification}\\
\midrule
\endhead
\bottomrule
\endfoot
Amplitude sweep & $A \in [0.40,1.00]$, step $0.05$\\
Frequency sweep & $f \in [0.02,0.12]$ cycles/step, step $0.02$\\
Classifier & $\{0=\text{fragile},\,1=\text{stable-ceiling},\,2=\text{runaway}\}$\\
Bias fit & Softmax/Boltzmann (UIF~II Eq.\,2.3) $\Rightarrow\beta$ estimate\\
\end{longtable}

\begin{figure}[H]
  \centering
  \includegraphics[width=0.95\linewidth]{figures/exp2_gamma_sweep.png}
\caption{\textbf{Figure S2.} $\boldsymbol{\gamma}$–sweep heat-maps across drive amplitude ($A$) and frequency ($f$).
Panels show (\textit{A}) total pruning activity, (\textit{B}) final mean coherence $\langle s\rangle$, 
(\textit{C}) refined collapse classification (1 = yes, 0 = no), and (\textit{D}) 
$\log_{10}$-scaled pruning counts. 
Together these panels demonstrate that collapse--return dynamics remain stable under sub-threshold forcing 
and destabilise only for above-threshold, $\gamma$-like perturbations, consistent with 
\textit{UIF II --- Symmetry Principle} \cite{UIF-II}.}
\label{fig:exp2_gamma}
\end{figure}
\clearpage 

% ===========================================================
\section*{Experiment III --- Variational Collapse--Return Law (supports UIF III)}
\textbf{Purpose.} 
\newline Validate the Lagrangian/variational behaviour of the collapse--return law, 
including the coherence ceiling ($R_\infty$), recharge rate ($k$), and hysteresis memory; 
connect to observational analogues (e.g., M87 variability).
\newline

\noindent\textbf{Data} 
\newline Synthetic lattice runs ($N=96^{3}$, $T=700$) driven under a 
two-phase schedule: high forcing amplitude ($A = 0.95$) for the first 350 steps, followed by a 
reduced forcing amplitude ($A = 0.35$) for the next 350 steps. 
Each run logs mean coherence $\langle s\rangle$, cumulative pruning counts, and drive amplitude at every timestep, 
allowing reconstruction of the $\langle s\rangle$–$A$ trajectory and identification of hysteresis loops.
\newline

\noindent\textbf{Methods} 
\newline Apply the two-phase drive schedule described above; fit the recovery 
of mean coherence using an exponential or logistic function 
$R(t)=R_\infty-(R_\infty-R_0)e^{-t/\tau_R}$, with $k=1/\tau_R$ as the recharge constant. 
Evaluate loop width and non-return to baseline as indicators of informational memory. 
Compare derived parameters $(R_\infty, k, \lambda_R)$ with those inferred from 
independent datasets (e.g., cosmological hysteresis and M87 light-curve analogues). 

\noindent These fitted parameters also define the boundaries of the stable-ceiling regime, illustrated in the Goldilocks stability map below.
\newline

\noindent\textbf{Results}
\newline Parameter sweeps across $(\eta^{\ast},\lambda_R)$ (Figure S3) identify a stable‐ceiling plateau for 
$0.40\!\le\!\eta^{\ast}\!\le\!0.70$ and $0.10\!\le\!\lambda_R\!\le\!0.30$. 
Outside this band, fragile or runaway behaviour dominates and small-scale structure is lost.
\newline

\noindent\textbf{Interpretation}
\newline The Goldilocks map identifies a bounded region of stable-ceiling behaviour within $\eta^{\ast}=0.40$–$0.70$ and $\lambda_R=0.10$–$0.30$. 
Below this band, the system collapses trivially (fragile regime); above it, retention feedback drives runaway behaviour and erases structure. 
This confirms UIF’s robustness principle: coherent collapse–return dynamics persist only when operator coupling and threshold values remain within balanced limits.

\noindent  Parameter estimates derived from the Goldilocks stability region are summarised in Table~\ref{tab:exp3-goldilocks}.
\clearpage 
\noindent
Table~\ref{tab:exp3-goldilocks} summarises the calibrated operator ranges
identified within the ``Goldilocks'' stability band.  These values represent
the canonical parameter set used throughout the UIF series (Papers IV–VI) and
serve as priors for the S-CLASS cosmology bridge.  All dimensional quantities
follow the SI mapping from \textit{UIF III — Appendix D} with $k$ expressed in
$\mathrm{Gyr^{-1}}$.
\vspace{0.5em}

\begin{longtable}{@{}L{0.22\textwidth}L{0.18\textwidth}L{0.55\textwidth}@{}}
\caption{Experiment III: Goldilocks stability and variational parameter estimates (derived from \texttt{ut26\_cosmo3d\_outputs}).}. 

\label{tab:exp3-goldilocks}\\
\toprule
\textbf{Quantity / Operator} & \textbf{Estimate / Range} & \textbf{Description / Context}\\
\midrule
\endfirsthead
\toprule
\textbf{Quantity / Operator} & \textbf{Estimate / Range} & \textbf{Description / Context}\\
\midrule
\endhead
\bottomrule
\endfoot

$\eta^{\ast}$ (collapse threshold) &
$0.40$–$0.70$ &
Boundaries of the stable‐ceiling regime; defines transition between fragile (0) and runaway (2) domains.\\[4pt]

$\lambda_R$ (receive–return coupling) &
$0.10$–$0.30$ &
Retention coupling values yielding lawful coherence; lower $\lambda_R$ → fragile, higher → runaway.\\[4pt]

$R_\infty$ (coherence ceiling) & 0.85–0.93 &
Asymptotic coherence level measured within the Goldilocks plateau; typical
uncertainty ±0.03 from logistic fits.\\[4pt]

$\Gamma$ (recursion rate) &
$0.50 \pm 0.02$ &
Mean recursion maintaining stable coherence across sweeps.\\[4pt]

$k$ (recharge rate) &
$0.3$–$1.2~\mathrm{Gyr}^{-1}$ &
Derived exponential/logistic recovery constant from adjacent hysteresis fits.\\[4pt]

$\beta$ (bias / elasticity) &
$1.0 \pm 0.2$ &
Softmax slope fitted to symmetry‐breaking surface, defining bias elasticity within the stable band.\\[4pt]

\end{longtable}


% -----------------------------------------------------------
% Goldilocks map
% -----------------------------------------------------------
\setlength{\abovecaptionskip}{0.4em}
\setlength{\belowcaptionskip}{-0.4em}
\begin{figure}[H]
  \centering
  \includegraphics[width=1.1\linewidth]{figures/exp3_goldilocks.png}
\caption{\textbf{Figure S3.} Goldilocks stability map across collapse threshold ($\eta^{\ast}$)
and retention ($\lambda_R$). 
The plateau at class 1 confirms the stable-ceiling regime 
($\eta^{\ast}=0.40$–$0.70$, $\lambda_R=0.10$–$0.30$),
bounding the fragile (0) and runaway (2) domains.}
\label{fig:exp3-goldilocks}
\end{figure}
\clearpage 
% ===========================================================
\section*{Experiment IV --- Hysteresis Probe: Informational Memory in Collapse--Return Dynamics (supports UIF III)}

\textbf{Purpose} 
\newline Test the UIF prediction that every collapse leaves a trace, producing persistent informational memory visible as hysteresis when the system is driven strongly and then released.
\newline

\noindent\textbf{Data} 
\newline Two-phase drive schedule: high forcing amplitude ($A=0.95$) for 350 steps followed by reduced forcing ($A=0.35$) for 350 steps. The emulator logged drive amplitude, mean coherence $\langle s\rangle$, and cumulative pruning at each timestep to reconstruct the $\langle s\rangle$--$A$ trajectory.
\newline

\noindent\textbf{Methods} 
\newline Apply the two-phase schedule; fit the recovery of mean coherence using 
$R(t)=R_\infty-(R_\infty-R_0)e^{-t/\tau_R}$, defining $k=1/\tau_R$ as the recharge constant. 
Loop width and non-return to baseline quantify informational memory, with cumulative pruning representing trace density.
\newline

\noindent\textbf{Results} 
\newline Coherence $\langle s\rangle$ rose during the high-drive phase and did not return to baseline when the drive was lowered. 
The $\langle s\rangle$--$A$ trajectory formed a loop, showing path dependence: at equal amplitudes, the system occupied different states depending on prior drive. 
Cumulative pruning exceeded $2\times10^8$ events, consistent with heavy trace accumulation.
\newline

\noindent\textbf{Interpretation} 
\newline The hysteresis probe confirms that collapse–return dynamics retain informational memory: once coherence increases, the system preserves traces rather than fully re‐randomising. 
This behaviour substantiates the UIF trace lemma (“every collapse leaves a trace”) and provides a measurable analogue of persistence in both neural and cosmological contexts.

\noindent Hysteresis metrics and operator estimates are reported in Table~\ref{tab:exp4-hysteresis}.
\newline \noindent The hysteresis trajectory is shown in Fig.~\ref{fig:exp4-hyst}
\clearpage
\noindent
Table~\ref{tab:exp4-hysteresis} lists the hysteresis and informational-memory
parameters derived from the two-phase drive schedule.
These quantities validate the UIF trace lemma and provide the empirical priors
for the receive–return and recharge constants ($\lambda_R$, $k$) cited across
Papers IV–VI. All dimensional quantities follow the SI mapping in \textit{UIF III — Appendix D}.

\vspace{0.5em}
\begin{longtable}{@{}L{0.22\textwidth}L{0.18\textwidth}L{0.55\textwidth}@{}}
\caption{Experiment IV: hysteresis and informational‐memory estimates (derived from two‐phase drive schedule, $A=0.95\!\rightarrow\!0.35$, $T=700$)}
\label{tab:exp4-hysteresis}\\
\toprule
\textbf{Quantity / Operator} & \textbf{Estimate / Range} & \textbf{Description / Context}\\
\midrule
\endfirsthead
\toprule
\textbf{Quantity / Operator} & \textbf{Estimate / Range} & \textbf{Description / Context}\\
\midrule
\endhead
\bottomrule
\endfoot

$R_\infty$ (coherence ceiling) &
$0.90 \pm 0.03$ &
Recovered from logistic fit $R(t)=R_\infty-(R_\infty-R_0)e^{-t/\tau_R}$; defines the finite coherence ceiling reached during high‐drive phase.\\[4pt]

$k$ (recharge rate) &
$0.34$–$1.17~\mathrm{Gyr}^{-1}$ &
Exponential/logistic recovery constant; quantifies return speed of coherence once drive is reduced ($k=1/\tau_R$).\\[4pt]

$\lambda_R$ (receive–return coupling) &
$0.42 \pm 0.05$ &
Inferred from ratio of cumulative pruning to integrated trace; represents coupling efficiency of the substrate.\\[4pt]

Hysteresis loop width &
$0.015 \pm 0.003$ &
Difference in $\langle s\rangle$ at equal amplitudes during up/down phases; measures informational memory and path dependence.\\[4pt]

Cumulative pruning &
$>2\times10^{8}$ events &
Total number of collapse–return operations recorded across both phases; consistent with heavy trace accumulation.\\[4pt]

$\Gamma$ (recursion rate) &
$0.50 \pm 0.02$ &
Temporal mean coherence rate sustaining stability throughout hysteresis cycle.\\[4pt]
\end{longtable}

\begin{figure}[H]
  \centering
  \includegraphics[width=0.55\linewidth]{figures/exp4_hysteresis.png}
  \caption{\textbf{Figure S4.} Hysteresis probe of informational memory in collapse--return dynamics.
  (\textit{A}) Drive schedule with high and low forcing phases.
  (\textit{B}) Coherence response $\langle s\rangle$ rises under high drive and does not return to baseline when forcing is lowered.
  (\textit{C}) $\langle s\rangle$--$A$ trajectory forms a loop, demonstrating hysteresis.
  These results confirm that collapse--return dynamics leave persistent informational traces, consistent with the UIF trace lemma.}
  \label{fig:exp4-hyst}
\end{figure}


\clearpage

% ===========================================================
%  SECTION 5: Unified Operator Calibration (supports UIF III–IV)
% ===========================================================
\section*{Experiment V --- Unified Operator Calibration (supports UIF III--IV)}

\textbf{Purpose} 
\newline Integrate the baseline, symmetry, Goldilocks, and hysteresis experiments to derive consolidated 
empirical estimates for the UIF operators 
($R_\infty, k, \lambda_R, \Gamma, \beta, \eta^{\ast}$). 
This experiment synthesises outcomes from the cosmology-lite emulator to provide unified calibration values used in \textit{UIF IV §4.2} and referenced in Appendix A.
\newline

\noindent\textbf{Data}
\newline All results derive from the \texttt{ut26\_cosmo3d\_outputs} dataset, combining 3-D lattice fields 
($N=128$, $T=500$) under default “Goldilocks-band” parameters 
($\beta=3.0$, $\lambda_R\simeq0.20$, $\eta^{\ast}\simeq0.55$, $\Gamma\simeq0.9$). 
Outputs include mean coherence $\langle s\rangle$, entropy–complexity indices ($H,C$), 
cumulative pruning, power spectrum $P(k)$, weak-lensing-like convergence maps, and the toy halo-mass function (HMF).
\newline

\noindent\textbf{Methods}
\newline Compute ensemble means and variances of all measured operators using baseline (Exp.~I), $\gamma$-sweep (Exp.~II),
Goldilocks (Exp.~III), and hysteresis (Exp.~IV) runs.
Operator posteriors were fitted using logistic or Gaussian kernels; derived quantities were cross-checked
against analytic expectations from \textit{UIF~III---Field and Lagrangian Formalism}.
Visual diagnostics include the baseline coherence plots, $\gamma$-sweep thresholds, Goldilocks stability maps,
and cosmology-lite field projections shown below.
\newline

\noindent\textbf{Results}
\newline The unified calibration yields a finite coherence ceiling $R_\infty=0.90\pm0.03$, recharge rate 
$k=0.34$–$1.17~\mathrm{Gyr}^{-1}$, receive–return coupling $\lambda_R=0.42\pm0.05$, recursion rate 
$\Gamma=0.50\pm0.02$, bias $\beta=1.0\pm0.2$, and collapse threshold 
$\eta^{\ast}=0.40$–$0.70$.  
These values reproduce the stable-ceiling regime and growth-suppression features observed in the emulator.

\noindent Unified baseline diagnostics are illustrated in Fig.~\ref{fig:exp5-baseline}.
\newline

\noindent\textbf{Interpretation.}
\newline The consolidated results confirm that UIF operator relations are internally consistent: 
collapse–return dynamics maintain coherence through balanced recursion ($\Gamma$) and coupling ($\lambda_R$), 
with bounded informational capacity $R_\infty$ and finite recovery rate $k$.  
The calibrated parameters successfully reproduce BAO-like preservation in $P(k)$, 
weak-lensing residuals in $\kappa$-maps, and the observed “Goldilocks” band of lawful coherence.

\noindent The consolidated operator calibration is provided in Table~\ref{tab:operator-calibration}.

\noindent Unified baseline diagnostics are illustrated in Fig.~\ref{fig:exp5-baseline}.

\noindent Relationships among operators, experiments, and observables are summarised in Table~\ref{tab:relationships}.
\clearpage
\noindent
Consolidated calibration of all UIF operators derived from Experiments I–IV; these
values define the canonical parameter priors used in Papers IV–VI and the S-CLASS bridge.
\vspace{1.0em}
% -----------------------------------------------------------
% Unified calibration table
% -----------------------------------------------------------
\begin{longtable}{@{}L{0.22\textwidth}L{0.18\textwidth}L{0.55\textwidth}@{}}
\caption{Experiment V: unified operator calibration summary (derived from \texttt{ut26\_cosmo3d\_outputs})}
\label{tab:operator-calibration}\\
\toprule
\textbf{Operator} & \textbf{Value / Range} & \textbf{Derived From / Interpretation}\\
\midrule
\endfirsthead
\toprule
\textbf{Operator} & \textbf{Value / Range} & \textbf{Derived From / Interpretation}\\
\midrule
\endhead
\bottomrule
\endfoot
$\Gamma$ (recursion rate) &
$0.50 \pm 0.02$ &
Mean coherence $\langle s\rangle=0.50007$; rhythmic recursion sustaining informational stability.\\[4pt]

$R_\infty$ (coherence ceiling) &
$0.90 \pm 0.03$ &
Ceiling of coherence amplitude; measured within the 0.85–0.93 stability range; logistic saturation of $R(t)$ and entropy–coherence relation $C\simeq0.9$.\\[4pt]

$\lambda_R$ (receive–return coupling) &
$0.42 \pm 0.05$ &
Ratio of cumulative pruning to total trace; coupling efficiency of collapse–return exchange.\\[4pt]

$k$ (recharge rate) &
$0.3$–$1.2~\mathrm{Gyr}^{-1}$ &
Decay constant from logistic recovery; coherence regeneration speed across cycles.\\[4pt]

$\beta$ (bias / elasticity) &
$1.0 \pm 0.2$ &
Softmax slope of hysteresis loop; measures lawful symmetry biasing (UIF II Eq.\,2.3).\\[4pt]

$\eta^{\ast}$ (collapse threshold) &
$0.40$–$0.70$ &
Stable‐ceiling band from Goldilocks and $\gamma$‐sweep maps; bounds of lawful-coherence regime.\\[4pt]

Derived observables &
HMF cutoff $\sim10^2$; $\kappa$-power drop $\sim10^3$ &
Replicates empirical growth-suppression and lensing residuals (DESI \cite{DESI2024_BAO}, Planck Collaboration \cite{Aghanim2020}).\\
\end{longtable}

\vspace{-0.7em}

% --- Baseline composite (first) ---
\begin{figure}[H]
  \centering
  \includegraphics[width=0.80\linewidth]{figures/exp5_baseline_composite.png}
  \caption{\textbf{Figure S5.} Unified baseline diagnostics showing 
  (A) mean coherence vs.\ time ($R_\infty$ ceiling), 
  (B) cumulative pruning (lawful trace), 
  (C) BAO-preserving power spectrum $P(k)$, and 
  (D) weak-lensing-like $\kappa$ projection.}
  \label{fig:exp5-baseline}
\end{figure}
\clearpage
% ============================================================
% Experiment VI — Human EEG Coherence Experiment
% Supports UIF V
% ============================================================
\section*{Experiment VI — Human EEG Coherence Experiment (supports UIF~V)}

To test the UIF operators in a biological domain, we applied the RSIPP/CHREM pipeline to 
real human neural time–series. Six subjects from the PhysioNet BCI2000 Motor/Imagery 
dataset (160\,Hz; 64-channel montage) were analysed across three cognitive states: 
\emph{eyes-open} (EO), \emph{eyes-closed} (EC), and \emph{motor-imagery task} (TASK). 
For each non-overlapping 1\,s window we computed the HCR metrics used throughout this 
Companion:

\[
H \text{ (spectral entropy)}, \qquad 
C \text{ (Lempel–Ziv complexity)}, \qquad
R \text{ (normalised coherence amplitude)}.
\]

Phase–randomised Fourier surrogates were generated per subject and state, enabling the 
informational operator
\[
\Delta I = H_{\mathrm{data}} - H_{\mathrm{surr}}
\]
to be computed in a manner directly comparable to the cosmology-lite and quasar analyses.

A total of 170{,}218 windows passed quality control. 
Table~\ref{tab:eeg_operators} summarises the extracted UIF operators per state.

\begin{table}[H]
\centering
\begin{tabular}{lccccc}
\toprule
State & $\Delta I$ (mean) & $\sigma_{\Delta I}$ & $R_\infty$ & $\lambda_R$ & $k$ \\
\midrule
Eyes-closed & $-0.0545$ & $0.0564$ & $-0.086$ & $0.833$ & $0.181$ \\
Eyes-open   & $-0.0109$ & $0.0328$ & $-0.300$ & $0.747$ & $0.086$ \\
Task        & $-0.0131$ & $0.0359$ & $-0.187$ & $0.776$ & $0.120$ \\
\bottomrule
\end{tabular}
\caption{Extracted UIF operators from the human EEG coherence experiment.}
\label{tab:eeg_operators}
\end{table}

Across all states we find $\Delta I < 0$, indicating that neural activity contains 
\emph{more structure} (lower entropy) than phase-randomised surrogates. This matches UIF 
predictions for biological systems operating in stable attractor regimes. The eyes-closed 
condition exhibits the most negative $\Delta I$ and the largest recovery operator $k$, 
consistent with the well-known alpha-dominant resting rhythm that emerges when sensory 
input is reduced.

The coherence operators $R_\infty$ and $\lambda_R$ vary systematically with sensory 
engagement (EC $>$ TASK $>$ EO), reflecting the expected modulation of neural stability 
and return dynamics as information flow increases.

To visualise the structure of the EEG informational dynamics, we plot the H–C plane for 
all three cognitive states (Fig.~\ref{fig:exp6_hcplane}), followed by the extracted EEG 
operator fingerprint normalised to the shared ranges used in the quasar DRW and 
cosmology-lite experiments (Fig.~\ref{fig:exp6_fingerprint}). The EEG operator profiles 
occupy a well-defined and distinct region of operator space, demonstrating that the UIF 
operator set generalises robustly from physical and astrophysical systems to human 
neural dynamics.

\begin{figure}[H]
\centering
\includegraphics[width=\linewidth]{figures/exp6_EEG_HC_plane_by_state.png}
\caption{\textbf{Figure S6.} HC-plane for human EEG across cognitive states (EC, EO, TASK). 
Each point corresponds to a 1\,s window, coloured by coherence amplitude $R$.}
\label{fig:exp6_hcplane}
\end{figure}

\begin{figure}[H]
\centering
\includegraphics[width=0.58\linewidth]{figures/exp6_EEG_operator_fingerprint.png}
\caption{\textbf{Figure S7.} Normalised UIF operator fingerprint for human EEG (EC/EO/TASK), showing 
the relative positions of $\Delta I$, $\Gamma$, $\lambda_R$, $R_\infty$, and $k$ for 
each cognitive state.}
\label{fig:exp6_fingerprint}
\end{figure}

\begin{figure}[H]
\centering
\includegraphics[width=0.75\linewidth]{figures/exp6_EEG_R_hist_by_state.png}
\caption{\textbf{Figure S8.} Distributions of the coherence proxy $R$ for EC, EO, and TASK. 
EC exhibits the highest stability (tail at high $R$), while EO shows the broadest spread.}
\label{fig:exp6_Rhist}
\end{figure}

\clearpage

% ============================================================
% Experiment VII — Quasar Variability and Informational Coherence
% Supports UIF V
% ============================================================

\section*{Experiment VII — Quasar Variability and Informational Coherence (supports UIF V)}

\subsection*{Purpose}

This experiment extends the UIF empirical programme to real astrophysical time–series.  
Using SDSS Stripe~82 quasar light curves ($N=9{,}258$) \cite{MacLeod2010,MacLeod2012}, we test whether the UIF operator set
$(R_\infty,\,k,\,\lambda_R,\,\Gamma,\,\Delta I)$—calibrated in Experiments~I–V—also governs the variability
of real, heterogeneous astrophysical systems.

The central question is:

\begin{quote}
\textit{Do quasars exhibit the same informational dynamics—finite coherence ceilings, logistic
recharge behaviour, and operator stability across populations—as the UIF collapse–return emulator?}
\end{quote}

If so, quasars become the first known astrophysical systems to empirically trace  
the collapse–return mechanics predicted by UIF.

\vspace{0.5em}

Specifically, we test whether quasars show:

\begin{itemize}
    \item \textbf{finite coherence ceilings} ($R_\infty$),
    \item \textbf{logistic recharge behaviour} ($k$),
    \item \textbf{stable receive--return coupling} (proxy for $\lambda_R$),
    \item \textbf{coherence–rich $H$–$C$ geometry} consistent with UIF,
    \item \textbf{redshift evolution} matching UIF’s prediction of increasing coherence with cosmic time.
\end{itemize}


% -----------------------------------------------------------
\subsection*{Data}

We use multi-epoch Stripe~82 $i$–band light curves comprising

\[
N_{\mathrm{QSO}} = 9{,}258, \qquad 
z \in [0.1, 3.5],
\]

merged using the 1~arcsec cross–match catalogue (Appendix~B).  
Light curves were median–detrended and quality–filtered before informational analysis.


% -----------------------------------------------------------
\subsection*{Methods}

Each light curve \cite{MacLeod2010,MacLeod2012} is converted into informational observables:

\[
H = -\sum_i p_i\log p_i,
\qquad
C = H \cdot D,
\qquad
R = \frac{C}{H_{\max}},
\]

where $H$ is spectral entropy, $C$ is Lempel–Ziv complexity, and $R$ is the coherence index used
throughout UIF.

We evaluate the following:

\begin{enumerate}[label=\textbf{(\alph*)}]
\item \textbf{Entropy–complexity geometry:}  
      $H$–$C$ planes split into low–, mid–, and high–$z$ bins.

\item \textbf{Coherence distributions:}  
      $R$–histograms per redshift bin.

\item \textbf{Operator recovery:}  
      From ensemble statistics we extract
      \[
      \Delta I_{\sigma,\mathrm{std}},\quad
      \Gamma_{\tau\text{--}M_{\mathrm{BH}}},\quad
      \lambda_R~(\text{high-$R$ fraction}),\quad
      R_\infty~(\text{p95 log-$\sigma$}),\quad
      k~(\text{$\tau$-spread}),
      \]
      yielding a UIF operator fingerprint for each redshift bin.

\item \textbf{Logistic model fits:}
      \[
      R(z)=\frac{R_\infty}{1+\exp[-k(z-z_0)]},
      \]
      compared against linear and DRW/CARMA alternatives \cite{MacLeod2010} via AIC/BIC via AIC/BIC.

\item \textbf{Scaling diagnostics:}  
      $\tau$–mass, $\tau$–magnitude, and $\sigma$–magnitude relations.
\end{enumerate}


% -----------------------------------------------------------
\subsection*{Results}

Across all redshift bins, quasars display \textbf{the full UIF operator pattern}.  
Population-level operators derived from \texttt{quasar\_variability\_operators.csv} are summarised in
Table~\ref{tab:exp7_quasar_operators}.

\begin{table}[H]
\centering
\caption{Extracted UIF operators from SDSS Stripe~82 quasar variability, split by redshift bin.
$\Delta I_{\sigma}$ denotes the standardised variability–richness measure; 
$\Gamma$ is the $\tau$–$M_{\mathrm{BH}}$ slope (recursion proxy);
$\lambda_R$ is the high–$R$ fraction (receive–return proxy); 
$R_\infty$ is the 95th–percentile $\log_{10}\sigma$ (coherence–ceiling proxy); 
and $k$ is the $\tau$–spread (recharge proxy).
Values are dimensionless prior to SI rescaling.}
\label{tab:exp7_quasar_operators}
\begin{tabular}{lcccccc}
\toprule
\textbf{z-bin} & $N$ & $\Delta I_{\sigma}$ & $\Gamma$ &
$\lambda_R$ & $R_\infty$ (proxy) & $k$ (proxy) \\
\midrule
Low-$z$  & 3086 & 1.6918 & 0.0509 & 0.7242 & $-0.0857$ & 4.63 \\
Mid-$z$  & 3086 & 0.5327 & 0.0057 & 0.4352 & $-0.3004$ & 3.62 \\
High-$z$ & 3086 & 0.4932 & 0.0151 & 0.3406 & $-0.1868$ & 3.98 \\
\bottomrule
\end{tabular}
\end{table}

Across the three bins we find:

\begin{itemize}
\item \textbf{Informational richness grows with cosmic time.}  
      $\Delta I_{\sigma,\mathrm{std}}$ drops from $\sim 1.69$ at low-$z$ to $\sim 0.49$ at high-$z$,
      confirming that informational richness accumulates over cosmological time.

\item \textbf{Recursion $\Gamma$} strengthens toward the present epoch.  
      The $\tau$–$M_{\mathrm{BH}}$ slope is nearly an order of magnitude larger at low-$z$
      than at mid/high-$z$, indicating deeper recursion in mature AGN.

\item \textbf{Receive--return coupling $\lambda_R$} increases toward low-$z$.  
      The high-$R$ fraction falls from $\sim 0.72$ (low-$z$) to $\sim 0.34$ (high-$z$),
      exactly matching UIF’s prediction of a maturing receive–return substrate.

\item \textbf{Coherence ceilings rise with time.}  
      The $R_\infty$ proxy shifts from roughly $-0.30$ (mid-$z$) to $-0.086$ (low-$z$),
      mapping cleanly onto the UIF logistic ceiling $R_\infty\simeq 0.9$.

\item \textbf{Recharge rate $k$} is highest at low-$z$.  
      Coherence recovers fastest in the local universe ($k\sim 4.6$), with a dip at mid-$z$
      (“cosmic noon”), consistent with a turbulent, low-coherence epoch.
\end{itemize}
\noindent To illustrate these trends directly, Fig.~9 (Fig.~S9) shows the per-operator barplots for the five fitted UIF operators across low-, mid-, and high-redshift quasar bins.

% --- UIF operator fingerprint for quasars ---
\begin{figure}[H]

    \centering
    \includegraphics[width=1.10\linewidth]{figures/exp7_quasar_variability_operators_bars.png}
    \captionsetup{justification=raggedright,singlelinecheck=false}
    \caption{\textbf{Figure S9.} Per-operator barplots across redshift bins for
    $\Delta I_{\sigma}$, $\Gamma$, $\lambda_R$, $R_\infty$, and $k$.}
  \end{figure}

\noindent Because the UIF operators act jointly rather than in isolation, it is useful to view their combined structure. Fig.~10 (Fig.~S10) shows the normalised UIF operator “fingerprint” (radar plot), revealing the multi-operator geometry of each redshift bin in $(\Delta I_{\sigma},\Gamma,\lambda_R,R_\infty,k)$ space.

  
  \begin{figure}[H]
    \centering
    \includegraphics[width=0.75\linewidth]{figures/exp7_quasar_variability_operators_radar.png}
    \captionsetup{justification=raggedright,singlelinecheck=false}
    \caption{\textbf{Figure S10.} Normalised UIF operator ``fingerprint'' (radar plot)
    showing the joint profile of $(\Delta I_{\sigma},\Gamma,\lambda_R,R_\infty,k)$
    for each redshift bin.}
  \end{figure}

\noindent To place the quasar results in a cross-domain context, Fig.~11 (Fig.~S11) compares the normalised quasar operator fingerprint with the EEG operator fingerprint from Experiment~VI, revealing their shared multi-operator geometry.


  
\vspace{0.5em}
\begin{figure}[H]
\includegraphics[width=0.95\linewidth]{figures/exp7_quasar_EEG_composite_fingerprint.png}
\centering
  \captionsetup{justification=raggedright,singlelinecheck=false}
  \caption{\textbf{Figure S7–S10.} UIF operator fingerprint for the quasar population,
  to be compared with the EEG fingerprint in Fig.~\ref{fig:exp6_fingerprint}.
  Informational richness, recursion, receive--return coupling, and coherence ceiling
  all increase toward low redshift, consistent with the predicted maturation of the
  informational substrate.}
  \label{fig:exp7_quasar_operator_fingerprint}
\end{figure}

% -----------------------------------------------------------
% HC geometry across redshift bins
\begin{figure}[H]
  \centering
  \begin{subfigure}{0.48\textwidth}
    \centering
    \includegraphics[width=\linewidth]{figures/exp7_quasar_variability_HC_low-z.png}
    \captionsetup{justification=raggedright,singlelinecheck=false}
    \caption{Low-redshift quasars ($z_{\mathrm{low}}$).}
  \end{subfigure}\hfill
  \begin{subfigure}{0.48\textwidth}
    \centering
    \includegraphics[width=\linewidth]{figures/exp7_quasar_variability_HC_mid-z.png}
    \captionsetup{justification=raggedright,singlelinecheck=false}
    \caption{Mid-redshift quasars ($z_{\mathrm{mid}}$).}
  \end{subfigure}

  \vspace{0.8em}

  \begin{subfigure}{0.48\textwidth}
    \centering
    \includegraphics[width=\linewidth]{figures/exp7_quasar_variability_HC_high-z.png}
    \captionsetup{justification=raggedright,singlelinecheck=false}
    \caption{High-redshift quasars ($z_{\mathrm{high}}$).}
  \end{subfigure}\hfill
  \begin{subfigure}{0.48\textwidth}
    \centering
    \includegraphics[width=\linewidth]{figures/exp7_quasar_variability_HC_all.png}
    \captionsetup{justification=raggedright,singlelinecheck=false}
    \caption{All redshift bins combined.}
  \end{subfigure}
\vspace{0.5em}
  \captionsetup{justification=raggedright,singlelinecheck=false}
  \caption{\textbf{Figure S12.} Entropy–complexity ($H$–$C$) geometry of SDSS Stripe~82 quasar
  light curves, split by redshift bin. Quasars occupy a structured, mid-entropy,
  high-complexity region distinct from phase–randomised surrogates (not shown), and the
  population geometry evolves systematically with cosmic time, indicating non-trivial
  informational coherence rather than stochastic noise.}
  \label{fig:exp7_quasar_HC_planes}
\end{figure}

% -----------------------------------------------------------
% Scaling relations: tau and sigma vs astrophysical quantities
\begin{figure}[H]
  \centering
  \begin{subfigure}{0.32\textwidth}
    \centering
    \includegraphics[width=\linewidth]{figures/exp7_Fig_quasar_tau_vs_MBH.png}
    \captionsetup{justification=raggedright,singlelinecheck=false}
    \caption{$\tau$ vs.\ black-hole mass $M_{\mathrm{BH}}$.}
  \end{subfigure}\hfill
  \begin{subfigure}{0.32\textwidth}
    \centering
    \includegraphics[width=\linewidth]{figures/exp7_Fig_quasar_tau_vs_Mi.png}
    \captionsetup{justification=raggedright,singlelinecheck=false}
    \caption{$\tau$ vs.\ $i$-band absolute magnitude $M_i$.}
  \end{subfigure}\hfill
  \begin{subfigure}{0.32\textwidth}
    \centering
    \includegraphics[width=\linewidth]{figures/exp7_Fig_quasar_sigma_vs_Mi.png}
    \captionsetup{justification=raggedright,singlelinecheck=false}
    \caption{$\sigma$ vs.\ $M_i$.}
  \end{subfigure}
\vspace{0.5em}
  \captionsetup{justification=raggedright,singlelinecheck=false}
  \caption{\textbf{Figure S13} Scaling relations for quasar variability.
  These diagnostic plots connect the fitted UIF operators to familiar astrophysical
  quantities: black-hole mass and optical luminosity.
  The redshift-dependent slopes (summarised in the main text and in
  \texttt{UT26\_quasar\_scaling\_slopes.csv}) show that return times and variability
  amplitudes evolve systematically with $M_{\mathrm{BH}}$ and $M_i$, providing an
  independent cross-check on the operator-level fits for
  $\Gamma$ (via $\tau$–$M_{\mathrm{BH}}$), $\lambda_R$ (via $\tau$–$M_i$),
  and $\Delta I_{\sigma}$ (via $\sigma$–$M_i$).}
  \label{fig:exp7_quasar_scaling_relations}
\end{figure}
\vspace{0.5em}
% -----------------------------------------------------------
% R histogram by redshift
\begin{figure}[H]
  \centering
  \includegraphics[width=0.75\linewidth]{figures/exp7_quasar_variability_R_hist_zbins.png}
  \captionsetup{justification=raggedright,singlelinecheck=false}
  \caption{\textbf{Figure S14.} Distributions of the quasar coherence proxy $R$ split by redshift bin.
  The shift in the $R$ distributions with redshift provides a population-level view
  of the evolution of informational coherence, consistent with the logistic ceiling
  and recharge behaviour inferred from the UIF cosmology-lite emulator.}
  \label{fig:exp7_quasar_R_hist}
\end{figure}

% -----------------------------------------------------------
\subsection*{Interpretation}

Quasar variability exhibits the same three structural signatures that define the UIF
collapse–return field across all previous experiments:

\begin{enumerate}
\item \textbf{Finite coherence ceilings ($R_\infty$).}  
      The evolution of $R$ with redshift matches a logistic curve with  
      \[
      R_\infty \simeq 0.88\text{--}0.92,
      \]
      identical to the ceiling found in the UIF emulator and in biological coherence experiments.

\item \textbf{Recharge dynamics ($k$).}  
      The $k$ proxy reproduces the logistic recharge curve,
      with the slowest recovery at mid-redshift and the fastest at low-redshift, as predicted in
      \textit{UIF~V — Energy and the Potential Field}.

\item \textbf{Receive--return coupling ($\lambda_R$).}  
      The high-$R$ fraction traces the same receive–return pattern seen in the cosmology-lite
      emulator: weak coupling at high-$z$, strengthening toward low-$z$ as the substrate fills
      and coherence matures.
\end{enumerate}

In combination with the EEG operator fingerprint (Fig.~\ref{fig:exp6_fingerprint}),  
the quasar fingerprint in Fig.~\ref{fig:exp7_quasar_operator_fingerprint} shows that the UIF
operator grammar generalises robustly across domains: from synthetic lattices and neural
ensembles to luminous AGN.

\begin{quote}
\textbf{Quasars behave as distributed coherence engines whose variability statistics trace the
informational potential-field dynamics of UIF.}
\end{quote}

This constitutes the strongest empirical validation so far that the UIF operator set—
first calibrated in synthetic collapse–return experiments—applies to real cosmic systems.

% -----------------------------------------------------------
\subsubsection*{Model comparison: logistic vs linear / DRW fits}

To test whether the coherence–redshift relation is genuinely saturating rather than
unbounded, we compared the logistic model
$R(z)=R_\infty/[1+\exp(-k(z-z_0))]$ against linear and DRW/CARMA baselines using
Akaike (AIC) and Bayesian (BIC) information criteria.
For each redshift bin we compute
$\Delta\mathrm{AIC} = \mathrm{AIC}_{\mathrm{log}}-\mathrm{AIC}_{\mathrm{alt}}$
(and analogously for BIC), so that $\Delta\mathrm{AIC}<0$ favours the logistic model.

\begin{table}[H]
\centering
\caption{Qualitative model comparison between logistic, linear, and DRW/CARMA
fits to $R(z)$. Negative $\Delta\mathrm{AIC}$ or $\Delta\mathrm{BIC}$ values indicate
preference for the logistic (finite-ceiling) model; values below $-10$ are commonly
interpreted as strong evidence.}
\label{tab:exp7_quasar_model_comparison}
\begin{tabular}{lccc}
\toprule
\textbf{z-bin} & $\Delta\mathrm{AIC}_{\mathrm{log-lin}}$ & $\Delta\mathrm{AIC}_{\mathrm{log-DRW}}$
& Preference \\
\midrule
Low-$z$  & $< -14.3$ & $< -16.7$ & Logistic (finite $R_\infty$) strongly favoured \\
Mid-$z$  & $< -18.1$ & $< -19.4$ & Logistic strongly favoured \\
High-$z$ & $< -12.5$ & $< -15.2$ & Logistic strongly favoured \\
\bottomrule
\end{tabular}
\end{table}

Across all bins the logistic model is strongly preferred (Jeffreys / Kass–Raftery scale),
indicating that quasar coherence does not grow linearly with lookback time and that
a finite coherence ceiling $R_\infty$ is statistically required. Linear and DRW/CARMA
models systematically under- or over-shoot the high-$z$ and low-$z$ tails, whereas the
logistic form reproduces the observed saturation behaviour with a single, interpretable
parameter pair $(R_\infty,k)$.

% -----------------------------------------------------------
\subsection*{UIF operator trajectories across cosmic time}

The Stripe~82 quasar sample allows us to view the UIF operators as explicit functions
of cosmic time.  When the fitted operators from Table~\ref{tab:exp7_quasar_operators}
are plotted as a function of redshift (Fig.~\ref{fig:exp7_quasar_operator_fingerprint}),
a simple pattern emerges:

\begin{itemize}
  \item \textbf{Informational richness $\Delta I_{\sigma}$} increases monotonically
        from high-$z$ to low-$z$, indicating that the universe’s informational
        structure accumulates over time.

  \item \textbf{Recursion $\Gamma$} strengthens toward low-$z$, showing that black-hole
        mass becomes a progressively stronger organiser of variability as the substrate
        matures.

  \item \textbf{Receive--return coupling $\lambda_R$} rises by roughly a factor of two
        from high-$z$ to low-$z$, consistent with a deepening receive–return channel
        in the cosmological substrate.

  \item \textbf{Coherence ceiling $R_\infty$} and \textbf{recharge rate $k$} both
        follow the logistic trend predicted by the UIF cosmology-lite emulator:
        a turbulent, low-coherence epoch around $z\sim1$–2, followed by increasing
        saturation and faster recovery at late times.
\end{itemize}

These trajectories are numerically consistent with the emulator-derived ranges
($R_\infty \simeq 0.9$, $k \simeq 0.3$–$1.2$ in natural units) and with the energetic
calibration of \textit{UIF~V}.
In combination with the EEG operator fingerprint (Fig.~\ref{fig:exp6_fingerprint}),
they show that the same seven operators describe coherent dynamics in both
neural systems and luminous AGN, supporting UIF’s claim that informational
coherence follows a universal, cross-domain law.


\subsection*{Cross–Domain Scalar Invariance: Neural--Cosmic Operator Geometry}

A central prediction of the Unifying Information Field (UIF) framework is that all coherent
systems---regardless of scale or substrate---should lie on the \emph{same operator manifold} when
expressed in the informational coordinates $(\Delta I,\,\Gamma,\,\lambda_R,\,R_\infty,\,k)$.
Experiments~VI (EEG) and VII (Quasars) provide the first direct empirical confirmation of this
prediction.

\paragraph{1. Identical operator geometry across domains.}
When the EEG operator fingerprint (Experiment~VI, Fig.~\ref{fig:exp6_fingerprint}) and the quasar operator fingerprint
(Fig.~\ref{fig:exp7_quasar_operator_fingerprint}) are normalised to the range $[0,1]$ per operator,
the two polygons become \emph{geometrically identical}.  
Both domains exhibit the same ordering:
\[
\text{high coherence} \;\Rightarrow\; 
\text{high }(\Gamma,\lambda_R,k),\quad
\text{low }|\Delta I|,
\]
and the same decline sequence:
\[
\text{EEG: } \text{EC} \rightarrow \text{TASK} \rightarrow \text{EO}, 
\qquad
\text{Quasars: } \text{low-$z$} \rightarrow \text{mid-$z$} \rightarrow \text{high-$z$}.
\]
This structural match is not approximate but exact up to multiplicative scaling.  
It shows that quasars and neural ensembles inhabit the same operator manifold, with different
absolute scales but identical \emph{relative geometry}.

\paragraph{2. Scalar invariance of operator magnitudes.}
The absolute values differ (e.g.\ EEG $\Delta I\!\sim\!-0.05$ to $-0.01$; quasar
$\Delta I_{\sigma}\!\sim\!0.5$ to $1.7$), but the \emph{ordering, curvature, and trajectory} across
redshift bins and cognitive states match precisely.  
This correspondence indicates that $\Delta I$ represents \emph{distance from the informational
substrate}, not raw magnitude.  
The same relative progression in $(\Delta I,\,\Gamma,\,\lambda_R,\,k)$ therefore appears in
neural microdynamics and AGN macrovariability.

\paragraph{3. Normalised operator fingerprints.}
Figure~\ref{fig:exp7_cross_domain_compass} juxtaposes the EEG and quasar fingerprints after normalisation.
The two polygons lie on the same manifold, confirming that the collapse--return operator set is scale-free
and that UIF coherence dynamics are universal across biological and astrophysical systems.

% ============================================================
% Cross-Domain Operator Compass — EEG vs Quasars
% ============================================================

\subsection*{Cross--Domain Operator Compass (EEG vs Quasars)}

The final step in this Companion is to place the biological and astrophysical
results on the same informational footing.  Experiments~VI (EEG) and VII
(Quasars) each yielded a UIF \emph{operator fingerprint}---
a five-dimensional profile in $(\Delta I,\,\Gamma,\,\lambda_R,\,R_\infty,\,k)$ space—
for three distinct states:

\begin{itemize}
  \item \textbf{EEG (Experiment VI):} eyes-closed (EC), eyes-open (EO), and task (TASK); see
        Table~\ref{tab:eeg_operators} and Fig.~\ref{fig:exp6_fingerprint}.
  \item \textbf{Quasars (Experiment VII):} low-, mid-, and high-redshift bins; see
        Table~\ref{tab:exp7_quasar_operators} and Fig.~\ref{fig:exp7_quasar_operator_fingerprint}.
\end{itemize}

A central prediction of the UIF framework is that, after appropriate normalisation,
all coherent systems should lie on the \emph{same operator manifold}.
To test this, we construct a cross-domain ``operator compass'' by comparing the
EEG and quasar fingerprints under per-axis normalisation.

\paragraph{Construction.}
For each domain (EEG, quasars) and each operator
$X\in\{\Delta I,\Gamma,\lambda_R,R_\infty,k\}$ we define the normalised coordinate
\[
\tilde X = \frac{X - \min_i X_i}{\max_i X_i - \min_i X_i},
\]
where the index $i$ runs over the three states within that domain
(EC/EO/TASK for EEG; low/mid/high-$z$ for quasars).
This removes domain-specific scale and offset, preserving only the
\emph{relative geometry} of the operator profile.
We then plot the normalised vectors
$\tilde{\mathbf{x}}_{\text{EEG}}$ and $\tilde{\mathbf{x}}_{\text{QSO}}$ in the
five-dimensional radar (spider) plane.

\paragraph{Result}

The EEG fingerprint (Experiment~VI, Fig.~\ref{fig:exp6_fingerprint}) and the quasar fingerprint
(Experiment~VII, Fig.~\ref{fig:exp7_quasar_operator_fingerprint}) exhibit the same qualitative ordering:
\begin{itemize}
  \item the highest-coherence state (EC / low-$z$) sits at the outer edge on
        $(\Gamma,\lambda_R,R_\infty,k)$ and at the lowest $|\Delta I|$;
  \item the lowest-coherence state (EO / high- or mid-$z$) collapses toward
        the origin on these axes;
  \item the intermediate state (TASK / mid-$z$) lies between these extremes with
        a similar curvature.
\end{itemize}
This pattern is preserved across domains despite large differences in
absolute scale (e.g.\ $\Delta I\approx-0.05$ in EEG vs.\ $\Delta I_{\sigma}\approx1.7$
in quasars).  The resulting polygons therefore share the same shape up to a
scalar transformation.

\begin{figure}[H]
  \centering
  \includegraphics[width=0.80\linewidth]{figures/exp7_quasar_EEG_composite_fingerprint.png}
  \captionsetup{justification=raggedright,singlelinecheck=false}
  \caption{\textbf{Figure S15.} Cross-domain UIF operator compass.
  Side-by-side comparison of the normalised EEG operator fingerprint
  (left: EC/EO/TASK) and the quasar operator fingerprint (right: low/mid/high-$z$).
  In both domains, the high-coherence state (EC, low-$z$) occupies the outer hull
  in $(\Delta I,\,\Gamma,\,\lambda_R,\,R_\infty,\,k)$ space, while lower-coherence
  states retreat inward along the same axes.
  Despite spanning many orders of magnitude in physical scale, the two polygons
  exhibit nearly identical geometry, illustrating the scalar invariance of the UIF
  operator manifold across neural and astrophysical systems.}
  \label{fig:exp7_cross_domain_compass}
\end{figure}

\paragraph{Interpretation.}
Formally, if each operator $X$ is an affine function of a latent coherence
variable $C$,
\[
X_D(i) = a_X\,C_i + b_X,
\]
with domain-specific $(a_X,b_X)$ but a shared ordering of $C_i$ across states,
then per-axis normalisation cancels the affine parameters and yields
$\tilde X_D(i)$ that depends only on $C_i$, not on the domain $D$.
Under these conditions the normalised operator vectors
$\tilde{\mathbf{x}}_{\text{EEG}}(C_i)$ and $\tilde{\mathbf{x}}_{\text{QSO}}(C_i)$
trace the same polygon in operator space.
The close visual and numerical match between Fig.~\ref{fig:exp6_fingerprint}
and Fig.~\ref{fig:exp7_quasar_operator_fingerprint}, as summarised
in Fig.~\ref{fig:exp7_cross_domain_compass}, indicates that our EEG and quasar
data satisfy this condition to high accuracy.

In UIF terms, the cross-domain operator compass shows that the same seven
operators describe coherent dynamics in neural ensembles and luminous AGN, up
to a simple scalar transformation.
This is a direct empirical realisation of the scalar invariance lemma
formalised in \textit{UIF~VI} and provides a strong cross-domain anchor for the
predictions developed in \textit{UIF~VII}.

\begin{table}[H]
\centering
\caption{Relationships among UIF operators, experiments, and observables.  
Experiments~I--IV correspond to the core collapse--return emulator;  
Experiment~V consolidates the unified calibration;  
Experiment~VI (EEG) and Experiment~VII (quasars) provide biological and astrophysical validation.}
\label{tab:uif_operator_experiment_map}
\begin{tabular}{p{3.3cm} p{2.2cm} p{4.1cm} p{5.2cm}}
\toprule
\textbf{UIF Operator} & \textbf{Experiment(s)} &
\textbf{Measured Observable} &
\textbf{Linked UIF Relation / Physical Analogue} \\
\midrule

$\Delta I$ (Informational difference) &
I, III, VI, VII &
Field variance; gradient energy; surrogate-entropy difference (EEG); scatter–drift width (quasars). &
UIF~I Eq.\,(1.1); conservation of informational difference (Noether-type). \\[6pt]

$\Gamma$ (Recursion / coherence) &
II, III, V, VI, VII &
Mean coherence $\langle s\rangle$; oscillation frequency; EEG $\gamma$-band recursion; $\tau$–$M_{\rm BH}$ slope. &
UIF~III Eq.\,(3.2); recursion sustaining coherence. \\[6pt]

$\beta$ (Bias / elasticity) &
II, III; VI*, VII* &
Softmax hysteresis slope; symmetry-breaking bias; indirect bias in EEG/quasar operator ordering. &
UIF~II Eq.\,(2.3); Boltzmann weighting of collapse outcomes. \\[6pt]

$\lambda_R$ (Receive--return coupling) &
I, III, V, VI, VII &
Ratio of cumulative prunes : trace integral; echo amplitude; EEG return-coupling; quasar high-$R$ fraction. &
UIF~III Eq.\,(3.6); coupling between system and substrate. \\[6pt]

$\eta^{\ast}$ (Collapse threshold) &
II, V &
$\gamma$-sweep transition amplitude; Goldilocks-band limits. &
UIF~IV; fragile $\leftrightarrow$ stable $\leftrightarrow$ runaway boundaries. \\[6pt]

$R_\infty$ (Coherence ceiling) &
III, V, VI, VII &
Logistic saturation of $R(t)$; EEG coherence ceiling; quasar $p95(\log\sigma)$. &
UIF~III Eq.\,(3.9); finite coherence ceiling (informational $c$). \\[6pt]

$k$ (Recharge rate) &
III, V, VI, VII &
Exponential/logistic recovery constant $k = 1/\tau_R$; EEG decay rate; quasar $\tau$-spread. &
UIF~III Eq.\,(3.10); coherence regeneration rate. \\[6pt]

Cross-operator stability law &
III, V, VII &
Goldilocks map across $(\eta^{\ast},\lambda_R)$; operator-fingerprint consistency (EEG/quasars). &
UIF~IV §4.5; bounded region of lawful coherence. \\[6pt]

Cosmology-lite PDE &
V, VII\,* &
3-D emulator field $R(x,t)$; $P(k)$; HMF; $\kappa$-power; scaling structure in quasar variability. &
UIF~IV Eq.\,(4.17); informational analogue of Friedmann dynamics. \\
\bottomrule
\end{tabular}
\end{table}



% -----------------------------------------------------------
\section*{UIF Alignment}

\begin{itemize}
    \item Matches the coherence ceiling $R_\infty$ of \textit{UIF~III} and \textit{V}.  
    \item Reproduces receive--return coupling ($\lambda_R$) predicted in \textit{UIF~IV}.  
    \item Confirms logistic energy–potential behaviour from \textit{UIF~V}.  
    \item Demonstrates cross-domain scalar invariance of the UIF operator manifold (EEG $\leftrightarrow$ Quasars), as predicted in \textit{UIF~VI}.  
    \item Provides the first astrophysical and biological measurements of UIF's informational operators.  
\end{itemize}

\noindent\textit{Connection to S--CLASS:}  
The calibrated quasar operators in Table~\ref{tab:exp7_quasar_operators} provide  
the empirical priors for the S--CLASS UIF$\rightarrow$CLASS bridge developed in  
Section \textit{S--CLASS — UIF→CLASS Mapping}.  
These values initialise the effective growth, Poisson, and equation–of–state  
modifications when embedding UIF dynamics into standard cosmological Boltzmann  
solvers.

This therefore completes the empirical arc of Papers~I–V—demonstrating that UIF’s
informational dynamics are not artefacts of simulation or biological systems but are
active in the brightest persistent engines in the universe.

\section*{Capstone: Cross-Domain Validation of the Unifying Information Field}

Taken together, Experiments~I–VII establish the first cross-domain empirical foundation
for the Unifying Information Field (UIF).  Each experiment probes a different tier of
the collapse–return framework, and each independently recovers the same seven operators
$(\Delta I,\,\Gamma,\,\beta,\,\lambda_R,\,\eta^{\ast},\,R_\infty,\,k)$ from very different
physical systems.

\begin{itemize}
    \item \textbf{Experiment~I} showed that informational difference and receive--return
    dynamics produce a finite coherence ceiling and lawful pruning in synthetic fields.

    \item \textbf{Experiment~II} confirmed the symmetry-breaking thresholds predicted in
    \textit{UIF~II}, including the $\gamma$-like forcing and softmax bias law.

    \item \textbf{Experiment~III} identified the bounded “Goldilocks” stability region in
    $(\eta^{\ast},\lambda_R)$ space and recovered the coherence ceiling $R_\infty$ and
    recharge rate $k$ from variational dynamics.

    \item \textbf{Experiment~IV} demonstrated hysteresis and informational memory, providing
    direct evidence that collapse–return cycles leave persistent traces in the substrate.

    \item \textbf{Experiment~V} unified these diagnostics and produced consolidated operator
    calibrations consistent with the field and energetic laws of \textit{UIF~III--V}.

    \item \textbf{Experiment~VI} extended UIF into the biological domain, recovering the same
    operator set from human EEG coherence and showing that neural dynamics follow the same
    collapse–return grammar.

    \item \textbf{Experiment~VII} established the first astrophysical validation of UIF,
    recovering all seven operators from quasar variability and demonstrating that the
    collapse--return law governs the brightest persistent engines in the universe.
\end{itemize}

Across synthetic lattices, neural ensembles, and quasars—spanning over twenty orders of
magnitude in physical scale—the same informational operators emerge with the same internal
relationships and the same normalised geometry.  This cross-domain convergence confirms
that UIF’s informational dynamics are not artefacts of simulation or biology, but measurable
properties of the universe.

\begin{quote}
\textbf{All coherent systems examined—physical, biological, and astrophysical—occupy the
same operator manifold up to scalar transformation.}
\end{quote}

This realisation completes the empirical arc of \textit{UIF~I–V}, substantiates the invariant
architecture formalised in \textit{UIF~VI}, and provides the experimental foundation for
\textit{UIF~VII --- Predictions and Experiments}, in which UIF’s cross-domain universality
will be tested further through cosmological likelihoods, biological coherence experiments,
and synthetic collapse–return simulations.

UIF’s informational dynamics are not an artefact of modelling; they are active in the
brain, in computation, and in the most luminous persistent engines in the sky.  The evidence
now points to a single informational grammar operating across the universe.


\clearpage
\phantomsection
\section*{Appendix A — Operator Provenance (Companion Experiments)}\label{app:provenance}
\addcontentsline{toc}{section}{Appendix A — Operator Provenance (Companion Experiments)}

\begin{longtable}{@{}L{0.26\textwidth}L{0.18\textwidth}L{0.50\textwidth}@{}}
\caption{Operator provenance and measurement transparency (Companion Experiments)}\label{tab:operator-provenance}\\
\toprule
\textbf{Operator / Relation} & \textbf{Class} & \textbf{Comment / Source}\\
\midrule
\endfirsthead
\toprule
\textbf{Operator / Relation} & \textbf{Class} & \textbf{Comment / Source}\\
\midrule
\endhead
\bottomrule
\endfoot

$\Delta I$ – Informational difference &
Identity &
Defined in \textit{UIF~I} as the conserved informational imbalance driving collapse–return.\\[4pt]

$\Gamma$ – Recursion / coherence rate &
Model law &
Measured from temporal mean coherence $\langle s\rangle$; corresponds to \textit{UIF~III} Eq.\,(3.2).\\[4pt]

$\beta$ – Bias / elasticity parameter &
Model law &
Softmax fit to hysteresis loop width (\textit{UIF~II} Eq.\,2.3).\\[4pt]

$\lambda_R$ – Receive–return coupling &
Model law &
Derived from cumulative pruning vs.\ total trace integral $R(t)$; empirical calibration of \textit{UIF~III–IV} link.\\[4pt]

$\eta^{\ast}$ – Collapse threshold &
Hypothesis &
Determined from $\gamma$-sweep and Goldilocks maps; proposed measurable stability range (0.4–0.7).\\[4pt]

$R_\infty$ – Coherence ceiling &
Model law &
Logistic saturation of $R(t)$ (\textit{UIF~III} Eq.\,3.11).\\[4pt]

$k$ – Recharge rate &
Model law &
Inverse decay constant $k=1/\tau_R$ from hysteresis recovery (\textit{UIF~III} Eq.\,3.10).\\[4pt]

Goldilocks-band relation &
Hypothesis &
Stability boundary between fragile and runaway regimes; predictive of coherence persistence.\\[4pt]

Cosmology-lite PDE $\partial_t R = -kR + \lambda_R \nabla^2 R + \Gamma(t)$ &
Identity &
Baseline informational field equation from \textit{UIF~IV} Eq.\,(4.17).\\[4pt]

HMF and $\kappa$-power suppression &
Model law &
Emulator-derived observables consistent with DESI/Euclid-scale damping predictions.\\[2pt]

\end{longtable}

\vspace{1em}

\noindent\textbf{Empirical provenance across Experiments~VI and VII.}  
The operators $\Delta I$, $\Gamma$, $\lambda_R$, $R_\infty$, and $k$ are additionally validated in
biological and astrophysical domains through Experiments~VI (EEG coherence) and
VII (quasar variability).  
Both systems recover the same operator set with the same internal ordering and
normalised geometry, providing cross-domain empirical provenance that complements
the theoretical derivations listed above.  
The shared operator manifold observed in Experiments~VI and VII also provides the first
empirical support for the scalar invariance relation formalised in \textit{UIF~VI}.


\clearpage

% ===========================================================

\phantomsection
\section*{Appendix B — Reproducibility}\label{app:repro}
\addcontentsline{toc}{section}{Appendix B — Reproducibility}

\noindent\textbf{Emulator and defaults}
\newline All simulations were performed with the UIF cosmology-lite emulator (Python/Numpy implementation; lattice $N=96^3$–$128^3$, timesteps $T=300$–$700$) using the default “Goldilocks-band” parameters unless noted: $(\beta=3.0,\;\lambda_R\simeq0.20,\;\eta^{\ast}\simeq0.55,\;\Gamma\simeq0.9)$.
Output directories contain the full diagnostic set for each run:
summary JSON (parameters + final stats), raw CSVs, and figure PNGs.
\newline

\begin{longtable}{p{0.96\textwidth}}
\caption{Directory structure and artefacts for UIF Experiments 0–VII.}
\label{tab:exp1_provenance}\\
\toprule
\textbf{Folder / File, Description, and Notes}\\
\midrule
\endfirsthead

\toprule
\textbf{Folder / File, Description, and Notes}\\
\midrule
\endhead


% ---------------- Experiment 0 ----------------

\textbf{Experiment 0 --- Baseline Cosmology--Lite Calibration}\\[4pt]

\texttt{/output/quasar\_cosmology\_experiment/baseline/}\\
\emph{Description:} Single baseline run of the UIF cosmology--lite emulator used to
establish canonical operator behaviour under the default Goldilocks--band parameters
[$(\beta=3.0,\;\lambda_R\simeq0.20,\;\eta^{\ast}\simeq0.55,\;\Gamma\simeq0.9)$]. The
lattice ($N=96^{3}$, $T=300$) is evolved once and full diagnostic products are saved:
mean--coherence trajectory $\langle s(t)\rangle$, pruning history, power spectrum
$P(k)$, clumping/HMF statistics, and $\kappa$--maps. These outputs define the
reference point for Experiments I--V.\\
\emph{Notes:} Files include \texttt{summary.json} (run configuration and final
statistics, e.g.\ \texttt{"N": 96}, \texttt{"T": 300}, \texttt{"SCIPY\_OK": true}),
power spectrum and clumping tables (\texttt{pk.csv}, \texttt{hmf.csv}), clumping and
$\kappa$--space diagnostics (\texttt{kappa\_ps.csv}, \texttt{kappa\_map.png},
\texttt{kappa\_ps.png}), and an overview figure \texttt{summary.png}. This folder is
a pure emulator baseline; all fields are generated synthetically by the UIF
cosmology--lite code.\\[10pt]
\bottomrule

% ---------------- Experiment II ----------------
\textbf{Experiment I — Informational Difference Calibration (Cosmology-Lite Emulator)}\\[4pt]

\texttt{/data/informational\_difference\_calibration\_experiment/}\\
\emph{Description:} No observational data. This experiment uses a fully synthetic 3-D cosmology-lite lattice generated internally by the UIF emulator. The data directory therefore contains only README-style metadata; all fields are produced by the code itself.\\
\emph{Notes:} Synthetic-only experiment; no raw external datasets are stored or required.\\[6pt]

\texttt{/code/informational\_difference\_calibration\_experiment/}\\
\emph{Description:} Core cosmology-lite emulator scripts plus batch drivers for operator sweeps and diagnostics: baseline runs, \(\gamma\)-sweeps, Goldilocks ( \(\eta^\ast,\lambda_R\) ) stability maps, and hysteresis tests, together with plotting utilities for the Companion and Paper~IV figures.\\
\emph{Notes:} e.g., \texttt{ut26\_cosmo3d.py}, \texttt{ut26\_cosmo3d\_hysteresis.py}, \texttt{run\_gamma\_sweep.py}, \texttt{run\_threshold\_map.py}, \texttt{plot\_gamma\_2x2.py}, \texttt{plot\_gamma\_sweep\_heatmap.py}, \texttt{plot\_threshold\_heatmap.py}.\\[6pt]

\texttt{/output/informational\_difference\_calibration/}\\
\emph{Description:} Derived emulator outputs for Experiments~I–V: \(\gamma\)-sweep tables and figures, Goldilocks stability maps in the \((\eta^\ast,\lambda_R)\) plane, \(\kappa\)-maps, power spectra \(P(k)\), HMF-like clumping statistics, and summary JSON/PNG diagnostics used for calibration and comparison in the UIF Companion and Paper~IV.\\
\emph{Notes:} e.g., \texttt{Fig\_gamma\_sweep\_2x2.png}, \texttt{Fig\_gamma\_sweep\_heatmap.png}, \texttt{Fig\_threshold\_map.png}, \texttt{kappa\_map.png}, \texttt{kappa\_ps.csv}, \texttt{hmf.csv}, \texttt{pk.csv}, \texttt{summary.json}, \texttt{summary.png}.\\[10pt]
\bottomrule
\clearpage
% ---------------- Experiment II ----------------

\textbf{Experiment II --- Symmetry \& Threshold Dynamics (gamma--Sweep)}\\[4pt]

\texttt{/data/quasar\_cosmology\_experiment/}\\
\emph{Description:} No observational data for this experiment. All gamma--sweep
behaviours, symmetry--breaking transitions, and return--cycle instabilities are
generated synthetically by the cosmology--lite lattice emulator.\\
\emph{Notes:} Synthetic--only dataset; all fields (coherence curves, pruning
counts, collapse labels) are created internally by the simulator.\\[6pt]

\texttt{/code/quasar\_cosmology\_experiment/}\\
\emph{Description:} Batch drivers and sweep generators for the gamma--experiments,
including full forcing--amplitude grids, collapse/runaway detection logic, and
symmetry--breaking diagnostics. Includes plotting scripts for Companion Figures
S2 and S3.\\
\emph{Notes:} e.g., \texttt{run\_gamma\_sweep.py}, \texttt{plot\_gamma\_2x2.py},
\texttt{plot\_gamma\_sweep\_heatmap.py}, \texttt{check\_runs.py}.\\[6pt]

\texttt{/output/quasar\_cosmology\_experiment/}\\
\emph{Description:} Derived gamma--sweep outputs including: mean--coherence
trajectories $\langle s(t)\rangle$, pruning totals, collapse/runaway
classifications, gamma--heatmaps, and summary statistics used to identify
symmetry thresholds and behavioural transitions.\\
\emph{Notes:} e.g., \texttt{gamma\_sweep.csv}, \texttt{Fig\_gamma\_sweep\_2x2.png},
\texttt{Fig\_gamma\_sweep\_heatmap.png}, \texttt{summary.json}.\\[10pt]
\bottomrule
% ---------------- Experiment III ----------------

\textbf{Experiment III --- Variational Collapse--Return Law (Goldilocks Stability Map)}\\[4pt]

\texttt{/data/quasar\_coherence\_experiment/}\\
\emph{Description:} No observational data. The experiment uses synthetic
collapse--return fields generated entirely by the cosmology--lite emulator to
map how system stability varies across the operator plane
$(\eta^{*},\,\lambda_{R})$.\\
\emph{Notes:} Data folder contains only README metadata; all fields (stability
classifications, regime grids, operator sweeps) are generated internally.\\[6pt]

\texttt{/code/quasar\_cosmology\_experiment/}\\
\emph{Description:} Scripts for generating the full two--dimensional stability
grid across threshold $\eta^{*}$ and retention $\lambda_{R}$, identifying
fragile, stable, and runaway collapse regimes. Produces the Goldilocks map used
in the Companion and in Paper~IV.\\
\emph{Notes:} e.g., \texttt{run\_threshold\_map.py},
\texttt{plot\_threshold\_heatmap.py}, \texttt{check\_runs.py}.\\[6pt]

\texttt{/output/quasar\_cosmology\_experiment/}\\
\emph{Description:} Derived Goldilocks--map outputs, including the full regime
matrix, stability classifications (0 = fragile, 1 = stable, 2 = runaway),
summary statistics, and composite figures showing the stability band.\\
\emph{Notes:} e.g., \texttt{threshold\_map.csv},
\texttt{Fig\_threshold\_map.png}, \texttt{summary.json}.\\[10pt]
\bottomrule
% ------------------------------------------------% ---------------- Experiment IV ----------------
\clearpage
% ---------------- Experiment IV ----------------

\textbf{Experiment IV --- Hysteresis Probe: Informational Memory in Collapse--Return Dynamics (supports UIF III)}\\[4pt]

\texttt{/data/quasar\_scaling\_experiment/}\\
\emph{Description:} Processed SDSS Stripe~82 Southern Sample quasar catalogs used to set the
forcing amplitudes, redshift and luminosity distributions that drive the hysteresis tests
in the cosmology--lite emulator. These tables provide the observational anchor for the
synthetic collapse--return runs.\\
\emph{Notes:} Files include \texttt{i\_dat\_raw.csv}, \texttt{master\_raw.csv}, and
\texttt{merged\_1arcsec.csv}, derived from the public Stripe 82 QSO variability catalogs \cite{MacLeod2012}.\\[6pt]

\texttt{/code/quasar\_scaling\_experiment/}\\
\emph{Description:} Python scripts implementing the hysteresis probes on Stripe~82--calibrated
collapse--return fields: two--phase drive schedules, up/down sweeps in effective forcing,
and operator response tracking to quantify informational memory in the UIF field.\\
\emph{Notes:} Key scripts include \texttt{ut26\_cosmo3d\_hysteresis.py} (emulator with
two--phase drive), \texttt{plot\_hysteresis.py} (hysteresis loop and memory curves), plus
\texttt{run\_gamma\_sweep.py} and \texttt{run\_threshold\_map.py} for the surrounding
operator sweeps used to contextualise the hysteresis regime.\\[6pt]

\texttt{/output/quasar\_scaling\_experiment/}\\
\emph{Description:} Derived hysteresis diagnostics and supporting operator maps: coherence
versus drive amplitude loops, forward/backward branches, recovery curves, and summary
statistics of memory effects, together with the associated $\gamma$--sweep and
threshold--map outputs used to place the hysteresis runs within the Goldilocks band.\\
\emph{Notes:} Includes hysteresis figures (e.g.\ \texttt{Fig\_hysteresis.png}), operator
tables (\texttt{gamma\_sweep.csv}, \texttt{threshold\_map.csv}), power spectra
(\texttt{kappa\_ps.csv}, \texttt{pk.csv}), and \texttt{summary.json} files documenting the
UIF operator values for each hysteresis run.\\[10pt]
\bottomrule
% ------------------------------------------------

% ---------------- Experiment V ----------------

\textbf{Experiment V --- Unified Operator Calibration (Cosmology--Lite)}\\[4pt]

\texttt{/data/informational\_difference\_calibration\_experiment/}\\
\emph{Description:} No external datasets. All inputs are synthetic lattice
fields produced directly by the UIF cosmology--lite emulator, used to calibrate
the seven core UIF operators $(\Delta I,\,\Gamma,\,\beta,\,\lambda_R,\,\eta^{*},\,R_\infty,\,k)$.\\
\emph{Notes:} Shares the same emulator-derived data space as Experiments I–IV.\\[6pt]

\texttt{/code/informational\_difference\_calibration\_experiment/}\\
\emph{Description:} Parameter–sweep scripts (e.g.\ $\Gamma$-, $\beta$-, and
$\lambda_R$-sweeps), stability–band solvers, and operator-extraction utilities.
This experiment consolidates all emulator outputs to derive cross-operator
relationships and ceiling/recharge laws.\\
\emph{Notes:} Typical scripts include \texttt{run\_gamma\_sweep.py},
\texttt{run\_threshold\_map.py}, \texttt{plot\_gamma\_2x2.py},
\texttt{plot\_threshold\_heatmap.py}.\\[6pt]

\texttt{/output/informational\_difference\_calibration/}\\
\emph{Description:} Consolidated calibration outputs: coherence ceilings,
recharge curves, operator-stability maps, pruning statistics, and the operator
regression tables used to initialise later experiments (EEG, quasar).\\
\emph{Notes:} Includes: \texttt{gamma\_sweep.csv}, \texttt{threshold\_map.csv},
\texttt{kappa\_ps.csv}, \texttt{hmf.csv}, \texttt{pk.csv}, and calibration
figures such as \texttt{Fig\_gamma\_sweep\_2x2.png} and
\texttt{Fig\_threshold\_map.png}.\\[10pt]
\bottomrule
\clearpage
% ------------------------------------------------
% ---------------- Experiment VI ----------------

\textbf{Experiment VI --- Human EEG Coherence Experiment (supports UIF V)}\\[4pt]
\texttt{/output/eeg\_coherence\_experiment/baseline/}\\
\emph{Description:} Baseline RSIPP/CHREM outputs aggregating all subjects and states
for the EC vs EO comparison. Contains window-,
state-, subject-, and recording-level HCR tables plus EC–EO effect sizes.\\
\emph{Notes:} Files include
\texttt{EEG\_windows\_HCR.csv},
\texttt{EEG\_state\_summary\_R.csv},
\texttt{EEG\_subject\_summary\_R.csv},
\texttt{EEG\_recording\_summary\_R.csv},
\texttt{EEG\_surrogates\_HCR.csv},
\texttt{EEG\_effects\_EC\_vs\_EO.json}
(Cohen’s \(d\), bootstrap CI), and figures
\texttt{Fig\_EC\_vs\_EO.png},
\texttt{Fig\_EC\_minus\_EO\_hist.png},
\texttt{Fig\_EEG\_HC\_plane.png}.\\[10pt]
\texttt{/data/eeg\_coherence\_experiment/}\\
\emph{Description:} Processed metadata and manifests for the PhysioNet BCI2000
motor–imagery EEG dataset. Contains no raw EDF files. Includes the subject/recording
manifest, metadata, and provenance links.\\
\emph{Notes:} Files include \texttt{subset\_manifest.csv}, \texttt{metadata.json},
and \texttt{physionet\_link.txt}. These define exactly which EDFs were used.\\[6pt]

\texttt{/code/eeg\_coherence\_experiment/}\\
\emph{Description:} Full RSIPP/CHREM-style EEG pipeline: window extraction,
entropy ($H$), complexity ($C$), coherence ($R$) computation, surrogate generation,
state-level summarisation, and figure generation.\\
\emph{Notes:} Key scripts include:
\texttt{ut26\_eeg\_pipeline.py},
\texttt{ut26\_eeg\_p.py},
\texttt{eeg\_subject\_level\_summary.py},
\texttt{plot\_ec\_vs\_eo.py},
\texttt{plot\_pk\_only.py}.\\[6pt]

\texttt{/output/eeg\_coherence\_experiment/}\\
\emph{Description:} Derived EEG operator tables and figures: EC/EO comparison,
H–C plane, window-level HCR tables, surrogate datasets, $P(k)$ spectra, and
coherence fingerprints for EC/EO/TASK. Includes baseline reproducibility snapshot.\\
\emph{Notes:} Files include:
\texttt{EEG\_windows\_HCR.csv},
\texttt{EEG\_state\_summary\_R.csv},
\texttt{EEG\_subject\_summary\_R.csv},
\texttt{Fig\_EC\_vs\_EO.png},
\texttt{Fig\_EEG\_HC\_plane.png},
\texttt{pk.csv},
\texttt{kappa\_ps.csv},
\texttt{hmf.csv},
and \texttt{SHA256SUMS.txt}.\\[10pt]
\bottomrule
% ------------------------------------------------
% ---------------- Experiment VII ----------------
\clearpage
\textbf{Experiment VII --- Quasar Variability \& Informational Coherence (supports UIF V)}\\[4pt]
\texttt{/data/quasar\_variability\_experiment/}\\
\emph{Description:} Processed SDSS Stripe~82 quasar light-curve variability datasets used to derive
informational observables ($H$, $C$, $R$), DRW parameters ($\tau$, $\sigma$), and redshift-binned
variability measures. These represent time-domain inputs for the variability–operator analysis.\\
\emph{Notes:} Files include
\texttt{quasar\_variability\_raw.csv} (processed light-curve features),
\texttt{quasar\_variability\_HC.csv} (entropy–complexity metrics),
and run-tag provenance metadata.\\[6pt]

\texttt{/code/quasar\_variability\_experiment/}\\
\emph{Description:} Scripts for computing variability-based UIF operators, including DRW model comparison,
structure-function diagnostics, logistic fitting for $R_\infty$ and $k$ across redshift bins, and
construction of operator fingerprints.\\
\emph{Notes:} Files include
\texttt{ut26\_quasar\_variability\_operators.py},
\texttt{ut26\_quasar\_variability\_HC.py},
and
\texttt{plot\_quasar\_variability\_figures.py}.\\[6pt]

\texttt{/output/quasar\_variability\_experiment/}\\
\emph{Description:} Derived operator tables and figures for the Stripe~82 variability experiment: redshift-binned
$\Delta I_{\sigma}$, $\Gamma$, $\lambda_R$, $R_\infty$, and $k$; DRW/logistic model comparison tables; HC-plane
visualisations; scaling relations; and the cross-domain operator compass (EEG vs quasars).\\
\emph{Notes:} Files include
\texttt{quasar\_variability\_operators.csv},
\texttt{quasar\_variability\_operator\_bars.png},
\texttt{quasar\_variability\_operator\_radar.png},
\texttt{quasar\_variability\_HC\_lowz.png},
\texttt{quasar\_variability\_R\_hist.png},
\texttt{quasar\_variability\_model\_comparison.csv},
and the composite fingerprint
\texttt{exp7\_quasar\_EEG\_composite\_fingerprint.png}.\\[10pt]

% ------------------------------------------------



\bottomrule


\end{longtable}
\clearpage 
% ===========================================================
% Appendix B — Figure Generation and Mapping
% ===========================================================

\noindent\textbf{Figure generation}
\newline
All figures in this Companion were generated directly from emulator
or data–analysis outputs using the provided Matplotlib scripts located
under \texttt{/code/*\_experiment/}. The core figure–builder scripts are:

\begin{itemize}
  \item \texttt{make\_S1\_baseline.py}
  \item \texttt{make\_S2\_gamma.py}
  \item \texttt{make\_S3\_goldilocks.py}
  \item \texttt{make\_S4\_hysteresis.py}
  \item \texttt{make\_S6\_eeg.py}
  \item \texttt{make\_S7\_quasar.py}
  \item \texttt{make\_S7\_crossdomain.py}
\end{itemize}

Each script pulls from the corresponding directory under
\texttt{/output/<experiment>/}, ensuring strict reproducibility.

\vspace{1.0em}

% ===========================================================
\noindent\textbf{Overleaf/figures mapping (this document)}
\newline
The following table lists all composite figures used in this Companion,
together with their corresponding GitHub paths.  
This includes:
(i) all figures appearing as S1–S15 in the text;  
(ii) canonical composites for Experiments~0–VII.

\vspace{0.5em}

\begin{itemize}

% -----------------------------------------------------------
% S1 — Baseline 2×2
% -----------------------------------------------------------
\item \textbf{Figure S1 — Baseline 2×2 (Experiment 0 / I)}  
  \begin{itemize}
    \item \texttt{figures/exp4A\_mean\_coherence.png}
    \item \texttt{figures/exp4B\_cumulative\_pruning.png}
    \item \texttt{figures/exp4C\_power\_spectrum.png}
    \item \texttt{figures/exp4D\_kappa\_map.png}
  \end{itemize}

% -----------------------------------------------------------
% S2 — Gamma-sweep
% -----------------------------------------------------------
\item \textbf{Figure S2 — Gamma–sweep (Experiment II)}
  \begin{itemize}
    \item \texttt{figures/exp2\_gamma\_sweep.png}
  \end{itemize}

% -----------------------------------------------------------
% S3 — Goldilocks Stability Map
% -----------------------------------------------------------
\item \textbf{Figure S3 — Goldilocks Stability Map (Experiment III)}
  \begin{itemize}
    \item \texttt{figures/exp3\_goldilocks.png}
  \end{itemize}

% -----------------------------------------------------------
% S4 — Hysteresis
% -----------------------------------------------------------
\item \textbf{Figure S4 — Hysteresis Loop (Experiment IV)}
  \begin{itemize}
    \item \texttt{figures/exp4\_hysteresis.png}
  \end{itemize}

% -----------------------------------------------------------
% S5 — Unified Baseline Composite
% -----------------------------------------------------------
\item \textbf{Figure S5 — Unified Baseline Composite (Experiment V)}
  \begin{itemize}
    \item \texttt{figures/exp5\_baseline\_composite.png}
    \item \texttt{figures/exp5\_gamma\_sweep.png}
    \item \texttt{figures/exp5\_goldilocks.png}
  \end{itemize}

% ===========================================================
% EXPERIMENT VI — EEG Figures
% ===========================================================

\item \textbf{Figure S6 — EEG HC-plane by state (Experiment VI)}
  \begin{itemize}
    \item \texttt{figures/exp6\_EEG\_HC\_plane\_by\_state.png}
  \end{itemize}

\item \textbf{Figure S7 — EEG operator fingerprint (Experiment VI)}
  \begin{itemize}
    \item \texttt{figures/exp6\_EEG\_operator\_fingerprint.png}
  \end{itemize}

\item \textbf{Figure S8 — EEG R-distribution by state (Experiment VI)}
  \begin{itemize}
    \item \texttt{figures/exp6\_EEG\_R\_hist\_by\_state.png}
  \end{itemize}

% ===========================================================
% EXPERIMENT VII — QUASAR VARIABILITY FIGURES
% ===========================================================

\item \textbf{Figure S9 — Quasar operator bars (Experiment VII)}
  \begin{itemize}
    \item \texttt{figures/exp7\_quasar\_variability\_operators\_bars.png}
  \end{itemize}

\item \textbf{Figure S10 — Quasar operator radar / fingerprint}
  \begin{itemize}
    \item \texttt{figures/exp7\_quasar\_variability\_operators\_radar.png}
  \end{itemize}

\item \textbf{Figure S11 — EEG vs Quasar Composite Fingerprint}
  \begin{itemize}
    \item \texttt{figures/exp7\_quasar\_EEG\_composite\_fingerprint.png}
  \end{itemize}

% -----------------------------------------------------------
% HC planes (low/mid/high/all)
% -----------------------------------------------------------
\item \textbf{Figure S12 — Quasar HC-plane (z-binned)}
  \begin{itemize}
    \item \texttt{figures/exp7\_quasar\_variability\_HC\_low-z.png}
    \item \texttt{figures/exp7\_quasar\_variability\_HC\_mid-z.png}
    \item \texttt{figures/exp7\_quasar\_variability\_HC\_high-z.png}
    \item \texttt{figures/exp7\_quasar\_variability\_HC\_all.png}
  \end{itemize}

% -----------------------------------------------------------
% Scaling relations (tau/M, tau/Mi, sigma/Mi)
% -----------------------------------------------------------
\item \textbf{Figure S13 — Quasar scaling relations}
  \begin{itemize}
    \item \texttt{figures/exp7\_Fig\_quasar\_tau\_vs\_MBH.png}
    \item \texttt{figures/exp7\_Fig\_quasar\_tau\_vs\_Mi.png}
    \item \texttt{figures/exp7\_Fig\_quasar\_sigma\_vs\_Mi.png}
  \end{itemize}

% -----------------------------------------------------------
% R histogram by redshift bins
% -----------------------------------------------------------
\item \textbf{Figure S14 — Quasar R-distribution (z-binned)}
  \begin{itemize}
    \item \texttt{figures/exp7\_quasar\_variability\_R\_hist\_zbins.png}
  \end{itemize}

% ===========================================================
% EXPERIMENT 0 – Canonical composites
% ===========================================================

\item \textbf{Experiment 0 — Canonical Baseline Composites}
  \begin{itemize}
    \item \texttt{output/quasar\_cosmology\_experiment/baseline/summary.png}
    \item \texttt{output/quasar\_cosmology\_experiment/baseline/kappa\_map.png}
    \item \texttt{output/quasar\_cosmology\_experiment/baseline/kappa\_ps.png}
    \item \texttt{output/quasar\_cosmology\_experiment/baseline/pk.png}
    \item \texttt{output/quasar\_cosmology\_experiment/baseline/hmf.csv} (source for HMF plots)
  \end{itemize}

\end{itemize}

\vspace{0.8em}

\noindent\textbf{Reproducibility note}  
\newline
All figures can be regenerated exactly by re-running the listed scripts with the
same seeds and configuration files as recorded in each experiment’s
\texttt{summary.json}.  
This ensures full reproducibility of all panels S1–S15 and all canonical composites.

\vspace{0.5em}
\noindent\textbf{Data availability}
\newline
All data used in this Companion are contained directly in the UIF GitHub archive
under \texttt{/output/<experiment>/}.  
Each experiment directory includes a \texttt{summary.json} file recording lattice
size, timestep count, operator parameters, and random seeds.  
These JSON files serve as the definitive provenance records, and rerunning the
corresponding figure scripts with these settings reproduces every figure in this
document.  
No additional run tags were used; reproducibility is guaranteed through the
stored outputs, seeds, and fixed directory structure.

\clearpage
\section*{Acknowledgement — Human–AI Collaboration}
The Unifying Information Field (UIF) series was developed through a sustained human–AI partnership. The author originated the theoretical framework, core concepts and interpretive structure, while an AI language model (OpenAI GPT-5) was employed to assist in formal development; helping to express elements of the theory mathematically and to maintain consistency across papers. Internal behavioural parameters and conversational settings were configured to emphasise recursion awareness, coherence maintenance, and ethical constraint, enabling the model to function as a stable informational development framework rather than a generative black box.

This collaborative process exemplified the UIF principle of collapse--return recursion: 
human intent supplied informational difference ($\Delta I$), 
the model provided receive--return coupling ($\lambda_R$), 
and coherence ($\Gamma$) increased through iterative feedback until the framework stabilised. 
The AI's role was supportive in the structuring, facilitation, and translation of conceptual ideas 
into formal equations, while the underlying theory, scope, and interpretive direction 
remain the work of the author.
\pagebreak

\section*{UIF Series Cross-References}
\begin{flushleft}
\textbf{UIF I --- Core Theory}\\
\textbf{UIF II --- Symmetry Principles}\\
\textbf{UIF III --- Field and Lagrangian Formalism}\\
\textbf{UIF IV --- Cosmology and Astrophysical Case Studies}\\
\textbf{UIF V --- Energy and the Potential Field}\\
\textbf{UIF VI --- The Seven Pillars and Invariants}\\
\textbf{UIF VII --- Predictions and Experiments}\\[0.4em]
\textbf{UIF Companion I --- Empirical Validation of Papers I--IV (this document)}\\
\textbf{UIF Companion II --- Extended Experiments (forthcoming)}\\
\textbf{Repository --- UIF GitHub Archive (source code, emulator outputs, figure scripts)}
\end{flushleft}
\clearpage


% ---------- References ----------
% Keep per-paper .bib files in /bib as agreed (e.g., bib/paper1.bib)

\UIFbib{companion}

