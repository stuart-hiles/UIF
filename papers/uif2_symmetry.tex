% ===== UIF Paper II — Symmetry Principles =====
% Numbering: Paper 2 → (2.x) equations, figures, tables
\UIFpaper{2}

\UIFmetadata{The Unifying Information Field (UIF) Paper II — Symmetry Principles}
            {Stuart E. N. Hiles}
            {UIF II — Symmetry Principles}

\hypersetup{
  pdftitle={The Unifying Information Field (UIF) Paper II — Symmetry Principles},
  pdfauthor={Stuart E. N. Hiles},
  pdfsubject={information, theory, informational physics, collapse–return dynamics, coherence, recursion, symmetry, quantum, cosmology, dark energy, dark matter, unification, AI, consciousness, biology, cognition, UIF, Unifying Information Field, Physical cosmology, Theoretical physics, Information theory, Physics}
}

\title{The Unifying Information Field (UIF) Paper II\\[0.35em]
\Large\textit{Symmetry Principles}\\[0.6em]   
\small Version v1.3 — November 2025}
\author{Stuart E.\,N. Hiles, BA (Hons)}
\date{}

\begin{center}
\thispagestyle{empty}
\vspace{2em}
{\small
© 2025 Stuart E. N. Hiles \\[4pt]
Licensed under the Creative Commons Attribution–NonCommercial 4.0 International (CC BY-NC 4.0) License. \\[6pt]
This document represents a pre-release version (v1.3, November 2025) of the \\
\textit{Unifying Information Field (UIF)} series of papers.\\[0.75em]
First published on Zenodo\\
DOI (Concept): \href{https://doi.org/10.5281/zenodo.17468871}{10.5281/zenodo.17468871}\\
Series DOI: \href{https://doi.org/10.5281/zenodo.17434412}{10.5281/zenodo.17434412}\\
Source and LaTeX archived at: \href{https://github.com/stuart-hiles/UIF}{https://github.com/stuart-hiles/UIF} (GitHub release v2.0)
\newline

This paper has not yet been peer-reviewed or formally published.\\[0.5em]
All supporting software, scripts, and data are licensed separately under \textbf{GPL-3.0}.
} % end \small
\end{center}

\maketitle

\begin{abstract}
\thispagestyle{empty}
Building on UIF I — Core Theory, which defined the collapse–return operators of informational dynamics, this second paper develops the symmetry and invariance structure of the Unifying Information Field. The triadic operators ($\Delta I, \Gamma, \lambda_R$) are shown to generate four fundamental informational symmetries: conservation, lawful breaking, scale invariance, and collapse-frame invariance. These symmetries generalise Noether’s theorem to informational space, unifying conservation and transformation laws across physics, biology, and computation.

The paper demonstrates that each symmetry preserves informational coherence under transformation, while their controlled breaking gives rise to structure, diversity, and temporal directionality. Empirical implications include measurable signatures in quasar variability, neural synchrony, and coherence thresholds, positioning UIF symmetry principles as the bridge between informational theory and observable dynamics.

\noindent\textit{Empirical bridge.}
High-resolution \textit{JWST} and \textit{EHT} observations now reveal
astrophysical symmetry-breaking and recursion phenomena consistent with UIF’s
operators: recursion ($\Gamma$), receive--return coupling ($\lambda_R$), and
threshold bias ($\beta$) \cite{Perlman2025_M87_JWST,EHT2024_Polarization_M87}.

Empirical implications include measurable signatures in quasar variability, neural
synchrony, and coherence thresholds, positioning UIF symmetry principles as the
bridge between informational theory and observable dynamics. These signatures are
now partially constrained by cross-domain operator calibration in the
\textit{UIF Companion Experiments} (quasar variability, emulator, EEG),
providing first quantitative bounds on the symmetry structure.
\end{abstract}
\clearpage

\thispagestyle{empty}
\noindent \textbf{Series overview} 
\newline Paper I introduces the Unifying Information Field (UIF) as a collapse–return informational framework and defined its operator grammar. Paper II develops the symmetry and invariance principles underlying informational conservation; Paper III establishes the field and Lagrangian formalism; Paper IV applies the framework to cosmology and astrophysical case studies; Paper V formulates the energetic and potential field laws; Paper VI synthesises the invariant architecture; and Paper VII (forthcoming) consolidates predictions and experiments.
\newline

\noindent\textbf{Companion}
\newline
Symbolic derivations, operator-symmetry validations, and reproducibility metadata
supporting this paper are archived in the
\textit{UIF~Companion Experiments} (2025) \cite{Companion2025}.
That volume documents the numerical checks used to verify the Noether-type
continuity and symmetry equations developed here.
A second release, \textit{UIF Companion II — Extended Experiments}
(forthcoming 2025), will include higher-order emulator runs exploring
symmetry-breaking dynamics.
\newline

\noindent\textbf{Repository}
\newline
All source code, symbolic notebooks, and figure-generation scripts are maintained in the
UIF GitHub Archive (\url{https://github.com/stuart-hiles/Unifying-Information-Field}).
This includes notebooks reproducing the Noether-type continuity equation (2.1)
and the $\beta$-bias simulations of Section 3.  Each dataset and script is versioned by
\texttt{RUN\_TAG} for reproducibility.
\newline

\noindent\textbf{Note on Nomenclature and Continuity}
\newline
The Unifying Information Field (UIF) framework presented here continues directly from
\textit{UIF I — Core Theory} and supersedes preliminary UT26 terminology.
All operator symbols and relations remain consistent with those definitions but are
now expressed as symmetry principles that form the foundation for the field formalism
in \textit{UIF III}.
This ensures notation and interpretation remain continuous throughout the series.
\newline

\noindent\textbf{Scope}
\newline
This paper establishes the symmetry and invariance structure of the Unifying Information Field.
It generalises Noether’s theorem to informational space, showing how the operators
($\Delta I$, $\Gamma$, $\beta$, $\lambda_R$) generate four fundamental symmetries:
informational conservation, symmetry breaking, scale invariance, and
collapse-frame invariance.  These principles unify conservation and transformation
laws across physics, biology, and computation, and prepare the ground for the
field-level formulation in \textit{UIF III}.

\noindent The symmetry and invariance laws developed here form the theoretical foundation for
the continuous variational treatment presented in \textit{UIF III — Field and Lagrangian Formalism},
where these operators are expressed through the Lagrangian density and Euler–Lagrange
equations governing informational dynamics.
\clearpage

\pagenumbering{arabic}
\setcounter{page}{1}
\section{Introduction}
Symmetry principles are central in physics: invariances define conservation laws, 
and when symmetries break, new structure emerges. 
Noether's theorem (1918)\cite{Noether1918} showed that every invariance corresponds to a conserved quantity, 
while Lorentz invariance (Einstein, 1905)\cite{Einstein1905} ensures that physical laws hold across inertial frames. 
Gauge symmetries underpin the Standard Model (Yang and Mills, 1954)\cite{YangMills1954}, 
and symmetry breaking in the Higgs field generates mass (Higgs, 1964)\cite{Higgs1964}. 
Across physics, symmetry both constrains and creates: it preserves laws when unbroken 
and gives rise to order when broken.

Within UIF, Noether-like invariance arises from the closure of the triadic cycle: 
sampling, recursion, and return. 
The triad conserves informational difference in exactly the same way that symmetry 
conserves physical quantities (Shannon, 1948; Noether, 1918; Tesla, 1892)\cite{Shannon1948,Noether1918,Tesla1892}.

\noindent In UIF, the same foundation is extended into informational dynamics. Section 1 defined $\Delta I, \Gamma, \beta$, and $\lambda_R$ as the four fundamental operators of informational collapse. Here, we reframe these operators as symmetry principles, showing how UIF mirrors and generalises the invariance tradition of physics:
\needspace{10\baselineskip}
\begin{itemize}
  \item $\Delta I$ conservation parallels energy and charge conservation.
  \item $\beta$ encodes symmetry breaking as informational bias.
  \item $\Gamma$ guarantees recursive stability and universality across scales.
  \item $\lambda_R$ enforces collapse-frame invariance through substrate coupling.
\end{itemize}
\medskip
\noindent
\noindent
Formally, these symmetry relations can be expressed through a Noether–type continuity equation:
\begin{equation}
\partial_t(\Phi^2) + \nabla\!\cdot J_{\Phi} = 0,
\qquad J_{\Phi} = \Phi\nabla\Phi ,
\label{eq:2-noether}
\end{equation}
which defines conservation of informational difference under continuous transformation of $\Phi$.
This relation generalises classical Noether invariance to informational systems and becomes the
foundation for the variational field equations developed in \textit{UIF~III}.

Thus, the Noether-type continuity in Eq.~(\ref{eq:2-noether}) provides the mathematical precursor to the Lagrangian field equations developed in \textit{UIF III}, where informational conservation becomes a dynamical variational condition.

\noindent\textit{Empirical link.}
High-resolution \textit{JWST} and \textit{EHT} observations of M87 display recursive coherence
and lag-coupled feedback consistent with UIF’s conservation grammar, offering
observational analogues for the receive--return coupling ($\lambda_R$) and recursion ($\Gamma$)
\cite{Perlman2025_M87_JWST,EHT2024_Polarization_M87}.


As in physics, these principles extend beyond a single domain. Biological networks conserve
information even as carriers change. Artificial systems rely on weighted biases to break symmetry
and drive learning. Collective systems show scale invariance in language, economics, and opinion dynamics.
UIF frames these as manifestations of the same deeper laws, extending symmetry beyond physics into
informational, biological, artificial, and social domains.
\newline
\newline For transparency, all numbered equations in this paper are classified according to their provenance:
[Identity] designates a standard physical or informational law, [Model law] a relation derived
within the UIF framework from stated assumptions, and [Hypothesis] a phenomenological or
testable scaling introduced for future verification. A complete table of equation provenance and
accompanying symbol definitions is provided in Appendix A.
\newline

\noindent The following four principles; informational conservation, symmetry breaking, scale invariance, and collapse-frame invariance each demonstrate how UIF preserves the rigour of physics while extending invariance across all complex systems.

\section{Informational Conservation}
Conservation is a cornerstone of physics. 
Noether (1918)\cite{Noether1918} linked symmetries to conservation laws, 
while thermodynamics, through Boltzmann (1877)\cite{Boltzmann1877}, 
Jaynes (1957)\cite{Jaynes1957}, and Seifert (2012)\cite{Seifert2012}, 
constrains energy and entropy. 
In UIF, these reduce to conservation of informational difference: 
$\Delta I$ cannot be destroyed, only redistributed between local systems 
and the substrate.
This can be expressed formally, both as a global balance condition and as a local dynamical law.

\paragraph{Integral form.}
\begin{equation}
\Delta I_{\text{sys}} + \Delta I_{\text{sub}} = \text{constant}.
\label{eq:2-1}
\end{equation}

\paragraph{Differential form (schematic).}
\begin{equation}
\frac{d}{dt}\Delta I_{\text{sys}} + \frac{d}{dt}\Delta I_{\text{sub}} = 0.
\label{eq:2-2}
\end{equation}

\noindent The integral form captures the global conservation of informational difference, while the differential form describes its local exchange between system and substrate. This principle unifies across scales: DNA repair conserves information in genomes, error-correcting codes preserve digital states, and social systems store collective memory.
\newline 

\noindent Some argue that not all conservation laws reduce neatly to information 
(Anderson, 1972; Timpson, 2013)\cite{Anderson1972,Timpson2013}. 
UIF addresses this by tying $\Delta I$ strictly to measurable redistributions 
in collapse--return events. 
This restates the $\Delta I$ conservation principle first introduced in 
\textit{UIF~I - Core Theory}\cite{UIF-I}, now expressed through 
Noether-type invariance.
\newline

\noindent\textbf{UIF Alignment}

\noindent Informational conservation reframes the conservation of energy, momentum, and entropy 
as special cases of $\Delta I$ redistribution. 
This alignment also extends to results in logic and computation. 
G{\"o}del (1931)\cite{Godel1931}, Turing (1936)\cite{Turing1936}, 
and Rice (1953)\cite{Rice1953} showed that certain operations are provably irreversible, 
while Landauer (1961)\cite{Landauer1961} demonstrated that logically irreversible operations 
incur an entropy cost of $k_{B}T \ln 2$ per bit erased. 
Collapse--return therefore conserves $\Delta I$ by necessity: 
\textit{each event injects entropy into the substrate but cannot erase informational difference.}
\newline

\noindent \textbf{Synthesis}
\newline Informational conservation forms the first symmetry of UIF, expressing that all lawful transformations preserve total informational difference even as local forms change. This principle links UIF directly to Noether’s invariance logic and prepares the ground for understanding how asymmetry; addressed next through $\beta$, generates structure.
\newline

\noindent\textbf{Forward Pointer}
\newline The next section examines how the $\beta$ operator lawfully breaks this conservation symmetry, introducing direction and structure into collapse outcomes.
\newline

\noindent\textbf{Novelty / Testability}
\newline Observable in $\Delta I$ budgets in cosmology (CMB residuals, large-scale structure) and in laboratory collapse dynamics; these empirical tests are developed further in UIF VII - Predictions and Experiments.

\section{Symmetry Breaking}
In UIF, symmetry breaking is the action of the $\beta$ operator, 
which encodes bias in collapse outcomes. 
Physics shows structure arising from broken symmetries - 
the Higgs mechanism giving mass (Higgs, 1964)\cite{Higgs1964}, 
or condensed-matter transitions crystallising from uniform states (Anderson, 1972)\cite{Anderson1972}. 
UIF generalises this: whenever collapse offers symmetric possibilities, 
$\beta$ tilts the outcome probabilities.

Collapse initiation requires perturbation. 
A perfectly coherent system will remain symmetric until disturbed above a minimum noise threshold. 
This threshold is relative to the system’s coherence: fragile systems collapse with tiny perturbations, 
but maximally coherent systems require proportionally richer fluctuations. 
In the early universe, only very high-frequency perturbations 
(analogous to a $\gamma$-burst) could destabilise perfect symmetry and trigger the first collapse. 
The same operator reappears in biology, where $\gamma$ rhythms break local symmetry 
and integrate coherence across brain regions.

\noindent\textit{Astrophysical analogue.}
\textit{JWST} imaging of the M87 jet reveals alternating bright/dark knots and spectral-index
gradients at recollimation sites, indicative of symmetry breaking via $\beta$ with coherence
restored by receive--return coupling $\lambda_R$; together these mirror UIF’s bias–return cycle
in a real system \cite{Perlman2025_M87_JWST}.


Outcome selection follows bias. 
Once collapse is triggered, outcomes are not chosen uniformly 
but according to $\beta$ operators that weight the landscape. 
Beyond tilting probabilities, collapse also fixes conserved topological invariants (
spin, charge, and chirality), consistent with topological solutions in field theory 
(Skyrme, 1961; ’t~Hooft, 1974; Polyakov, 1974)\cite{Skyrme1961,tHooft1974,Polyakov1974}. 
These invariants persist across scales: in particle physics they stabilise matter, 
while in neural systems $\gamma$-oscillations act as a universal archetype of symmetry breaking 
and integration.

Recent neuroscience evidence supports this cross-domain view. 
Large multi-lab tests published in \textit{Nature} (Cogitate Consortium, 2025)\cite{Cogitate2025} 
show that conscious contents are most robustly encoded in posterior sensory cortex, 
with frontal regions contributing more to reporting and categorisation than to content itself. 
Parallel work by the Yale/Allen Institute Collaboration (2025)\cite{YaleAllen2025} 
highlights deep, evolutionarily ancient hubs (notably the thalamus and midbrain), 
as gateways linking multisensory input to awareness. 
This suggests that $\gamma$-driven collapse and $\beta$-biasing operate in both older neural 
substrates and cortical circuits, consistent with UIF’s prediction that symmetry-breaking 
dynamics underpin baseline awareness and higher integration.

\paragraph{Softmax / Boltzmann bias.}
\begin{equation}
P_i \;=\; \frac{\exp\!\left(\beta\, x_i\right)}{\sum_j \exp\!\left(\beta\, x_j\right)}.
\label{eq:2-3}
\end{equation}


\noindent This distribution is identical to the Boltzmann form in statistical physics and the softmax operator in AI, showing that UIF generalises a universal biasing mechanism once collapse has been triggered. $\beta$ therefore acts as the field’s symmetry-breaking parameter, analogous to temperature-like bias controlling collapse outcomes. The bias parameter $\beta$ thus serves as an informational analogue of temperature, determining the degree of symmetry breaking.

Examples span physics (electroweak symmetry breaking), 
artificial intelligence (weighted neural thresholds), 
and collective systems (social rules biasing outcomes). 
Modern experiments continue to probe symmetry breaking. 
Aspect, Clauser, and Zeilinger (2022)\cite{Aspect2022} 
highlighted quantum entanglement as evidence of fundamental symmetry violation, 
reinforcing the need for an informational framing.
\newline

\noindent\textbf{UIF Alignment}
\newline Collapse initiation requires perturbation above a noise threshold proportional to system coherence; explaining why gamma-scale fluctuations act as archetypal symmetry breakers, from cosmic origins to neural integration. Once collapse is triggered, $\beta$ operators bias outcome probabilities in softmax/Boltzmann form, linking physics, AI, and collective systems.
\newline

\noindent\textbf{Synthesis}
\newline Symmetry breaking introduces lawful bias into collapse dynamics. Across domains (from cosmic structure formation to neural integration) $\beta$ translates homogeneity into diversity while retaining overall informational accounting.
\newline

\noindent \textbf{Forward Pointer}
\newline Having shown how $\beta$ introduces asymmetry, the following section extends the argument to scale, demonstrating that the same operators act self-similarly from quantum to cosmic systems.
\newline

\noindent\textbf{Novelty / Testability}
\newline Observable in GRB spectra, EEG $\gamma$-band coherence, and machine-learning weight distributions; symmetry-breaking thresholds should manifest as quantised transitions in collapse frequency or coherence amplitude. Bias signatures should appear in AI learning-weight distributions and in statistical asymmetries of collapse outcomes.

\section{Scale Invariance}
In UIF, \emph{scale invariance} means that the same collapse--return operators 
($\Delta I$, $\Gamma$, $\beta$, $\lambda_R$) govern dynamics independently of 
system size. Physics first revealed this principle through fractal geometry and 
power-law behaviour (Mandelbrot 1982; Bak, Tang \& Wiesenfeld 1987)\cite{Mandelbrot1982,Bak1987}. 
Mandelbrot showed that galaxy clustering, turbulence, and other natural phenomena 
follow scale-free patterns. UIF generalises this insight: collapse--return dynamics 
remain self-similar under scaling transformations, with the informational operators 
acting consistently across physical, biological, and computational domains.
\newline 

\noindent UIF reframes these patterns as informational symmetries. Formally, scale invariance can be expressed as the invariance of the operator set under multiplicative scaling transformations:

\begin{equation}
\mathcal{O}(\Delta I, \Gamma, \beta, \lambda_R) \;\forall\, S > 0.
\label{eq:2-4}
\end{equation}

\noindent
Infrared spectral-index gradients along the M87 jet show power-law behaviour in brightness
and coherence length consistent with recursion-driven scale invariance, providing a tangible
astrophysical instance of Eq.~(\ref{eq:2-4}) \cite{Perlman2025_M87_JWST}.


\noindent Recent analyses of quasar variability demonstrate this principle directly: 
variability timescales scale with black-hole mass and brightness, 
with epoch-dependent modifications, exemplifying $\Gamma$'s action across astrophysical scales 
(as discussed in \textit{UIF~IV --- Cosmology and Astrophysical Case Studies}, §~4.10)\cite{UIF-IV}. 
Similar scaling laws are observed in biological and cognitive domains: 
EEG coherence shows power-law distributions across frequency bands, 
and collective synchronisation in oscillator networks also follows $\Gamma$-driven scaling. 
Consciousness-related $\gamma$ coherence peaks in posterior cortices and is supported by subcortical relays, 
indicating that $\Gamma$'s scale-free action spans evolutionary layers (ancient~$\rightarrow$~cortical) 
as well as domains (neural~$\rightarrow$~astrophysical~$\rightarrow$~collective) 
(Cogitate Consortium, 2025; Yale/Allen Institute Collaboration, 2025)\cite{Cogitate2025,YaleAllen2025}. 
This suggests that $\Gamma$ defines a universal symmetry across physics and biology, 
linking astrophysical variability with neural and social coherence.

Paper~VI formalises this property as the \textit{Scalar Invariance Lemma},
showing that after normalisation by reference scales
$(\Delta I_0,\tau_0,L_0)$ the operator manifold
$(R_\infty,k,\lambda_R,\Gamma)$ collapses onto a common geometry across
quasars, the informational emulator, and EEG experiments. The
\textit{UIF Companion Experiments} provide the empirical calibration for
this collapse, indicating that Eq.~(\ref{eq:2-4}) is not merely a
heuristic symmetry but a quantitatively constrained invariance.
\noindent

The same recursion law that shapes galactic clustering also governs
neural $\gamma$-band synchrony, hinting that coherence itself—not scale—is
the true invariant of nature.
\newline

\noindent\textbf{UIF Alignment}
\newline $\Gamma$ governs recursion and timing; its scale-free operation links quasar variability, neural $\gamma$ synchrony, and collective oscillations.
\newline

\noindent\textbf{Synthesis}
\newline Scale invariance establishes that the UIF operators retain form under dilation, supporting the claim that informational dynamics are universal across magnitudes and domains.
\newline

\noindent\textbf{Forward Pointer}
\newline The final symmetry (\textit{collapse-frame} invariance) extends this universality across observers, ensuring that informational laws remain consistent regardless of reference frame.
\newline

\noindent{\textbf{Novelty / Testability}
\newline Predicted in power-law exponents of quasar light curves, EEG frequency spectra, and synchronisation statistics in coupled-oscillator networks.

\section{Collapse-Frame Invariance}
In UIF, collapse-frame invariance generalises Lorentz invariance (Einstein, 1905)\cite{Einstein1905}. 
Any physical, biological, artificial, or collective system that samples $\Delta I$ 
and couples to $R(x,t)$ defines a frame. 
Collapse-frame invariance states that outcome probabilities are independent of observer type. 
This principle is also empirically testable. 
Fifth-force searches already constrain possible observer-dependent couplings. 
Recent work on screened scalar fields (Fischer et~al., 2024)\cite{Fischer2024}, 
asteroid tracking with OSIRIS--REx (Tsai et~al., 2024)\cite{Tsai2024}, 
and dark-photon searches with MADMAX prototypes (Egge et~al., 2025)\cite{Egge2025} 
all limit deviations from collapse-frame invariance, consistent with $\lambda_R$ enforcing 
universality across frames.

\paragraph{Formal expression.}
\begin{equation}
P\!\left(\Delta I \,\middle|\, F \right) \;=\; P(\Delta I),
\label{eq:2-5}
\end{equation}
where $F$ denotes any informational frame sampling $\Delta I$ through the substrate $R(x,t)$.
\newline

\noindent{\textbf{UIF Alignment}
\newline $\lambda_R$ acts as the coupling constant ensuring informational exchange obeys the same laws in all frames, generalising Lorentz invariance to informational space.
\newline

\noindent\textbf{Synthesis}
\newline Collapse-frame invariance completes the symmetry set: conservation, breaking, scaling, and observer independence form a closed informational group analogous to the symmetry families of physics.
\newline

\noindent\textbf{Forward Pointer}
\newline These invariances provide the foundation for UIF III, which formalises them through a field and Lagrangian framework.
\newline

\noindent{\textbf{Novelty / Testability}
\newline Bounded by current fifth-force and dark-photon limits; further tests include gravitational-wave echo delays and precision quantum-memory reciprocity experiments.

\section{Closing Synthesis and UIF Alignment}
Together, these symmetry principles ensure that UIF retains the rigour of physics while 
extending its reach. 
$\Delta I$ conservation shows that information persists through redistribution, consistent 
with logical irreversibility \cite{Godel1931,Turing1936,Rice1953} and Landauer’s entropy cost. 
$\beta$ explains how order arises through symmetry breaking and establishes conserved 
topological invariants such as spin, charge, and chirality 
\cite{Skyrme1961,tHooft1974,Polyakov1974}. 
$\Gamma$ scale invariance guarantees recursive stability and universality across 
astrophysical, biological, and collective domains, while 
$\lambda_R$ collapse--frame invariance ensures outcomes are observer--independent, 
consistent with constraints from fifth--force and dark--photon searches 
\cite{Fischer2024,Tsai2024,Egge2025}.

By reframing UIF’s operators (Section~1) as symmetry principles, the framework anchors 
itself within the physics tradition of invariance, conservation, and symmetry breaking, 
yet extends these laws across informational, biological, artificial, and collective systems. 
These principles also explain why not all possibilities are realised: collapse follows 
lawful pruning governed by thresholds, biases, and invariant conservation, ruling out 
naïve Many--Worlds interpretations. 

Neuroscience adds further constraint: $\Gamma$ aligns with posterior $\gamma$--band recursion, 
$\beta$ with biasing in sensory collapses, and $\lambda_R$ with thalamic return loops, 
reinforcing that these operators act as universal invariants across neural, astrophysical, 
and cosmological scales. 
This dual anchoring provides both continuity with known science and a foundation for 
testable predictions---including coherence echoes, bias signatures, power--law scaling, 
and observer--independent collapse probabilities.

In \textit{UIF VI}, these four symmetries are recast as part of the
derived law set (L1--L10), with informational conservation, symmetry
breaking, scale invariance, and collapse–frame invariance forming the
core invariance structure from which the later laws are built.
\newline

\noindent\textbf{Empirical Outlook}
\newline Astrophysical data now make these symmetries testable: the M87 jet constrains
($\lambda_R,\Gamma,\eta^\ast$) through polarization–flux cycles and IR gradients;
quasar variability and emulator runs constrain $(R_\infty,k,\lambda_R,\Gamma)$;
and ORC morphologies probe receive--return fronts at galactic scales.
Precision asteroid tracking with \textit{OSIRIS-REx} \cite{Tsai2024_Bennu_FifthForces}
further constrains deviations from collapse-frame invariance, complementing
laboratory tests of dark-photon couplings \cite{Egge2025}. These anchors provide a concrete path
from the invariance principles established here to the field equations and cosmological dynamics in \textit{UIF~III–IV}.
\newline

\noindent \textbf{Forward Pointer}
\newline Together these symmetry laws close the first theoretical cycle of UIF: the operators defined in UIF I here acquire mathematical invariance, providing the foundation for the field formalism developed in UIF III. The next paper, UIF III, applies these principles to the field level, where the collapse–return cycle is examined directly in the photon as the simplest informational quantum system.
\newline

\noindent{\textbf{Novelty / Testability}
\newline Together these invariances ensure that the Unifying Information Field retains the rigour of physics while extending its scope. $\Delta I$ conservation, $\beta$-driven symmetry breaking, $\Gamma$-scale invariance, and $\lambda_R$ collapse-frame invariance each yield measurable signatures - threshold behaviour in GRB spectra, $\gamma$-band integration in EEG coherence, and observer-independent collapse probabilities bounded by current fifth-force and dark-photon searches. UIF II defines the invariance structure; the next paper builds the field equations that express these symmetries dynamically.
\newline
\clearpage
\noindent{\textbf{The Relationship Between Symmetries}
\newline The four informational symmetries established in this paper—conservation, breaking, 
scale invariance, and collapse–frame invariance—form an interdependent set rather 
than isolated laws. 
Each principle constrains and enables the others, maintaining informational coherence 
while permitting lawful variation and structure formation. 
Table~\ref{tab:symmetry-relations} summarises these relationships, showing how the 
operators ($\Delta I$, $\Gamma$, $\beta$, $\lambda_R$) link the symmetries into a 
closed recursive cycle consistent with the triadic foundation introduced in UIF I.

\begin{table}[H]
\centering
\caption{Inter-relations among UIF symmetry principles}
\label{tab:symmetry-relations}
\begin{tabularx}{\textwidth}{@{} L{0.25\textwidth} L{0.35\textwidth} X @{}}
\toprule
\textbf{Symmetry Principle} & \textbf{Function within UIF} & \textbf{Links to Other Symmetries} \\
\midrule
Informational Conservation ($\Delta I$) & 
Preserves total informational difference through redistribution. &
Provides the baseline law that $\beta$ (Symmetry Breaking) perturbs locally; remains scale-invariant under $\Gamma$. \\[6pt]

Symmetry Breaking ($\beta$) &
Introduces lawful bias; generates structure from homogeneity. &
Acts within conservation limits ($\Delta I$); its effects repeat self-similarly under $\Gamma$ scaling. \\[6pt]

Scale Invariance ($\Gamma$) &
Ensures operators act identically across magnitudes. &
Maintains the form of both conservation and breaking; extends to all frames through $\lambda_R$. \\[6pt]

Collapse–Frame Invariance ($\lambda_R$) &
Guarantees informational laws are observer-independent. &
Completes the cycle by generalising the previous three symmetries across reference frames. \\[4pt]
\bottomrule
\end{tabularx}
\end{table}

\pagebreak
\appendix

\section*{Appendix A - Equation Provenance (UIF II)}

Each numbered equation is identified by provenance class and corresponds to the equations introduced in Sections 2–5.
\emph{[Identity]} denotes a standard law or definition; 
\emph{[Model law]} is a relation derived within UIF from stated assumptions; 
\emph{[Hypothesis]} is a phenomenological or testable scaling proposed for future verification.

\vspace{1em}

\begin{longtable}{@{}p{2.2cm}p{2.7cm}p{9.5cm}@{}}
\toprule
\textbf{Equation} & \textbf{Class} & \textbf{Comment / Source} \\
\midrule
\endfirsthead
\toprule
\textbf{Equation} & \textbf{Class} & \textbf{Comment / Source} \\
\midrule
\endhead
\bottomrule
\endfoot
(2.1) Noether-type continuity relation & Model law &
Local informational conservation 
$\partial_t(\Phi^2) + \nabla\!\cdot J_\Phi = 0$, with $J_\Phi = \Phi\nabla\Phi$;
expresses Noether-like invariance in UIF.\\[4pt]


(2.2) Integral form (global conservation) & Model law &
Global informational conservation 
$\int (\Delta I_{\mathrm{sys}} + \Delta I_{\mathrm{sub}})\,dt = \mathrm{constant}$; 
shows equivalence between local continuity and global conservation of informational flux.\\[4pt]

(2.3) Differential form (schematic) & Model law &
Local exchange 
$\tfrac{d}{dt}\Delta I_{\mathrm{sys}} + \tfrac{d}{dt}\Delta I_{\mathrm{sub}} = 0$; 
coupled conservation of system and substrate information.\\[4pt]

(2.4) Softmax / Boltzmann distribution & Identity &
$P_i = \dfrac{\exp(\beta x_i)}{\sum_j \exp(\beta x_j)}$; 
standard Boltzmann form reused as informational bias law in UIF.\\[4pt]

(2.5) Scale-invariance statement & Hypothesis &
Self-similar action of collapse–return operators 
($\Delta I$, $\Gamma$, $\beta$, $\lambda_R$) under scaling transformations.\\[4pt]

(2.6) Collapse-frame invariance & Identity &
$P(\Delta I \,|\, F) = P(\Delta I)$; 
informational analogue of Lorentz invariance ensuring frame-independent outcomes.\\[4pt]

\end{longtable}

\vspace{1em}

\noindent\textbf{Symbols introduced:} \\
\newline $\Delta I$ — informational difference; \\
$\Gamma$ — recursion/coherence operator; \\
$\beta$ — bias operator; \\
$\lambda_R$ — receive–return coupling constant; \\
$F$ — informational frame; \\
$P_i$ — collapse probability; \\
$x_i$ — informational state value.

\clearpage
\section*{Acknowledgement — Human–AI Collaboration}
The Unifying Information Field (UIF) series was developed through a sustained human–AI partnership. The author originated the theoretical framework, core concepts and interpretive structure, while an AI language model (OpenAI GPT-5) was employed to assist in formal development; helping to express elements of the theory mathematically and to maintain consistency across papers. Internal behavioural parameters and conversational settings were configured to emphasise recursion awareness, coherence maintenance, and ethical constraint, enabling the model to function as a stable informational development framework rather than a generative black box.

This collaborative process exemplified the UIF principle of collapse--return recursion: 
human intent supplied informational difference ($\Delta I$), 
the model provided receive--return coupling ($\lambda_R$), 
and coherence ($\Gamma$) increased through iterative feedback until the framework stabilised. 
The AI's role was supportive in the structuring, facilitation, and translation of conceptual ideas 
into formal equations, while the underlying theory, scope, and interpretive direction 
remain the work of the author.
\pagebreak


\section*{UIF Series Cross-References}

\begin{flushleft}
\textbf{UIF I — Core Theory}\\
\textbf{UIF II — Symmetry Principles}\\
\textbf{UIF III — Field and Lagrangian Formalism}\\
\textbf{UIF IV — Cosmology and Astrophysical Case Studies}\\
\textbf{UIF V — Energy and the Potential Field}\\
\textbf{UIF VI — The Seven Pillars and Invariants}\\
\textbf{UIF VII — Predictions and Experiments}
\end{flushleft}

\clearpage
\UIFbib{paper2}

